% $Id: front.tex,v 1.1 1992/05/10 19:41:50 rz Exp rz $

\title{Effective Polynomial Computation \\ (second edition)}
\author{Richard Zippel \\ Google, Inc.}

\date{\today}

\maketitle

\pagenumbering{roman}
\setcounter{page}{5}

\cleardoublepage

\tableofcontents

\makeatletter\chapter*{Preface\@mkboth {Preface}{Preface}}
\makeatother

What distinguishes computer algebra from other areas of mathematical
computation is the attention paid to accurately modeling the
mathematical semantics of the objects being manipulated.  The first
manifestation of this is that while conventional computation systems
typically define an integer to be a whole number with absolute value
less then $2^{31}$ or $2^{63}$, algebraic manipulation systems define
integers to be whole numbers with no limit set to their size.  Though
this distinction makes little difference in many computations,
algebraic manipulations focuses on techniques and problems where exact
computation or precise modeling of a complex mathematical concept is
important.

This book considers algorithms that operate on polynomials: algorithms
for polynomial arithmetic, for computing the greatest common divisor
of several polynomials and for factoring polynomials.  Our emphasis is
on {\em effective} computation.  Simple, clear and easy to implement
algorithms are preferred to those that are slightly quicker, but more
opaque.  The behavior of algorithms in practice is of more concern
than their \key{worst case asymptotic complexity}.

Worst case asymptotic complexity is not a very sharp scalpel for
discriminating the behavior between different algorithms.  For
instance, both polynomial and exponential time algorithms for
factoring univariate polynomials with integer coefficients exist, but,
on any machine available today, I know of no polynomial that can be
factored faster with a polynomial time algorithm than with the best
exponential time algorithms.\index{factorization! of polynomials}

Many of the algorithms used in polynomial computation have analogs
that are fundamental techniques in computational number theory and
many of the issues that arise in the analysis of the polynomial
algorithms have their roots in number theory.  So, this text begins
with an examination of the parts of number theory that I have found
most useful and stimulating.

One of the problems I have found teaching computer algebra is that
students typically have one of two different backgrounds:
\begin{itemize}
\item A mathematics background and are familiar with much of the
commutative algebraic and number theory that is used in computer
algebra, but they are unfamiliar with programming and computational
issues and are uncomfortable analyzing computational costs.
\item A computing background, but are unfamiliar with the commutative
algebra and number theory required. 
\end{itemize}
Beginning a computer algebra course with number theory provides plenty
of examples for the mathematical students to learn programming and
sufficient motivation (and time) for the students with a computing
background to learn the requisite mathematics.

The material in this book has been used for one semester courses in
computer algebra at Cornell University and at the Massachusetts
Institute of Technology.  Although there is a great deal of additional
material in computer algebra that I would like to cover in a computer
algebra course, \eg, manipulation of algebraic functions, Gr\"{o}bner
bases, multivariate resultants, computations in algebraic geometry,
integration of transcendental and algebraic functions and qualitative
methods in ordinary and partial differential equations, I find it
difficult to thoroughly cover much more than what is in this book in
one semester.  Many of the topics mentioned above are deep and rich
enough in themselves to deserve their own book, and in some cases such
books are now appearing.  In some cases, their inclusion dramatically
increases the mathematical prerequisites.

For much of the material in this book, only the most basic
understanding of modern algebra is required.  I have tried to define
most of the algebraic terms when they are first encountered.  The first
semester of most undergraduate modern algebra courses or of the
typical computer science ``discrete mathematics'' course suffices.  In
a few places, far deeper mathematics is used, in particular in
\chapref{Irred:Chap}.  Where it is necessary to use more
advanced mathematical principles I have tried to give the reader
appropriate references.  Throughout, if the reader is willing to
accept a theorem without proof, those sections can be skipped.

\paragraph{Acknowledgements}

This book has been in development for a very long time.  It started
out as a much more comprehensive study of computer algebra and has
gradually been refined to the present form.  Numerous students at
different institutions have suffered through the earlier versions of
this book and their comments have been very useful.  I'd especially
like to acknowledge the numerous comments and suggestions from Eric
{\BachE}, James {\DavenportJ}, Richard {\Fateman}, Patrizia {\Gianni},
Carl Hoffman,\index{Hoffman, Carl W.} Susan {\LandauS}, Jonathan
Rees,\index{Rees, Jonathan A.} and Barry {\Trager}.

This book would not be here today were it not for the constant
encouragement, support and prodding of a number of close colleagues
and friends: Hal Abelson,\index{Abelson, Harold} Nancy
Bell,\index{Bell, Nancy} Patrizia Gianni,\index{Gianni, Patrizia}
Juris Hartmanis,\index{Hartmanis, Juris} Dick Jenks,\index{Jenks,
Richard} Jacob Katzenelson,\index{Katzenelson, Jacob} Pat
Musa,\index{Musa, Pat} Fred Schneider,\index{Schneider, Fred B.} Jon
and Rachel Sieber,\index{Sieber, Jon}\index{Sieber, Rachel} Gerry
Sussman,\index{Sussman, Gerald Jay} and Barry {\Trager}.  To all of
you, thank you very much.

For the support I have received while at the Massachusetts Institute
of Technology and Cornell University, and through contracts from the
Advanced Research Projects Agency of the Department of Defense, the
Office of Naval Research, the Army Research Office and the National
Science Foundation, I am most grateful.

\noindent
\begin{tabular*}{\textwidth}{@{}l@{\extracolsep{0pt plus30pc}}r@{}}
& {\em Ithaca, April 1993}
\end{tabular*}


