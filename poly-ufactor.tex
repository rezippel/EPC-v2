%$Id: poly-factor.tex,v 1.1 1992/05/10 19:41:26 rz Exp rz $
\chapter{Univariate Factorization}
\label{UFactoring:Chap}

This chapter discusses the factorization of univariate polynomials
over the rational integers.  For many years this problem has been
attacked by a combination of the Hensel lifting techniques and one of
the algorithms for factorization over finite fields.  For many
problems this has been quite effective, but in the worst case this
approach can take exponential time.  In 1982, A.~K.~{\LenstraA},
H.~W.~{\LenstraH} and {\Lovasz} \cite{LenstraAK82} developed an
algorithm for factoring univariate polynomials over $\Z$ in polynomial
time by using lattice basis reduction.  Suddenly, the complexity of a
number of other problems was reduced to polynomial time, \eg,
\cite{Landau85a}.  However, experience with actual implementations has
shown\Marginpar{Need than references} that the lattice reduction
techniques are not faster than the older exponential time techniques
for any factorization problem that can be completed in reasonable time
today.

\sectref{UF:Reductions:Sec} discusses some simple reductions of the
factorization problem, make the polynomial square free, etc., that can
be used independent of the algorithm used for the core of the
factorization problem.  In \sectref{UF:Simple:Sec} we discuss the
classical, exponential time factorization algorithm, which is probably
still the optimal approach.  Finally, in \sectref{UF:LLL:Sec} we
discuss the asymptotically good algorithms that make use of lattice
reduction techniques. 

\section{Reductions}
\label{UF:Reductions:Sec}

When factoring a univariate polynomial, 
\[
F(Z) = f_0 Z^d _+ f_1 Z^{d-1} + \cdots + f_d,
\]
over $\Z$, two simplifications are effective.  First, removing the
integer content of $F(Z)$ decreases the size of the coefficients of
$F(Z)$, and doesn't change the number of non-zero terms.  For all
univariate factorization algorithm known to this point, it is
worthwhile.

As with all univariate problems we define the size of a univariate
polynomial, $F(Z)$, to be $(\deg F)\cdot \|F\|_{\infty} = d
|F|$.\index{size, of univariate polynomial} Since the sum of the
degrees of the factors of $F$ is equal to the degree of $F$, the size
of the factorization of $F$ can only be larger than that of $F$
because of growth in the size of the coefficients.  Thus, by
\longpropref{Factor:CBound:Prop} the size of the factorization can
only increase by about a factor of $d$.  Furthermore, no factorization
of $F$, into irreducible components or not, is more than a factor of
$d$ larger than the size of $F$.  Thus it is worthwhile to compute a
\key{square free decomposition} of $F(Z)$.

Notice that one could also try to monicize the $F(Z)$:
\[
f_0^{d-1}F(Z) = (f_0 Z)^d + f_1 f_0 (f_0 Z)^{d-1} + \cdots + f_d
f_0^{d-1},
\]
but this could increase the size of the coefficients of $F$ by as much
as $f_0^{d-1}$.  Furthermore, neither of the two factoring algorithms
discussed benefit significantly from being restricted to monic
polynomials.

\section{Simple Algorithm}
\label{UF:Simple:Sec}

We assume that $F(Z)$ is a univariate polynomial over $\Z$ of degree
$d$, square free and primitive.  The classical univariate
factorization algorithm consists of three steps:
\begin{enumerate}
\item Choose a ``good'' random rational prime $p$ and factor $F(Z)$
into irreducible factors modulo $p$ 
\[
F(Z) = F_1(Z) F_2(Z) \cdots F_k(Z) \pmod{p}
\]

\item Use Newton's iteration to lift the $F_i$ to factors modulo
$p^{\ell}$,
\[
F(Z) = \hat{F}_1(Z) \cdot \hat{F}_2(Z) \cdots \hat{F}_k(Z)
\pmod{p^{\ell}}
\]

\item Combine the $\hat{F}_i$ as neeeded to into true divisors of $F$
over $\Z$.
\end{enumerate}

Each of these steps needs a a bit of clarification.  The behavior of
this algorithm is most unsatisfactory when the final stage needs to
try many combinations of the $\hat{F}_i$ to get true factors over
$\Z$. Thus the best primes in the first step, are those for which the
factorization of $F(Z)$ modulo $p$ is as close as possible to the
factorization of $F(Z)$ over $\Z$.  Since univariate factorization is
efficient, one usually tries several primes $p$ and picks the one
that gives the coarsest factorization.  In particular, one should
avoid primes for $F(Z)$ is not square free.  Depending on the
particular constants of the finite field factorization algorithm,
primes near the word size of the processor are usually chosen to
minimize the number of Newton steps required.  In current
implementations, between five and ten different primes may be tried.

The coefficients of the divisors of $F(Z)$ are bounded in absolute
value by
\[
B = \sqrt{d+1} \cdot 2^d |F|.
\]
Since the coefficients may either positive or negative $p^{\ell}$ must
exceed $2B+1$.  Finally, notice that the finite field factorization
will yield monic polynomials.  Thus the coefficients of $\hat{F}$ are
images of $f_i/f_0$.  In order to recover both the numerator and
denominator we need an additional factor of $2$.  So $\ell$ needs to
be chosen such that
\[
p^{\ell} > 2(2B+1).
\]

Either the system of equations formulation or the Zassenhaus
formulation can then be used to compute $\hat{F}_i(Z)$ from the
$F_i(Z)$.  The Zassenhaus formulation is often chosen because there is
no need to worry about sparsity and since the overhead of the
Zassenhaus formulation is often smaller in this case.

The most time consuming step is the last one.  It actually has two
parts, once we have a potential factor of $F$ modulo $p^{\ell}$, we
need to convert it to a factor over $\Z$ and then we have to perform
the actual test divisions.

The factors modulo $p^{\ell}$ are monic.  One way to convert them to
potential divisors over $\Z$ is to multiply the divisor by the leading
coefficient of $F$, $f_0$ and convert to a a balanced representation.
The primitive part of the resulting polynomial is the potential
divisor.  If the absolute value of any of the coefficients is larger
then the coefficient bound then there is no need to even attempt the
trial division.

While trial division was also used in the sparse modular {\sc gcd}
algorithm, there it is expected that most trial divisions will be
successful.  With polynomial factorization, the opposite is true.
Very often the trial division will be unsuccessful.  The following
observation can be used to significantly minimize the cost of trial
divisions.  Note that if the polynomial $G(Z)$ divides $F(Z)$ then,
for all integers $r \in \Z$, $G(r)$ divides $F(r)$.  There is no point
in performing a polynomial trial division if this division fails.
Experience has shown \cite{Abbot85} that using $r= 0$ and $r=1$ will
often obviate the need to perform any polynomial test divisions.

\section{Asymptotically Good Algorithms}
\label{UF:LLL:Sec}

To illustrate the ideas behind the asymptotically good, lattice
reduction algorithm we present a simple factorization algorithm that
uses lattices in the same spirit as the Lenstra-Lenstra-Lov\'asz
algorithm discussed below.

Let $F(Z)$ be a polynomial over $\Z$:
\[
F(Z) = f_0 Z^d + f_1 Z^{d-1} + \cdots f_d,
\]
and let $\alpha$ be an approximation of some zero of $F(Z)$ that has
been compute to some precision.  The minimal polynomial for $\alpha$
is an irreducible polynomial that divides $F(Z)$, say $G(Z)$.
Repeating this process with $F(Z)/G(Z)$ gives the factorization of
$F(Z)$.

The minimal polynomial for $\alpha$ is found by first try to find a
linear polynomial satisfied by $\alpha$, then a quadratic and so on.
To determine if there is a polynomial of degree $k$ that is satisfied
by $\alpha$, we create the $k+1$ dimensional lattice ${\scr L}_k$ that
has as basis vectors:
\[
(\alpha^k, 0, \ldots, 0), \quad(0, \alpha^{k-1}, 0, \ldots, 0), 
\quad\ldots, \quad(0, \ldots, 0, 1).
\]
The Lov\'asz basis reduction algorithm of \sectref{} can then be used
to find a small vector in ${\scr L}_k$, \ie, rational integers $g_0,
\ldots, g_k$ such that 
\[
\left|g_0 \alpha^k + g_1 \alpha^{k-1} + \cdots + g_k\right| =
\epsilon_k
\]
is small.  If $\epsilon_k$ is sufficiently small then the zero that
$\alpha$ represents is actually a zero of
\[
G(Z) = g_0 Z^k + g_1 Z^{k-1} + \cdots + g_k
\]
and $G(Z)$ is a divisor of $F(Z)$.


