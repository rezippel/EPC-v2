\chapter{Subfield Determination}
\label{Subfield:Chap}



\section{Cyclic Subfields}

\begin{figure}
\[
\begin{diagram}
\node{L} \arrow{s,-} \arrow{e,-} \node{B} \arrow{s,-} \\
\node{E}\arrows{s,-} \arrow{e,-}\node{A} \arrow{s,-} \\
\node{\Q} \arrow{s,-} \node{\Z}
\end{diagram}
\]
\end{figure}

Let $L$ be an algebraic number field over $\Q$.  Assume $L$ has a
subfield $E$ which is a quadratic extension of $\Q$.  Let $A$ be the
ring of integers of $E$ and $B$ be the ring of integers of $L$.  Then
$E$ is generated by $\sqrt{m}$.  Let the primes that divide $m$ be
$p_i$.  The ideals associated with these $p_i$ are ${\frak p}_i$.  The
principal ideal $(m)$ in $\Z$ must ramify in $A$ and thus in $B$ since
$(m) = (\sqrt{m})^2$.  Let ${\frak P}_j$ be the primes of $B$ that
divide $(m)$.  Since each ${\frak P}_j$ lies over precisely one
${\frak p}_i$, each ${\frak p}_i B$ is divisible by some ${\frak P}_j$
and $p_i$ will be ramified at each prime ideal that lies over it.

Consider the polynomial $f(x) = x^4 - 26 y^2 + 1$.
We denote by $L$ the quartic number field it generates, $B$ is
its ring of integers.  We are looking for a quadratic subfield
$E$ with a ring of integers $A$.  Its discriminant is $2^{14}\,3^2 \,7^2$,
so the possible candidates for ramification are 2, 3 and 7.  Factoring
$f(x)$ modulo 2, 3 and 7 we get: 
\[
(y + 1)^4,\qquad         (y - 1)^2 (y + 1)^2,\qquad         (y^2 + 1)2
\]
respectively.  We now know that the primes 2, 3 and 7 all ramify
globally in $B$ ($(2) = {\frak p}_1^4$, 
$(3) = {\frak p}_2^2\,{\frak p}_3^2$, $(7) = {\frak p}_4^2$).
Factoring $f(x)$ in the quadratic fields $\Q[\sqrt{2}]$, $\Q[\sqrt{2}]$,
$\Q[\sqrt{3}]$, $\Q[\sqrt{7}]$, $\Q[\sqrt{6}]$, $\Q[\sqrt{14}]$ and
$\Q[\sqrt{21}]$ we get the following factorizations:
\[
\begin{aligned}
f(x) &=(x^2 - 2\sqrt{6} x - 1) (x^2 + 2\sqrt{6} x - 1)\\
&=(x^2 - 2\sqrt{7} x + 1)\ (x^2 + 2\sqrt{7} x + 1)
\end{aligned}
\]
In all the other fields the $f(x)$ is irreducible.  From this we know that
$L$ contains $\Q[\sqrt{6}]$ and $\Q[\sqrt{7}]$.  The compositum of these
two fields is of degree four (by Kummer theory) and is contained in $L$ so
it must be equal to $L$.  So $L = \Q[\sqrt{6}]\otimes_{\Q}\Q[\sqrt{7}]$.
In this case all we had to do is determine the subfields: it is not always
so easy.

Reviewing what we have done we can state the following
proposition: 

\begin{proposition}
Let $p(x)$ be a polynomial irreducible over the integers with
discriminant $\Delta$, which generates the field $L$.  If $L$ can be
written as $K\otimes_{\Q}\Q[\root r \of{m}]$, then each prime $p$
that divides m must satisfy the following two conditions (1) $p$
divides $\Delta$ and (2) $p(x)$ must be a perfect $r$-th power modulo
$p$.
\end{proposition}

\section{General Case}

Let $E$ be a field and $\alpha$ a primitive element of a finite
separable algebraic extension of $E$, $E[\alpha]$.  In this section we
will show how to determine the proper subfields of $E[\alpha]$.  The
bulk of this work appeared in Landau's thesis \cite{Landau85b}.

Denote the minimal polynomial of $\alpha$ over $E$ by $f(Z)$ and let
$L$ be the splitting field of $E[\alpha]$.  There is a one to one
correspondence between the lattice of fields between $L$ and $E$ and
the subgroups of the Galois group of $L/E$, $G$.  There is also a one
to one correspondence between the subgroups of $G$ and certain subsets
of the zeroes of $f(Z)$.  These correspondences are the basis of the
subfield determination algorithm discussed in this section.

\begin{figure}
\[
\begin{diagram}
\node{L} \arrow{s,-} \arrow{e,-} \node{e} \arrow{s,-} \\
\node{E[\alpha_1]} \arrow{s,-} \arrow{e,-} 
  \node{G_0} \arrow{s,-} \arrow{e,-} \node{ \{ \alpha_1 \}} \\
\node{K} \arrow{s,-} \arrow{e,-}
   \node{H} \arrow{s,-} \arrow{e,-} \node{\{ \alpha_i, \ldots, \alpha_j\}} \\
\node{E} \arrow{e,-} \node{G} \arrow{e,-} 
    \node{\{ \alpha_1, \ldots, \alpha_n \}}
\end{diagram}
\]
\caption{Subfield Determination\label{Subfield:Fig}}
\end{figure}

\figref{Subfield:Fig} illustrates these correspondences.  Denote the
set of conjugates of $\alpha$ by $\Omega = \{\alpha= \alpha_1,
\alpha_2, \ldots, \alpha_n\}$.  The Galois group of $L$ over $E$, $G$,
is the the set of all automorphisms of $L$ that leave $E$ fixed.
Writing an element of $L$ as a polynomial in the $\alpha_i$, the
effect of an automorphism of $G$ is to permute the $\alpha_i$.
However, not every permutation of the $\alpha_i$ need be an
automorphism.

$L$ is also Galois over $E[\alpha_1]$.  We denote by $G_{\alpha}$ the
Galois group of $L$ over $E[\alpha]$.  It consists of those
permutations in $G$ that leave $\alpha_1$ fixed.  The fundamental
theorem of Galois theory asserts that there is a one to one
correspondence between the fields between $E[\alpha]$ and $E$ and the
subgroups of $G$ that contain $G_{\alpha}$.

Associate with $G_{\alpha}$ the set $\{\alpha_1\}$.  Let $H$ be a
subgroup of $G$ that contains $G_{\alpha}$.  Let $A$ be a subset of
$\Omega$ that contains $\alpha_1$, with the property that for every
$\sigma \in H$, $\sigma A = A$.  The smallest subset of $\Omega$ with
this property is the {\em block} associated with $H$.  Thus the block
associated with $G$ is $\Omega$ (since $G$ is transitive).  This block
is the subset of the roots of $f(Z)$ that corresponds to the
intermediate field $K$.  Notice that $K$ is generated by the
symmetric functions on the elements of the block.  In the next
section we will make precise the definition of a block of roots of
$f(Z)$ and show that each block containing $\alpha_1$ corresponds to a
group between $G_{\alpha}$ and $G$.

\subsection{Permutation Groups}
Throughout this section we assume that $\Omega$ is a set of $n$
distinct symbols and that $G$ is a group of permutations on these
symbols.  We can think of $\Omega$ as the set of zeroes of an
irreducible separable polynomial.  If $\Delta$ is a subset of $\Omega$
we define $G_{\Delta}$ to be the set of permutations that fix each
element of $\Delta$, \ie
\[
G_{\Delta} = \{ \sigma \in G \mid \forall \delta \in \Delta.
\sigma(\delta) = \delta\}.
\]
If $\Delta$ consists of the single element $\alpha$ then we also
denote $G_{\{\alpha\}}$ by $G_{\alpha}$.  This matches the notation
used earlier.

A subset  $\psi \in \Omega$ is called a {\em block} of $G$ if for
each $\sigma \in G$ either $\sigma(\psi)  \subset \psi$ or $\sigma(\psi)
\cap \psi = \phi$.  The blocks $\{\alpha_i \}$ and $\Omega$ are
called {\em trivial blocks\/}.

For example, let $G$ be the cyclic group $C_6$, $\sigma$ a generator
of $G$ such that $\sigma(\alpha_i) = \alpha_{i+1}$, $\sigma(\alpha_6)
= \alpha_1$.  The sets $S_1 = \{\,\alpha_1, \alpha_4\,\}$ and $S_2 =
\{\,\alpha_1, \alpha_3, \alpha_5\,\}$ are non-trivial blocks.  Notice
that $\sigma^i S_1 = \{\, \alpha_{1+i}, \alpha_{4+i}\,\}$ so $\sigma^i
S_1 = S_1$ if and only if $i = 3$ or $0$, and $\sigma^i S_1 \cap S_1 = \phi$
otherwise.

The following proposition establishes the correspondence between
subgroups and blocks mentioned earlier.  It is the key result on which
the subfield determination algorithm rests.  The proof is adapted from
Wielandt \cite{Wielandt64}.

\begin{proposition}\label{Group:Zeroes:Prop}
The lattice of groups between $G_{\alpha}$ and $G$ is isomorphic to
the lattice of blocks of $G$ that contain $\alpha$.
\end{proposition}

\begin{proof}
Let $H$ be a group that lies between $G_\alpha$ and $G$, and let
$\psi$ be the orbit of $\alpha$ under $H$, 
\[
\psi = \{ \sigma \alpha \mid \sigma \in H \}.
\]
We claim $\psi$ is the block corresponding to $H$.  First we show that
$\psi$ is a block.  It is clear that $\sigma \psi = \psi$ for all
$\sigma \in H$.  We want to show that for all other elements of $\tau
\in G$, $\psi \cap \tau \psi = \phi$.  Let $\beta$ be an element of
$\psi \cap \psi^{\tau}$, where $\tau$ is an element of $G$.  By the
definition of $\psi$ there is an element $\sigma \in H$ such that
$\beta = \sigma(\alpha)$.  Since $\beta$ is also an element of
$\psi^{\tau}$, $\beta = \sigma'(\tau(\alpha))$, where $\sigma' \in H$.
Therefore, $\alpha$ is fixed by $\sigma^{-1} \circ \tau \circ \sigma'$
which is an element of $G_{\alpha} \subset H$.  Since $\sigma$ and
$\sigma'$ are elements of $H$, so must be $\tau$.

Next we show that each non-trivial block of $G$ that contains $\alpha$
corresponds to a subgroup of $G$.  Let $\psi$ be such a block and let
$H = \{\sigma \mid \sigma(\psi) \subset \psi\}$.  $H$ is clearly a
subgroup of $G$.  Because $\psi \varsubsetneq \Omega$ and $G$ is
transitive on $\Omega$ $H \varsubsetneq G$.  $G_{\alpha}$ is contained
in $H$ since $\alpha \in \psi$.  By assumption, $\psi$ contains a $\beta \not=
\alpha$.  Because $G$ is transitive there exists $\tau$ such that
$\tau(\alpha) = \beta$.  But $\alpha$ and $\beta$ are elements of
$\psi$, so $\tau \in H$ and is not an element of $G_{\alpha}$.  Thus
\[
G_{\alpha} \varsubsetneq H \varsubsetneq G.
\]
\end{proof}


\subsection{Playing with Blocks}

We continue the notation of the previous few paragraphs, but add the
assumption that $E$ has property ${\cal F}$.
Consider the factorization of $f(Z)$, the minimal polynomial of
$\alpha$ over $E[\alpha]$:
\begin{equation}
\label{Minpoly:F:Eq}
f(Z) = f_1(Z, \alpha) \times f_2(Z, \alpha) \times \cdots 
   \times f_s(Z, \alpha).
\end{equation}
Without loss of generality we can assume that $f_1(Z, \alpha) = Z -
\alpha$.  Each conjugate of $\alpha$ is a zero of $f(Z)$, so the
factorization in \eqnref{Minpoly:F:Eq} has separated the conjugates of
$\alpha$ into disjoint sets where $\psi_i$ denotes the set of zeroes
of $f_i(Z, \alpha)$.  We can assume that the conjugates have been
numbered so that $\alpha_i$ is an element of $\psi_i$, for $1 \le i
\le s$.

Because $\psi_1$ contains only the single element, $\alpha_1$, we call
$\alpha_1$ {\em isolated\/}.  We have an explicit representation of
the subgroup of $G$ which sends $\alpha_1$ to each of the isolated
$\alpha_i$.  The problem now divides into two cases: (1) only
$\alpha_1$ is isolated, or (2) $\alpha_1, \ldots, \alpha_k$ are
isolated.  

The second case is quite easy.  Let $H$ denote the group generated by
the isolated zeroes of $f(Z)$.  If $H$ is a proper subgroup of $G$ then the
fixed field of $H$ is a subfield of $G$.  If not, determine if there
is a proper subgroup of $H$ by examination of its operation table.
This is discussed in a bit more detail in the presentation of the
algorithm.

It is the first case that takes some effort.  Throughout the rest of
this section, we assume that $\alpha_1$ is the only isolated zero of
$f(Z)$.

Let $\sigma$ be an element of $G_{\alpha}$ which is the Galois group
of $L/E[\alpha]$.  Since the coefficients of each of the $f_i(Z,
\alpha)$ are in $E[\alpha]$ they are fixed by $\sigma$ and thus the
effect of $\sigma$ is to permute the elements of each of the $\psi_i$
independently.  In other words, the $\psi_i$ are blocks of
$G_{\alpha}$.  

We want to construct a block $\Psi$ that contains $\alpha_1$ and at
least one other element $\alpha_i$.  The group corresponding to
$\Psi$, which we denote by $H$, will lie between $G_{\alpha}$ and $G$
and thus its fixed field ($K$) will be a proper subfield of
$E[\alpha_1]$.  The following proposition shows that if $\Psi$
includes $\alpha_i$ then $\Psi$ must include all the other zeroes of
$f_i(Z, \alpha_1)$.

\begin{proposition}
Let $E$ be a field, $E[\alpha]$ a separable extension of $E$ and
$f(Z)$ the minimal polynomial of $\alpha$.  Let $g(Z)$ be an
irreducible factor of $f(Z)$ over $E[\alpha]$.  If one zero of $g(Z)$
lies in a block containing $\alpha$ then all the zeroes of $g(Z)$ lie
in that block.
\end{proposition}

\begin{proof}
Let $\psi$ be a block containing $\alpha$.  Denote by $H$ the Galois
group corresponding to $\psi$ and $K = E[\alpha]^H$, the fixed field
of $H$.  The coefficients of
\[
h(Z) = \prod_{\alpha \in \psi}\left(Z - \alpha\right),
\]
are fixed by $H$ and lie in $K$.  Since $g(Z)$ is irreducible over
$E[\alpha]$ it is irreducible over $K \subseteq E[\alpha]$.  Since
$g(Z)$ and $h(Z)$ have a common factor, $g(Z)$ must divide $h(Z)$.
Thus all the zeroes of $g(Z)$ are zeros of $h(Z)$ and are elements of
$\psi$
\end{proof}

Since $G_{\alpha}$ fixes $\alpha$ and $G_{\alpha} \varsubsetneq H$,
$H$ must contain an automorphism of $L$ that sends $\alpha_1$ to some
other zero of $f(Z)$, say $\alpha_{\ell}$.

The factorization of $f(Z)$ over $E[\alpha_1]$ and $E[\alpha_\ell]$
are
\[
\begin{aligned}
f(Z) & = f_1(Z, \alpha_1) \times f_2(Z, \alpha_1) \times \cdots
  \times f_s(Z, \alpha_1), \\
     & = f_1(Z, \alpha_{\ell}) \times f_2(Z, \alpha_{\ell}) \times \cdots
  \times f_s(Z, \alpha_{\ell}), 
\end{aligned}
\]
respectively.  Denote the set of zeroes of $f_i(Z, \alpha_1)$ by
$\psi_i$ and those of $f_j(Z, \alpha_{\ell})$ by $\phi_j$.  

Since $K \subseteq E[\alpha_1] \cap E[\alpha_{\ell}]$, $\Psi$ is both
the union of some of the $\psi_i$ and the union of some of the
$\phi_j$.  If $\Psi$ contains $\psi_i$ and $\phi_k \cap \psi_i$ is not
empty, then $\Psi$ also contains $\phi_k$.  Similarly, if $\psi_j \cap
\phi_k$ is not empty then $\psi_j$ is contained in $\Psi$.  

If we knew the $\psi_i$ and $\phi_j$ explicitly, we could find the
admissible blocks as follows:  Form a bipartite graph whose vertices
are  $\{\psi_i\} \cup \{\phi_j\}$, and where there is an edge between
$\psi_i$ and $\phi_j$ if and only if $\psi_i \cap \phi_j$ is not empty.
Since $f(Z, \alpha_1) = Z - \alpha_1$.  The $\psi_i$ that are in the
component connected to $\psi_1$ must all be contained in $\Psi$.
All that is left at this point is to show that $\Psi$ is a block.

\subsection{The Algorithm}

\begin{figure}
\small
\noindent{\bf Algorithm A:} Given $f(Z)$, the minimal polynomial
for $E[\alpha]$, return $h(Z, \alpha)$, whose roots generate a
non-trivial block of $\Lambda$ and whose coefficients generate a
subfield of $E[\alpha]$.
\begin{enumerate}
\item Let $\alpha$ be a zero of $f(Z)$.  Factor $f(Z)$
over $E[\alpha]$: 
\[
f(Z) = f_1(Z, \alpha) f_2(Z, \alpha) \cdots f_s(Z, \alpha),
\]
where $f_1(Z, \alpha) = (Z - \alpha)$.
Throughout this algorithm $\alpha_i$ denotes a zero of $f_i(Z, \alpha)$.
Notice that, in general, the minimal polynomial of $\alpha_i$ over
$E[\alpha_j]$ is $f_i(Z, \alpha_j)$.
\item If $f(Z)$ has only two factors then $E[\alpha]$ has no proper
subfields.  Return $f(Z)$.
\item If $E[\alpha]$ is normal over $E$, then the factorization gives an
explicit representation of the Galois group of $L = E[\alpha]$ over $E$.
Choose a proper subgroup $H \varsubsetneq G(L/E)$ and return
\[
h(Z, \alpha) = \prod_{\sigma \in H} (Z - \sigma(\alpha)).
\]
\item If $f(Z)$ has more than one linear factor over
$E[\alpha]$, but is not normal, then return the product of the linear
factors.
\item For each $\ell$, $1 < \ell \le k$ perform steps $6$ and $7$,
returning the $h_{\ell}(Z, \alpha)$ that has the smallest degree.
\item Construct a graph, $G_{\ell}$,  that has vertices $\{\psi_1,
\ldots, \psi_s, \phi_1, \ldots, \phi_s\}$ where there is an edge between
$\langle \psi_i, \phi_j \rangle$ if and only if the GCD of $f_i(x,
\alpha_1)$ and $f_j(x, \alpha_{\ell})$ over 
$E[\alpha_1, \alpha_{\ell}]$.  This requires computing a primitive element.
\item Let $Y_{\ell}$ be the connected component of $G_{\ell}$ that
includes $\psi_1$. Set
\[
h_{\ell}(Z, \alpha) = \prod_{\psi_i \in Y_{\ell}} f_i(Z, \alpha).
\]
\end{enumerate}
\caption{Subfield Determination Algorithm \label{Alg:A:Fig}}
\end{figure}

The first step of the algorithm is to factor $f(Z)$ over
$E[\alpha]$:
\[
f(Z) = f_1(Z, \alpha)\times f_2(Z, \alpha)\times 
   \cdots \times f_s(Z, \alpha).
\]
Since we know that $(Z - \alpha)$ divides $f(Z)$, we need only factor
$f(Z)/(Z - \alpha)$ over $E[\alpha]$, which may be somewhat easier in
practice.  This factorization can be done by the techniques discussed
in \sectref{Algebraic:Factoring:Sec}. 

Step $2$ of algorithm {\bf A} identifies a simple case where there can
be no proper subfields.  If $f(Z)$ has only two factors then there are
no coarser non-trivial factorizations of $f(Z)$ and there cannot be
any proper subfields of $E[\alpha]$.

Steps $3$ and $4$ correspond to cases of more than one isolated zero
of $f(Z)$.  Let the factorization of $f(Z)$ be
\[
f(Z) = (Z - \alpha) \times (Z - p_2(\alpha)) \times \cdots
  \times (Z - p_{\ell}(\alpha)) \times f_{\ell+1}(Z, \alpha) \times
\cdots \times f_{s}(Z, \alpha),
\]
where $f_{\ell + 1}, \ldots, f_s$ are irreducible polynomials of degree greater
than $1$ in $Z$.
The automorphisms $\sigma_i: \alpha \mapsto p_i(\alpha)$ of
$E[\alpha]$ over $E$ form a group of $\ell$ elements.  The product of two
elements of $G$ can be computed as
\[
\begin{aligned}
\sigma_i \sigma_j & = p_i(p_j(X)) \pmod{\hbox{f}(Z)} \\
& = p_k(X) = \sigma_k.
\end{aligned}
\]



If $\ell < s$ then $H = \{\sigma_i\}$ is a proper subgroup of $G$.
This is the case considered in step 4.
If $E[\alpha]/E$ is normal, as is step 3, then $\ell = s$.
Otherwise, the elements of order less than $n$ each generate proper
subgroups of $G$.  Let $H$ be any such subgroup of $G$ and define the
polynomial
\[
h(Z, \alpha) = \prod_{\sigma \in H} \left(Z - \sigma(\alpha)\right).
\]
The elements of of $H$ merely permute the zeroes of $h(Z,
\alpha)$, so the coefficients of $h(Z, \alpha)$, which are
the symmetric functions in the zeroes are left fixed by $H$ and thus
lie in, and generate $K = E[\alpha]^H$, a proper subfield of
$E[\alpha]$.


Steps $5$, $6$ and $7$ implement the case where only $\alpha_1$ is
isolated. 

\medskip
The cost of the algorithm lies almost entirely with the factorization
of $f(t, Z)$ over $E[\alpha]$.  Assume the degree of $f(t, Z)$ is $n$
in $Z$.  Using {\Kronecker}'s techniques for factoring over algebraic
extensions \cite{Trager76a}, the computation in step $1$ is
essentially the factorization of a bivariate polynomial of degree
$n^2$.  In principle this factorization can be computed in polynomial
time \cite{Landau85a,LenstraAK83b,LenstraAK87}; however, in practice
exponential time algorithms are faster.



