%$Id: interp.tex,v 1.1 1992/05/10 19:42:22 rz Exp rz $
\chapter{Real Zeroes of Polynomials}
\label{Sturm:Chap}

By extracting
certain sign information from the remainder sequence we can compute a
\keyi{Sturm sequence}, which can be used to isolate the zeroes of a 
polynomial.  Sturm sequence computations are described in
\sectref{Sturm:Seq:Sec}.




In this section and the next we will consider various relationships
between the zeroes of polynomials and their coefficients.  Most
directly, these relationships permit us to develop algorithmic
techniques for determining the zeroes of a polynomial over the reals.
This section discusses techniques that are based on the polynomial
remainder sequences that have just been discussed.  

\section{Descartes' Rule of Signs}
\label{Sturm:Descartes:Sec}

We write $p(x)$ as
\begin{equation}
\label{p:Definition:Eq}
p(x) = p_{0} x^{d} + p_{1} x^{d-1} + \cdots + p_{d}.
\end{equation}
Denote the zeroes of $p(x)$ by $\alpha_1, \ldots, \alpha_d$ where
multiple zeroes have been repeated.  We can write $p(x)$ as a product
of linear factors
\[
p(x) = p_0 (x - \alpha_1) (x - \alpha_2) \cdots (x - \alpha_d).
\]
By expanding this equation and comparing coefficients of powers of $x$
with those of \eqnref{p:Definition:Eq} we have
\begin{equation}
\label{Symmetric:Funcs:Eq}
\begin{aligned}
\alpha_1 + \cdots + \alpha_d & = - \frac{p_1}{p_0} \\
\alpha_1 \alpha_2 + \cdots + \alpha_{n-1} \alpha_n & = \frac{p_2}{p_0} \\
\vdots & \\
\alpha_1 \alpha_2 \cdots \alpha_n & = (-1)^{n} \frac{p_n}{p_0}
\end{aligned}
\end{equation}

Let ${\cal P} = p_0, p_1, \ldots, p_n$ be a finite sequence of real
numbers.  We define the {\em number of sign changes}\index{sign
changes!in a sequence} of ${\cal P}$ as the number of $i$ for which
$p_i p_{i+1} < 0$.  The number of sign changes of the polynomial
$p(x)$ is the defined to be the number of sign changes in the sequence
of coefficients $p_0, \ldots, p_d$.\index{sign changes!in a
polynomial}

\begin {proposition}\label{PRS:SignChange:Propa}
Assume there exist precisely two sign changes in the sequence of real
numbers
\[
a_j, a_{j+1}, \ldots, a_k, a_{k+1},
\]
and that $a_j a_{j+1}$ and $a_k a_{k+1}$ are both less than zero, \ie,
the sign changes are at the beginning and ends of the sequence.  Also
assume $\alpha$ is a positive real number.  Then the sequence
\begin{equation}\label{PRS:SignSeq:Eq}
\alpha a_{j+1} - a_j, \alpha a_{j+2} - a_{j+1}, \ldots, 
\alpha a_{k+1} - a_k
\end{equation}
contains an odd number of sign changes and thus contains at least one
sign change. 
\end{proposition}

\begin{proof}
Since $a_j$ and $a_{j+1}$ have different signs and $\alpha$ is
positive, $\sign \alpha a_{j+1} - a_j = \sign a_{j+1}$.  Similarly, 
$\sign \alpha a_{k+1} - a_k = \sign a_{k+1}$.  The sign of $a_{j+1}$
and $a_{k+1}$ differ so the first and last elements of
\eqnref{PRS:SignSeq:Eq} are different.
\end{proof}

\begin{proposition}\label{PRS:SignChange:Propb}
Let $a_0, a_1, a_2, \ldots, a_n$ be a sequence of real numbers with
$C$ sign changes, and $\alpha$ a positive real number.  Then
\begin{equation} \label{PRS:SignSeq:Eqb}
\alpha a_0, \alpha a_1 - a_0, \alpha a_2 - a_1, \ldots, \alpha a_n -
a_{n-1}, - a_n
\end{equation}
has at least $C+1$ sign changes.
\end{proposition}

\begin{proof}
Assume the sign changes of \eqnref{PRS:SignSeq:Eqb} are between
$a_{\nu_i}$ and $a_{\nu_i+1}$ for $i = 1, \ldots, C$.  We can divide
\eqnref{PRS:SignSeq:Eqb} into $C+1$ subsequences as follows
\begin{equation}\label{PRS:SignSubSeq:Eq}
\begin{array}{ccccc}
\alpha a_0, & \alpha a_1 - a_0,& \ldots, &
 \alpha a_{\nu_1+1} - a_{\nu_1}, \\
\alpha a_{\nu_1+1} - a_{\nu_1}, &
  \alpha a_{\nu_1+2} - a_{\nu_1+1},& \ldots, &\alpha a_{\nu_2+1} - a_{\nu_2},\\
\vdots &\vdots & & \vdots \\
\alpha a_{\nu_{C-1}+1} - a_{\nu_{C-1}}, &
  \alpha a_{\nu_{C-1}+2} - a_{\nu_{C-1}+1}, &\ldots, 
   &\alpha a_{\nu_C+1} - a_{\nu_C}, \\
\alpha a_{\nu_C+1} - a_{\nu_C},& \alpha a_{\nu_C+2} - a_{\nu_C+1}, 
   &\ldots, &- a_n
\end{array}
\end{equation}

Notice that the last element of each line is the first element of the
following line.  This duplication of elements doesn't change the
number of sign changes.  We claim that the first and last elements of
each row have different signs.  Since 
\[
\sign a_0 \not= \sign a_{\nu_1+1} = \sign \alpha a_{\nu_1+1} -
a_{\nu_1}
\]
this is true of the first row.  Similar reasoning shows this is true
of the last row.  The intermediate rows each have a least one sign
change by \propref{PRS:SignChange:Propa}.
\end{proof}

Let $\alpha$ be a positive real zero of $P(X)$, so 
\[
P(X) = (\alpha - X) \cdot Q(X).
\]
Let the coefficients of $Q(X)$ be 
\[
Q(X) = a_n X^n + a_{n-1} X^{n-1} + \cdots + a_0,
\]
so,
\[
P(X) = - a_n X^{n+1} + (\alpha a_n - a_{n-1}) X^n + \cdots + \alpha
a_0.
\]
By \propref{PRS:SignChange:Propb}, the number of sign changes in the
coefficients of $P(X)$ exceeds the number of sign changes in $Q(X)$ by
at least $1$.  More generally, let
\[
P(X) = (X - \alpha_1) \cdots (X - \alpha_Z) Q(X)
\]
where $\alpha_1, \ldots, \alpha_Z$ are positive reals and $Q(X)$ is a
polynomial with no positive real zeroes.  The number of sign changes
in $Q(X)$ is non-negative.  If $C$ denotes the number of sign changes
in $P(X)$ then $C - Z$ is positive.  This suggests the following proposition.

\begin{proposition}[Descartes]
\label{Descartes:Sign:Prop}
Let $C$ denote the number of sign changes in the polynomial $p(x) \in
\R[x]$ and $Z$ the number of positive real zeroes.  Then $C - Z$ is a
non-negative even integer.
\end{proposition} 

\section{Sturm Sequences}
\label{Sturm:Seq:Sec}

