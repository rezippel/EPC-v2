\chapter{Primality Testing}
\label{Integer:Primality:Chap}

\index{primality testing} 
We want to determine if a rational integer, $N$, is a prime number,
and when it is provide a succinct proof.  The simplest and most direct
technique of checking the primeness of an integer is to divide it by
each of the prime numbers less than $\sqrt{N}$, or more easily by all
the odd integers less than $N$ (see Exercise~\ref{Primality:GCD:Ex},
for an improvement).  If none divide $N$ then the number is prime.
This technique takes $O(N^{{1 \over 2} + \epsilon})$ time, which is
also the time required to factor a number near $N$ using the simplest
possible algorithm.  Furthermore the proof that $N$ is a prime, is the
set of results from the trial divisions.  Thus the size of the proof
is also $O(\sqrt{N})$.

We can produce a much smaller proof of the primality of $N$ by considering
the group structure of $\Z/N\Z$.  The elements of $\Z/N\Z$ form a group of
$\phi(N)$ elements which is equal to $N - 1$ if $N$ is prime, otherwise it
is smaller.  Thus exhibiting an element of $\Z/N\Z$ of order $N-1$
demonstrates that $N$ is prime.  The order of every element of $\Z/N\Z$
divides $N-1$.  To show that $a \in \Z/N\Z$ has order $N-1$ we must
demonstrate that $a^{M} \not= 1\pmod{N}$, where $M$ is a divisor of $N-1$.
We should not try to compute $a^{M}\mod{N}$ for every $M$ that divides
$N-1$ since there could be too many of them.  Instead we use least common
multiples of the $M$'s that are less than $N-1$.  In particular, 
let the factorization of  $N-1$ be
\[
N-1 = p_1^{e_1} \cdots p_k^{e_k}.
\]
Then to demonstrate that order of $a$ is $N-1$ we need to show that 
\[
a^{(N -1)/ p_i^{f_i}} \not\equiv 1 \pmod{N}
\]
for all $p_i$ and for $1 < f_i \le e_i$.  Thus we need try only $e_{1}+
\cdots e_{k} \approx O(\log N)$ possible $M$'s.  Each exponentiation takes
about $O(\log N)$ operations using repeated squaring.  Thus a proof that
$a$ is of order $N-1$, and thus that $N$ is prime takes $O(\log^2 N)$
operations, and the resulting proof is of length $O(\log^3 N)$.

There are two problems with this approach.  First, we must factor $N-1$.
This is slightly easier than factoring $N$ since we know that $2$
divides $N-1$, but not a lot.  One way in which factoring $N-1$
might be easy is if $N-1$ is the the product of small primes.  In
this case $N-1$ is said to be {\em smooth\/}.\index{smooth numbers}  

Second, we must find a element of order $N-1$.  There is no
unconditional, direct technique for finding such elements quickly, as
this is being written (Ed: give results using Riemann hypothesis.)
Nonetheless this technique can be very effective when it does work.  It
is called the $N-1$ primality test.

However there are probabilistic approaches that are efficient.  They are
discussed in the following section.  \Marginpar{Need to say something about
{\Adleman}-Rumeley. Solovay-Strassen \cite{Solovay:Strassen}, Pratt
\cite{Pratt:Primality}}

\section{Modular Techniques}

The idea behind the basic algorithms for testing the primality of $N$
is to find an integer $a$ which is of order $N-1$ modulo $N$.  So the
key question is, How do you prove an integer has order $n-1$?  

If $a^m \not\equiv 1 \pmod{N}$ for every $m < N -1 $ then $a$ has order
$N-1$., but this approach is clearly too costly.  We can do better
under certain circumstances.  If $a^{N-1} \not\equiv 1$, then $a$
cannot have order $N-1$ and $N$ cannot be a prime.  If $a^{N-1}\equiv
1$ then the order of $a$, which we denote by $d$, divides $N-1$.  If
\[
N - 1 = q_1^{e_1} q_2^{e_2} \cdots q_k^{e_k}.
\]
then we must have
\[
d =  q_1^{f_1} q_2^{f_2} \cdots q_k^{f_k},
\]
where $f_i < e_i$.  If 
\[
a^{(N-1)/q_1} \not\equiv 1 \pmod{N}
\]
then $d$ does not divide $(N - 1)/q_1 = q_1^{e_1-1} \cdots q_k^{e_k}$.
This means that $f_1 = e_1$.    This gives the following sufficient
condition for the primality of $N$.

\begin{proposition}
Let $N-1 = q_1^{e_1} q_2^{e_2} \cdots q_k^{e_k}$, where the $q_i$ are
distinct primes.  If there exists and $a$ such that
\[
a^{N-1} \equiv 1 \pmod{N},
\]
and for each $q_i$
\[
a^{(N-1)/q_i} \not\equiv 1 \pmod{N}
\]
then $N$ is a prime
\end{proposition}

Unfortunately, this proposition requires the complete factorization of
$N-1$.  This can be quite difficult.  The following proposition, shows
that we only need factor ``most'' of $N-1$.

\begin{proposition}
Assume $N-1 = R \cdot q_1^{e_1} \cdots q_k^{e_k}$, where the
$q_i$ are distinct primes and none divide $R$.  Further assume that 
$R < q_1^{e_1} \cdots q_k^{e_k}$.  If an integer $a$ can be found such
that 
\begin{equation}\label{Prim:Mod:GCD:Eq}
(a^{(N-1)/q_i} -1, N) = 1
\end{equation}
for all $q_i$, but
\[
a^{N-1} \equiv 1 \pmod{N}
\]
then $N$ is a prime.
\end{proposition}

Equation \eqnref{Prim:Mod:GCD:Eq} is stronger than saying
$a^{(N-1)/q_i} \not\equiv 1 \pmod{N}$ since it implies that 
\[
a^{(N-1)/q_i} \not\equiv 1 \pmod{P}
\]
for every prime divisor of $N$.  Fortunately, \eqnref{Prim:Mod:GCD:Eq}
is only slightly more difficult to evaluate.

\begin{proof}
Let $P$ be the smallest odd prime divisor of $N$ and let the order of
$a$ modulo $P$ be denoted $d$.  Using \eqnref{Prim:Mod:GCD:Eq}, we
have
\[
a^{(N-1)/q_i} \not\equiv 1 \pmod{P},
\]
so by the reasoning of the previous proposition $q_1^{e_1} \cdots
q_k^{e_k}$ divides $d$ and thus also divides $P$.  Since $q_1^{e_1}
\cdots q_k^{e_k} > R$, $q_1^{e_1} \cdots q_k^{e_k}$ is greater than
$\sqrt{N}$.  This means $P > \sqrt{N}$, which contradicts $P$ being
the smallest factor of $N$.
\end{proof}

\section{Probabilistic Approaches}

Short proofs of compositeness are easy to exhibit.  To prove that $N$ is
composite we need only exhibit a factorization of $N$.  Unfortunately this
proof is hard to find.  An easier proof can be found using Fermat's little
theorem, Proposition \ref{FermatLittle:Prop}, which says that if $p$ is a
prime and $a$ is not divisible by $p$, then $a^{p-1} \equiv 1 \pmod{p}$.
If we can find an integer $a$ for which this equation is false, then $p$
must be composite.  An $a$ for which $a^{N-1} \not\equiv 1\pmod{N}$ is
called a {\em simple witness}\index{witness} to the compositeness of $N$.
Unfortunately, there are some composite numbers that do not possess any
simple witnesses.  These numbers are called
\keyi{Carmichael numbers}.

We call $a$ a {\em witness} to the compositeness of $N$ if it satisfies one
of the following two relations: 
\begin{enumerate}
\item $a^{N-1} \not\equiv 1 \pmod{N}$ or 
\item there exists an $i$ such that $2^i$ divides $N-1$ and $1 <
(b^{(N-1)/2^i} - 1, N) < N$.
\end{enumerate}

\begin{proposition}[Solovay and Strassen] \label{PrimeWitness:Prop}
If $N$ is composite then at least $3/4$
of the elements of $\Z/n\Z$ are witnesses to the compositeness of $N$.
\end{proposition}

\begin{proof}
\end{proof}

This gives a very efficient technique for proving an integer is
composite.  It also allows us to gain confidence in our suspicion that
an integer $N$ is prime.  If in one hundred trials we are unable to find
a witness to the compositeness of $N$ we can conclude, with a confidence
of $1 - 2^{-200}$ that $N$ is prime.  This does not prove that $N$ is
prime, nor will increasing the number of candidates.  Only examining all
$N-1$ positive integers less than $N$ reduces the uncertainty to $0$.  

\section{Goldwasser-Kilian Primality Test}

The fundamental problem with $N-1$ primality test is the difficulty of
factoring $N-1$---the order of the group of units modulo $N$.
Replacing the group of units modulo $N$ by the group of points on an
elliptic curve places us in a situation where the order of the group
ranges between $N+1 -2\sqrt{N}$ and $N+1+2\sqrt{N}$.  It is likely
that some number in this range is easy to factor.

{\Goldwasser} and {\Kilian} \cite{Goldwasser:Killian} generalized the
standard technique for proving an integer prime by using the group of
points on an elliptic curve.  They proceeded in a recursive fashion,
roughly as follows, 
\begin{itemize}
\item Choose an elliptic curve $E$ such that the number of points in
$E(\F_N)$ is $2q$, and where $q$ appears to be a prime using a
probabilistic primality test.
\item Choose an element  $P \in E$ such that $qP = 0$, \ie $P$ has
order $q$.
\item Recursively prove that $q$ is a prime.
\end{itemize}


\section{Wagstaff's Results}

\section{Adleman-Rumeley Approach}
