%$Id: factoring.tex,v 1.2 1992/05/26 22:09:23 rz Exp rz $

\chapter{Factoring Integers}
\label{Int:Factor:Chap}

The problem of factoring an integer into its prime components has been
a favorite problem of number theorists for centuries.  Recently, the
development of the RSA encryption algorithm \cite{Rivest78} has made
the integer factoring problem important to non-mathematicians.  Assume
we have message we want to transmit which consists of elements of the
set $A$.  An encryption function $\xi$ maps elements of $A$ into
elements of the set $B$.  These elements are transmitted across a
channel that might be observed by ``fiends.''  The decryption function
$\psi$ maps these elements of $B$ into elements of $A$ such that
$\psi(\xi(x)) = x$.  {\Rivest}, {\Shamir} and {\Adleman} observed that
if $A$ and $B$ are both $\Z/m\Z$ for some integer $m$ then for all $x
\in A$
\[
x^{\phi(m)+1} = x\pmod{m}.
\]
Thus if we choose two integers $e_{1}$ and $e_{2}$ such that $e_{1}e_{2}$
is a multiple of $\phi(m)+1$, we can use $\xi(x) = x^{e_{1}}\pmod{m}$ as the
encryption function and $\psi(x) = x^{e_{2}}\pmod{m}$ as the decryption
function.  Though the fiend may be aware of $m$ and $e_{1}$ it appears that
without the factorization of $m$, from which it is easy to determine
$\phi(m)$, $e_{2}$ is difficult to determine.  

The most straightforward technique for factoring an integer $N$ is to
perform trial divisions with each of the odd numbers less than $\sqrt{N}$.
This technique will work well for numbers whose factors are all quite
small, but in the worst case it will require $O(\sqrt{N})$ trial divisions.

An implementation of this factoring technique is quite simple.  However, we
will take this opportunity to create an extensible structure for the more
sophisticated factoring algorithms we will use later.  The entry point for
this procedure is the function {\tt factor} given below.  
\begin{verbatim}
(defun factor (N)
  (let ((*factor-method* *factor-method*)
        ans factors)
    (when (minusp N)
      (push (cons -1 1) ans)
      (setq N (- N)))
    (count-multiple-integer-factors N 2)
    (count-multiple-integer-factors N 3)
    (count-multiple-integer-factors N 5)
    (unless (= N 1)
      (loop
        (multiple-value-setq (N factors) 
             (funcall *factor-method* N))
        (setq ans (append factors ans))
        (if (= N 1) (return t))))
    (uniformize-factor-list ans)))
\end{verbatim}

{\tt Factor} takes an integer $N$ as its argument and returns a list of
pairs of prime divisors and their degree.  If 
\[
N = p_{1}^{e_{1}} \cdots p_{k}^{e_{k}},
\]
where $p_{1} < p_{2} < \cdots < p_{k}$ then {\tt factor} will return the
list
\begindsacode
(($p_{1}$ . $e_{1}$) ($p_{2}$ . $e_{2}$) $\cdots$ ($p_{k}$ . $e_{k}$))
\enddsacode

This version of {\tt factor} filters out a bunch of easy cases: $N$ being
negative or having small integer divisors.  The macro {\tt
count-multiple-integer-factors} takes care of this updating $N$ as
necessary and adding the factors it found to the variable {\tt ans}.
\begin{verbatim}
(defmacro count-multiple-integer-factors (N divisor)
  `(loop with i = 0
         do (multiple-value-bind (quo rem) (truncate ,N ,divisor)
              (when (not (zerop rem))
                (if (not (zerop i))
                    (push (cons ,divisor i) ans))
                (return t))
              (setq ,N quo)
              (incf i))))
\end{verbatim}
Depending upon the problems to be attacked and performance measurements,
other techniques might be inserted.\Marginpar{Should actually use GCD's to
eliminate all prime divisors below $p_{n}$ for some reasonable sized value
of $n$}

The core of the factoring routine is the loop
\begin{verbatim}
  (loop
    (multiple-value-setq (N factors) 
        (funcall *factor-method* N))
    (setq ans (append factors ans))
    (if (= N 1) (return t))))
\end{verbatim}
Each time through the loop some prime factors are split off of $N$ and
added to the list {\tt ans}.  This permits changes in the factorization
method as the number to be factored becomes smaller and also allows for
methods that do not obtain prime factors of $N$ but merely split $N$ into
two smaller pieces.  A simple example of this given in the next section.
When $N$ is reduced to $1$ the prime factors in {\tt ans} are sorted and
combined as necessary by the function {\tt uniformize-factor-list}.

The trial divisor algorithm is implemented in the following procedure.  
\begin{verbatim}
(defvar *skip-chain-for-3-and-5* (circular-list 4 2 4 2 4 6 2 6))

(defun simple-integer-factor (N)
  (let ((increments *skip-chain-for-3-and-5*)
        (divisor 7)
        ans)
    (flet ((simple-integer-factor-internal (N)
             (let ((limit (isqrt N)))
               (loop 
                 (cond ((= N 1)
                        (return (values N ans)))
                       ((> divisor limit)
                        (return (values 1 (list (cons N 1)))))
                       (t (count-multiple-integer-factors N divisor)
                          (when ans 
                            (return (values N ans)))))
                 (setq divisor (+ divisor (pop increments)))))))      
      (setq *factor-method* #'simple-integer-factor-internal)
      (simple-integer-factor-internal N))))
\end{verbatim}

The circular list contains the differences of those numbers that are not
congruent to $0$ modulo $2$, $3$ or $5$.  Thus reducing the number of trial
divisors by a factor of $\frac{1}{2}\cdot\frac{2}{3}\cdot\frac{4}{5}
\approx 0.267$.  Eliminating multiples of $7$, $11$ etc, would increase the
size of this list and yield only modest gains.

The task of finding the next prime divisor of $N$ is handled by the function
{\tt simple-integer-factor-internal}.  Each time it is called it advances
{\tt incre\-ments} and {\tt divisor} until it finds a prime divisor of $N$
and then returns.  To make this work properly {\tt simple-integer-factor}
changes the factor method to be a lexical of the internal function.  Thus
the different invocations of {\tt
simple-integer-\discretionary{}{}{}factor-internal} will use the updated
values of {\tt increments} and {\tt divisor}.

\section{Fermat's Technique}

The routine {\tt simple-integer-factor} searches for divisors of $N$ from
$2$ upwards.  Thus numbers that are product of two primes which are of
about the same size will take the full $O(\sqrt{N})$.  It is also possible
to search for (not necessarily prime) factors from $\sqrt{N}$ downwards.
This will work well for numbers that are the product of two large primes,
but it will also work well for number that have lots of small factors since
it will split them into two pieces of roughly the same size quickly and
will recursively recover the small prime factors in approximately $O(\log
N)$ operations.

Rather than just counting down from $\sqrt{N}$, {\Fermat} suggested using a
slightly different formulation that allows the sieving type of techniques
discussed in the previous section to work.  Assume $N = uv$ and that $u$
and $v$ are both relatively close to $\sqrt{N}$.  Rather than trying to
obtain $u$ and $v$ directly we try to find an $x$ and $y$ such that $x+y =
u$ and $x-y = v$, \ie
\[
N = x^{2} - y^{2}.
\]
We search for $x \in \{\lceil{\sqrt{N}}\rceil, \ldots, N/2\}$ looking for an
integer $y$ such that $N = x^{2} - y^{2}$.  Using the framework set up
earlier, this can be implemented as
\begin{verbatim}
(defun fermat-integer-factor (N)
  (loop for x = (1+ (isqrt N)) then (+ x 1)
        for w = (- (* x x) N) 
        for y = (isqrt w)
        do (when (zerop (- w (* y y)))
             (let ((u (+ x y))
                   (v (- x y)))
               (return (if (1? v)
                           (values 1 (list (cons u 1)))
                           (values u (factor v)))))))
\end{verbatim}

This technique is reasonable good at finding factors of $N$ if they are
close to $\sqrt{N}$, but if they are far away, \eg{} if $N$ is prime it can
take substantially longer than the simple factorization method.  However
there are two optimizations that make this technique attractive.  

For clarity, denote the values of {\tt x}, {\tt w} and {\tt y} at each
iteration with subscripts.  Then bulk of the computation in this method is
following set of assignments
\begindsacode
$x_{k+1} \leftarrow x_{k}+1$; \\
$w_{k+1} \leftarrow x_{k+1}^{2} - N$; \\
$y_{k+1} \leftarrow \lfloor \sqrt{w_{k+1}} \rfloor$; \\
zerop($w_{k+1} - y_{k+1}^{2})$
\enddsacode
First, notice that we can eliminate the multiplications from the calculation
of $w_{k+1}$ by observing that $x_{k+1}^{2} = x_{k}^2 + 2 x_{k} +1$, so we
should use
\begindsacode
$w_{k+1} \leftarrow w_{k}+ 2 x_{k}+1$;
\enddsacode
Second, the purpose of the last two lines is to determine whether or not
$w_{k+1}$ is a perfect square.  We can eliminate many of the $x_{k+1}$ from
consideration by observing that if $w_{k+1}$ is not a square modulo $p$
then it is not a square as an element of $\Z$.


\section{Factoring Using Continued Fraction}

\section{Pomerance's Quadratic Sieve}

\section{Pollard's $(p - 1)$ Method}

{\Pollard}'s $p - 1$ method is an ingenious application of \key{Fermat's
little theorem} to factor an integer $N$.  It is particularly useful
when there is a factor $p$ of $N$ such that $p - 1$ is divisible only
by small primes, \ie{} smooth.

Assume $Q$ and $p$ are relatively prime and that $p - 1$ divides $Q$.
By \key{Fermat's little theorem} $p$ will divide $a^Q - 1$.  Now
assume that $p$ is a divisor of $N$, and $p-1$ does not.  Then
$a^Q\not\equiv 1$ modulo $N$.  But since $p$ divides both $a^Q - 1$
and $N$, we can determine $p$ by taking the GCD of $a^Q - 1 \bmod {N}$
and $N$.

{\Pollard}'s idea is to vary $Q$ checking to see if $(a^Q - 1, N)$ is greater
than $1$.  This would be comparable to checking divisors of $N$ except
that we can be somewhat cleverer in our choice of $Q$. If we choose $Q$ to
be a product of small primes, then computing one GCD will test all primes
$p$, such that $p - 1$ divides $Q$. Thus we will check smooth primes quickly.

Now the problem is reduced to determining good choices for $Q$.  The usual
technique used is to generate a list of all primes and prime powers up to
some limit $L$.  After replacing the prime powers by the prime, we get a 
sequence that begins
\[
2, 3, 2, 5, 7, 2, 3, 11, 13, 2, 17, 19, \ldots
\]
We will let $Q_k$ be the product of the first $k$ integers in this list.
Notice that this is effectively computing the least common multiple of 
${1, 2, \ldots , k}$.  Then repeatedly compute the GCD of $a^{Q_k} - 1$ and 
$N$, where $a$ is a random integer.  To speed things up, we can perform
the GCD for several $Q_i$s at once by calculating
\[
M_n = \prod_{i=1}^n      a^{Q_i} - 1\pmod{N}
\]
and computing the GCD of $N$ and $M_n$ for $n = 100, 200, 300, \ldots$.

A simple implementation is given below.\Marginpar{to be provided}

In analyzing {\Pollard}'s $p-1$ algorithm we need to know what the
likelihood of an integer being smooth is.  The following theorem, due
to {\Knuth} and {\TrabbPardo} \cite{Knuth76}, quantifies this

\begin{theorem}[{\Knuth} and {\TrabbPardo}]
The probability that all the prime factors of a positive
integer $x$ are less than $x^\alpha$ is close to $\alpha^{1/\alpha}$.
\end{theorem}

\begin{proof}
\end{proof}

\section{Lenstra's Elliptic Curve Algorithm}

{\LenstraH}'s algorithm is a generalization of {\Pollard}'s $p - 1$ algorithm
described earlier.  {\Pollard}'s technique makes use of the multiplicative
group of integers modulo $N$, the number to be factored.  If $N = pq$ is
the product of two primes, then the order of this group is $(p - 1)(q -
1)$.  The searching technique is adept at finding a factor $p$ if $p - 1$
is smooth.  If a different group is used then an algorithm can be developed
that finds a factor of $N$ if $p + 1$ is smooth.  This algorithm was
developed by {\WilliamsH} \cite{Williams82}.

Throughout this section we will use additive notation for groups.
Thus \key{Fermat's little theorem}
\[
a^p - 1\equiv 1 (p)
\]
will be replaced by $ma = 0$ where $m$ is the order of the group to which
$a$ belongs.

In this and the next few paragraphs we follow the presentation of Jeff
{\Shallit}.  Assume we wish to factor $N = pq$ and that $p\equiv 3 \pmod4$.
We first choose a random point $P$ that is an element of the circle
group modulo $C_N$.  Then if we can find an $m = \lcm\{1, 2, \ldots,
k\}$ such that $mP = 0$ in $C_p$ but not $C_q$, we will be able to split
off $p$ from $N$.  If $mP = (x, y)$ in $C_N$, then $p$ divides $y$, so
the GCD of $y$ and $N$ will reveal $p$.  For every point in $C_p$ to be
a zero of $mP$ then $(p + 1)$ must divide $m$.  This technique will work
when $p + 1$ is smooth.  As with Pollard's $p-1$ technique, we can
compute $mP$ using $O(\log m)$ operations by ``repeated squaring.''

The biggest problem with this technique is finding a random point $P$
that lies on the circle $x^2+y^2=1$.  We can make this problem easier by
introducing a new degree of freedom.  We can define a group law using
curves other than circles.  By replacing the circle $x^2+y^2=1$ by the
curve $x^2 - D y^2 = 1$, we get the group law
\[
P_1 + P_2 = (x_1 x_2 + D y_1 y_2, x_1 y_2 + y_1 x_2).
\]
Now we can choose a point at random, and use the values of $x$ and $y$
to determine $D$.  The following proposition answers gives the number of
points on a quadratic curve generally.

\begin{proposition}
Let $p$ be an odd prime number, $D$ an integer relatively prime to
$p$.  Denote $E_p(D)$ be the number of elements $(x, y)$ of $\Z/(p)
\times \Z/(p)$ that are solutions of $x^2 - D y^2 = 1$.  Then
\[
E_p(D) = 
\sum_{0\le y < p} \left({1 + D y^2 \legendre p} + 1\right)
= p - \left(\frac{D}{p}\right).
\]
\end{proposition}

\begin{proof}
First equality is immediate since, for a fixed $y$, the number of solutions
of $X^2 = 1 +Dy^2$ is 
\[
{1 + D y^2 \legendre p} + 1
\]
by proposition \ref{NumQuadSol:Prop}.

The hard part of the proof is to be supplied later. 
\end{proof}

Since all quadratic curves can be transformed in to equations of the
form $x^2 - D y^2 = 1$, this proposition shows that using quadratic
curves will only work if $p \pm 1$ is smooth.  {\LenstraH}'s idea is
to use ``elliptic curves'' (curves of degree 3) instead.  The number
of points on these curves vary between $p + 1 -2\sqrt{p}$ and $p + 1
+2\sqrt{p}$.  This gives many more candidates for smoothness than the
previous techniques.


[see also \cite{Adleman91,LenstraAK90}
