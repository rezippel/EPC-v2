%$Id: euclids-alg.tex,v 1.1 1992/07/12 19:39:34 rz Exp rz $
\chapter{Euclid's Algorithm}
\label{Integer:GCD:Chap}

Among the mathematical problems we will investigate is the computation
of greatest common divisors, factoring into prime factors and finding
rational approximations.  These computations may be performed on a
variety of different mathematical quantities: polynomials, rational
integers, power series, differential operators, etc.  The most
familiar of these algebraic structures are the \keyi{natural numbers}:
${\mathbb N} = \{1, 2, 3, \ldots\}$.  If we include zero and 
the negative integers we have $\Z$, the ring of \keyi{rational
integers}. The elements of $\Z$ are commonly called the  
\keyi{integers}, but we will use the term integer for more complex
algebraic structures. 
\addsymbol{$\mathbb N$}{The natural, or counting numbers, $1, 2, 3, \ldots$}
\addsymbol{$\Z$}{The ring of rational integers}
\addsymbol{$\Q$}{The field of rational numbers}

This chapter begins our study of mathematical algorithms with an
investigation of a variety of computations involving the rational
integers.  Later chapters discuss more elaborate algorithms that deal
with polynomials and other algebraic structures.  Interestingly, the
resolution of some of these problems for the rational integers can
lead to algorithms that make use of polynomials, thus making the whole
process somewhat circular.

\medskip
An integral part of our elementary education is learning to add and
multiply integers.  The hardware used in computers usually only
implements arithmetic operations for numbers of a specified length,
\ie, $32$-bit or $64$-bit integers.  In algebraic computation we often
use much larger integers; however, this is not because the answers are
so large.  Instead these large numbers arise in the middle of some
computation and then shrink towards the end.  This phenomenon is
called \keyi{intermediate expression swell}.

Detailed algorithms for integer arithmetic may be found in many texts,
\eg, {\Knuth} \cite{Knuth:II} and {\Aho}, {\Hopcroft} and {\Ullman}
\cite{Aho:Hopcroft:Ullman74}. But implementing efficient variable
precision arithmetic algorithms is a very delicate process that raises
many issues not touched upon in these texts, such as register usage,
cache turbulence and pipeline stalls.  For instance, properly
implemented classical algorithms work very well for larger integers
than one might expect.  Because of all of these reasons, because
numerous variable precision integer packages are available and because
an increasing number of programming languages now provide variable
precision integers as one of their basic types, we do not discuss
these arithmetic algorithms.

\medskip
We begin with some basic definitions.  When a rational integer $m$ can
be written as a product of two other integers $p$ and $q$, $m = pq$,
then $p$ and $q$ are called {\em divisors}\index{divisor} of $m$ and
$m$ is said to be {\em divisible} by $p$ and
$q$.\index{divisibility!integer} We write $p \mid m$ to indicate that
$p$ divides $m$.  We indicate that $p$ does not divide $m$ by $p \nmid
m$.  The only rational integers that have multiplicative inverses are
$1$ and $-1$.  Because they have multiplicative inverses they are
called {\em units}.\index{unit, of a ring} Two integers whose ratio is
a unit are said to be \keyi{associates}.  For simplicity we will
ignore associates when enumerating the divisors of an integer.  Thus
every non-unit has at least two divisors, $1$ and itself (up to
associates).  An integer that has only these two ``trivial''
divisors\index{trial division} is called a \keyi{prime}.  All other
non-units are called \keyi{composite}.
\addsymbol{$p \mid m$}{$p$ divides $m$}
\addsymbol{$p \nmid m$}{$p$ does not divide $m$}

A presentation of a positive integer $N$ as a product of positive
integers, $N = m_1 \cdots m_k$, is called a \keyi{factorization} of
$N$.  A factorization of an integer is said to be {\em
complete}\index{factorization!complete} if each of the $m_i$ is a
prime or a power of a prime.  Unless otherwise noted, all
factorizations are complete.  We usually write such factorizations as
$N= p_1^{e_1} \cdots p_k^{e_k}$.  According to the
\key{fundamental theorem of arithmetic} the factorization of a
rational integer is unique.

The factoring problem for an integer $N$ is, at the moment, very
difficult.  Trying all possible divisors takes time $O(\sqrt{N})$.
Very roughly, the best known algorithms take time $O(\exp(\sqrt{\log
N}))$, which, though much better, is not near what would be expected
of a polynomial time algorithm: $O(\log^k N)$.  Interestingly, the
time required to factor a polynomail of degree $n$ is only about $O(n^5)$.

An integer $d$ that divides both $m$ and $n$ is called a \keyi{common
divisor} of $m$ and $n$.  The number $m/d$ is called the
\keyi{cofactor} of $m$ with respect to $d$.  The largest common
divisor of two integers is called a \keyi{greatest common divisor}
({\sc gcd}).  If the greatest common divisor of two integers is $1$,
then we say they are {\em relatively prime\/}.\index{relatively prime}
When dealing with rational numbers, more time is probably spent
computing the {\sc gcd} of integers than any operation other than
addition and multiplication.  When using rational functions, it is not
unusual for the time spent computing polynomial {\sc gcd}s to dominate
all other arithmetic operations.  \sectref{Integer:Euclidean:Sec}
discusses Euclid's algorithm for computing the {\sc gcd} of two
integers.  Euclid's algorithm provides an entr\'e to many facets of
elementary number theory.  The other sections of this chapter
illustrate several of these ideas.  Later chapters delve into each of
the issues raised in more detail.

\section{Euclidean Algorithm}
\label{Integer:Euclidean:Sec}

\index{Euclidean algorithm|(}
The {\em Euclidean algorithm} for computing the {\sc gcd} of two
integers has been known at least from the time of \key{Euclid}
\cite{Euclidean:Algorithm}, who called it ``\key{anthyparesis}.''  Most
likely, it is the oldest mathematical algorithm in existence.  The
principles underlying the Euclidean algorithm are fundamental to the
solution of many problems in mathematics ranging from diophantine
approximation to elimination.  Because of its importance, and because
it is the first algorithm we consider, we derive Euclid's algorithm in
some detail.

Let $m > n$ be two positive rational integers.  Denote by $d$ some
common divisor of $m$ and $n$.  $d$ also divides, $m+n$ and $m-n$.  In
fact, $d$ divides every number of the form $am - bn$, where $a$ and
$b$ are rational integers.  Denote by $M$ the set of all such numbers
\[
M = \{ am - bn \mid a,b \in \Z\}.
\]
Since every common divisor of $m$ and $n$ divides every element of
$M$, each is also a multiple of the greatest common divisor of $m$ and
$n$.  Thus the smallest positive element of $M$ is the {\sc gcd} of $m$ and
$n$.

To make this an algorithm, we use a simple recursive technique that
converts a pair of elements in $M$ into a pair of smaller elements.
This process terminates with the pair $0$ and the {\sc gcd} of $m$ and
$n$.  The simplest approach is to replace the pair $(m,n)$ by
$(m-n,n)$, and then $(m-2n, n)$, and so on until $m-kn < n$.  At this
point the two values are interchanged and the new pair is $(n, m-kn)$.

\index{remainder}
All of the subtraction operations prior to an interchange of $m-n$ and
$n$ can be collapsed into a single division:
\begin{equation}
m = q n + r.\qquad\qquad 0 \le r < n 
\label{Remainder:Eq}
\end{equation}
In this case, $r$ is called the {\it remainder} of $m$ divided by $n$.
(This definition works both for positive and negative values of $m$
and $n$.)  Now let $n_1 = m$ and $n_2 = n$.  The sequence of
divisions and remainders used in the {\sc gcd} algorithm would look like:
\begin{equation} \label{Int:RemainderSeq:Eq}
  \begin{aligned}
   n_1 &= q_1 n_2 + n_3\\
   n_2&=q_2 n_3 + n_4\\
     &\vdots\\
   n_k& = q_k n_{k+1} + 0
  \end{aligned}
\end{equation}
This sequence of operations is called a \keyi{remainder sequence}, and
its use to compute the {\sc gcd} of two numbers is called Euclid's
algorithm.  \figref{Int:Euclid:Alg:Fig} illustrates the computation of
the {\sc gcd} of $15027133$ and $8562227$.  In this case the {\sc
gcd} is $163$ which we write as 
\[
(10527133, 8562227) = 163.
\]

\begin{figure}
\[
\begin{aligned}
  15027133 &= 1 \cdot 8562227 + 6464906\\
  8562227  &= 1 \cdot 6464906 + 2097321\\
  6464906  &= 3 \cdot 2097321 + 172943\\
  2097321  &= 12 \cdot 172943  + 22005\\
  172943   &= 7 \cdot 22005   + 18908\\
  22005    &= 1 \cdot 18908   + 3097\\
  18908    &= 6 \cdot 3097    + 326\\
  3097     &= 9 \cdot 326     + 163\\
  326      &= 2 \cdot 163     + 0
\end{aligned}
\]
\caption{Euclidean algorithm for two integers\label{Int:Euclid:Alg:Fig}}
\end{figure}


The following recursive function gives a direct implementation of the
Euclidean algorithm for two positive integers, based on the discussion
of the previous few paragraphs.

\begin{verbatim}
func IntGcd(m, n int) int {
	if m < n {
		return IntGcd(n, m)
	} else {
		if n == 0 {
			return m
		} else {
			return IntGcd(n, m%n)
		}
	}
}
\end{verbatim}

The bulk of the work is performed in the remainder calculation.  For
some machines, it may be substantially cheaper to compute something
slightly larger than the remainder.  For instance, the first several
bits of $m$ and $n$ allow us to estimate the size and first few bits
of the quotient of $m$ and $n$ quickly.  Using this value, something
close to the remainder can be computed quite quickly.  Other
algorithms are described in Section 4.5.2 of {\Knuth} \cite{Knuth:II}.

The second question we have about every algorithm is: How fast is it?
(The first question is whether or not the algorithm gives the correct
answer!)  To answer this question we bound the number of remainder
steps required 
to compute the {\sc gcd} of two integers.  Denote by $F(N)$ the
maximum number of 
remainder steps required to compute the {\sc gcd} of two positive
numbers less than $N$.

Assume $u > v$ are two positive integers without a common factor, each
larger than $N$ and chosen such that the quotients in their remainder
sequence are all $1$.  Thus, the {\sc gcd} of two integers smaller
than $N$ cannot take any more steps than the number required by $u$
and $v$.  The pair of integers $u+v$ and $u$ also has a remainder
sequence where each $q_i$ is $1$, and the remainder sequence for
$(u+v, u)$ is one step longer than that for $(u, v)$.  The sequence
$v, u, u+v, 2u+v,
\ldots$, can be continued as far as one would like.

To obtain a bound on the number of steps in a {\sc gcd} algorithm, we need
to be able to compute the growth rate of the terms of this sequence.
A crude bound can be determined by the following simple reasoning.

Assume that when the Euclidean algorithm is applied to two integers
$F$ and $G$, $F > G$, the remainder sequence is as long as possible.
For this to be the case, the quotient of $F$ divided by $G$ should be
$1$.  Denote the remainder by $H$.  For $F$ and $G$ to have a long
remainder sequence $G$ and $H$ must also.  Thus the quotient of $G$
divided by $H$ should also be one.  This suggests that we should
consider the sequence of numbers $f_i$ satisfying the 
recurrence
\begin{equation}
\label{Fibonacci:Eq}
f_{i+1} = f_i + f_{i-1},
\end{equation}
where we can take the basis of the \keyi{difference equation} to be $f_{1} =
f_{2} = 1$.  $f_i$ has a remainder sequence of length $i-1$ when
divided by $f_{i-1}$ and no smaller number can have a remainder
sequence so long.\label{Fibonacci:Remainder:Sequence}

Since the integer quotient of $f_{i}$ divided by $f_{i-1}$ is $1$,
$f_{i} < 2 f_{i-1}$.  This allows us to bound $f_{i+1}$ above and
below as follows:
\[
3 f_{i-1} > f_{i+1} > \frac{3}{2} f_{i},
\quad\mbox{or}\quad
c_1\left(\sqrt{3}\right)^i > f_{i+1} > c_2 (\frac{3}{2})^{i}.
\]
Consequently,
\[
F(N) = O(\log N).
\]

We can improve the constants in the estimates for $f_{i+1}$ above with
only a bit more work.  Assume that for large $i$ the ratio between
$f_{i+1}$ and $f_i$ tends to a constant, $\phi$.  Then for sufficiently
large $n$ we have 
\[
\phi = \frac{f_{n+1}}{f_n} = \frac{f_n + f_{n-1}}{f_n} = 1 + \frac{1}{\phi}.
\]
This gives a quadratic equation for $\phi$, $x^2 - x - 1=0$, whose
only positive solution is
\[
\phi = \frac{1 + \sqrt{5}}{2} \approx 1.618033988749894.
\]
So we expect $f_i = c_1 \phi^i$.  
Notice that this lies between the bounds given above of $\sqrt{3} =
1.732$ and $3/2 = 1.5$.  

We can go further and determine an exact formula for $f_i$.  Notice that if
$\alpha$ is a zero of $x^{2} -x -1$ then $\alpha^{i}$ will satisfy
\eqnref{Fibonacci:Eq}: 
\[
f_{i+1} - f_{i} - f_{i-1} = \alpha^{i+1} - \alpha^{i} - \alpha^{i-1}
 = \alpha^{i-1} ( \alpha^{2} - \alpha -1) = 0.
\]
Since \eqnref{Fibonacci:Eq} is a linear \key{difference equation},
linear combinations of the solutions of \eqnref{Fibonacci:Eq} are also
solutions.  The negative solution of the quadratic equation is $1-\phi
= -1/\phi$. Consequently,
\[
f_i = c_1 \phi^i + c_2 (-\phi)^{-i},
\]
where, as with ordinary differential equations, $c_1$ and $c_2$ are
determined by the ``initial conditions.''  Using $i = 1$ and $i=2$ we
have
\[
\begin{aligned}
  c_1 \phi - c_2 \phi^{-1} &= f_{1} = 1,\\
  c_1 \phi^2 + c_2 \phi^{-2}&= f_{2} = 1,
\end{aligned}
\]
whose solution is
\[
\begin{aligned}
  c_1 &= \frac{\phi +1}{\phi^3 +\phi} = \frac{1}{\sqrt{5}},\\
  c_2&=\frac{\phi^2 - \phi^3}{\phi^2 + 1} = - \frac{1}{\sqrt{5}}
\end{aligned}
\]
Thus,
\[
f_i = 
\frac{1}{\sqrt{5}} \left[
\left(\frac{1 + \sqrt{5}}{2}\right)^i -
        \left(\frac{1 - \sqrt{5}}{2}\right)^i \right]
\]

This sequence of numbers is called the \keyi{Fibonacci numbers}.  They
were first mentioned by {\Fibonacci} \cite[pages 283--285]{Fibonacci}
and have important applications in number theory, complexity theory
and other fields.

Since the negative solution of the quadratic equation has absolute
value less than $1$, for large $i$, the Fibonacci numbers can be
approximated by
\[
f_i = \frac{\phi^i}{\sqrt{5}}
\]
for large $i$. Consequently,
\[
F(N) \approx \frac{\log \sqrt{5} N}{\log \phi} = \log_{\phi} \sqrt{5} N.
\]
These results were first demonstrated  by {\Lame} in 1844 \cite{Lame44}.

The techniques used to solve this simple \key{difference equation} are
similar to those used to solve constant coefficient linear
differential equations.  More complicated difference equations often
occur when analyzing algorithms and in other problems.  Obtaining
closed form and asymptotic solutions of difference equations is
somewhat similar to the corresponding problems for differential
equations where symbolic techniques have had dramatic success in
recent years.  The field of difference equation problems has not been
as closely examined and remains a fertile area for future research.

\index{Euclidean algorithm|)}

\section{Diophantine Approximations}
\label{Euclid:DA:Sec}

Returning to the original problem, assume we are interested in
computing the {\sc gcd} of $n_1$ and $n_2$.  For simplicity we assume
the {\sc gcd} is $1$.  Define $M$ to be the set
\[
M = \{ n_1 X - n_2 Y \mid X, Y \in \Z \}.
\]
The {\sc gcd} of $n_1$ and $n_2$ is the smallest positive element of $M$.
In other words, we want to find the smallest non-trivial value of the
binary form $|n_1 X - n_2 Y|$.  Another way to say this, is that we want
to find the best possible rational number approximation to $n_1/n_2$:
\[
\left|\frac{n_1}{n_2} - \frac{Y}{X}\right| \le \frac{1}{n_2 X} < \frac{1}{X^2}.
\]

Looking closely at the remainder sequence\index{remainder
sequence!integer} \eqnref{Int:RemainderSeq:Eq}, we can see how to find
the candidate solutions $X$ and $Y$.  By isolating the remainder terms
on one side of the equation we have
\[
  \begin{aligned}
    n_1 - q_1 n_2 &= n_3,\\
    n_2 - q_2 n_3 &= n_4,\\
     &\vdots\\
    n_{k-1} - q_{k-1} n_{k} &= 1.
  \end{aligned}
\]
We can substitute the value of $n_3$ given in the first equation into
the second equation, to get a relationship between $n_1$, $n_2$ and
$n_4$:
\begin{equation}\label{Euclid:Int:D:Eq}
-q_2 n_1 + (q_1 q_2 + 1) n_2 = n_4.
\end{equation}
Writing this relation in the form of an approximation to $n_1/n_2$  gives
\[
\left|\frac{n_1}{n_2} - \frac{q_1 q_2 + 1}{q_2}\right| 
= \frac{n_4}{n_2 q_2}.
\]

Using \eqnref{Euclid:Int:D:Eq}, and $n_3 = n_1 - q_1 n_2$, we can
obtain a relationship between $n_1$, $n_2$ and $n_5$:
\[
\left|\frac{n_1}{n_2} - \frac{q_1 q_2 q_3 + q_1 + q_3}{q_2 q_3 +1 }\right| 
= \frac{n_5}{n_2 (q_2 q_3 +1)}.
\]
Notice that 
\[
\frac{n_4}{n_2 q_2} > \frac{n_5}{n_2 (q_2 q_3 +1)},
\]
so the Euclidean algorithm is actually producing increasingly accurate
approximations to $n_1/n_2$.

When this process is taken to its logical conclusion, one obtains two
polynomials $X$ and $Y$ in the quotients $q_1, \ldots, q_{k-1}$ such
that
\[
 n_1 X - n_2 Y = 1,
\]
since $n_1$ and $n_2$ are relatively prime.  $Y/X$ is a quite
good approximation to $n_1/n_2$:
\[
\left|\frac{n_1}{n_2} - \frac{Y}{X}\right| = \frac{1}{n_2 X} 
   <  \frac{1}{X^2}.
\]
Observe that fractions with a denominator of $X$ can only approximate
real numbers to an accuracy of $1/2X$.  

The problem just studied can be phrased as trying to minimize the
value of the form $X r - Y$, where $r$ is a rational number.  There are
many interesting generalizations.  If $r$ is replaced by an algebraic
number, then it turns out that there are an infinite number of
``exceptionally'' good approximations to $r$ if $r$ is the solution of
a quadratic equation with integer coefficients, but only a finite
number of such solutions when $r$ is of higher degree.

It is also interesting to consider multiple approximations.  Given
real numbers $\alpha_1, \alpha_2, \ldots, \alpha_k$, we would like to
find integers that minimize the linear form:
\[
\left|X_0 + X_1 \alpha_1 + X_2 \alpha_2 + \cdots + X_k \alpha_k \right|.
\]
This type of problem is discussed in \chapref{Lattice:Chap} and can be
used to develop theoretically efficient algorithms for factoring
polynomials.

\section{Continued Fractions}
\label{Euclid:CF:Sec}

The approximation ideas of the previous section can be further developed
by rewriting the remainder sequence \eqnref{Int:RemainderSeq:Eq} in terms
of fractions.  This gives:
\begin{equation}\label{Euclid:CF:Eq}
  \begin{aligned}
    \frac{n_1}{n_2} &= q_1 + \frac{n_3}{n_2},\\
    \frac{n_2}{n_3} &= q_2 + \frac{n_4}{n_3},\\
     &\vdots\\
    \frac{n_{k-1}}{n_k} &=  q_{k-1} + \frac{1}{n_k}.
  \end{aligned}
\end{equation}
Replacing $n_3/n_2$ by the value in the second of these equations,
and continuing we have:
\[
\begin{aligned}
\frac{n_1}{n_2} &= q_1 + \frac{n_3}{n_2} = 
 q_1 + \frac{1}{\displaystyle q_2 + \frac{n_4}{n_3}}, \\
& = \frac{n_1}{n_2} = q_1 + \frac{1}{\displaystyle q_2 + 
    \frac{1}{\displaystyle q_3 + 1 \frac{1}{\displaystyle q_4 +
\ddots}}}.
\end{aligned}
\]
This strange way of writing the fraction $n_1/n_2$ is called a
\keyi{continued fraction}.   The $q_i$ that appear in the continued
fraction are called the {\em partial quotients}.\index{partial
quotient} From
\eqnref{Euclid:CF:Eq} and using the fact that $n_1 > n_2 > n_3 >
\cdots$ we see that  $q_i$ is the integer part of $n_i/n_{i+1}$.

Truncating the continued fraction and clearing the denominators:
\[
\begin{aligned}
q_1 + \frac{1}{q_2} & = \frac{q_1 q_2 + 1}{q_2}, \\
q_1 + \frac{1}{\displaystyle q_2 + 
    \frac{1}{\displaystyle q_3}} & = 
  \frac{q_1 q_2 q_3 + q_1 + q_3}{q_2 q_3 + 1}.
\end{aligned}
\]
Notice that these are the ``exceptionally'' good approximations to
$n_1/n_2$ discovered earlier.  This property is one of the reasons for
the interest in continued fractions.  \chapref{CF:Chap} discusses some
of the properties of continued fractions in more detail and how to
perform calculations with them.

\section{Diophantine Equations}
\label{Euclid:DE:Sec}

If we are only interested in the integral (or perhaps rational)
solutions of a polynomial equation (or system of equations), then we
call the problem a {\em diophantine} problem.\index{diophantine
equation} The simplest diophantine equation has already appeared.  To
compute the best rational approximation of $n_1/n_2$ we were looking
for integers $X$ and $Y$ such that
\begin{equation}\label{Euclid:Approx:Eq}
n_1 X - n_2 Y = 1.
\end{equation}
This is a  diophantine equation in $X$ and $Y$ since only elements of
$\Z$ are of interest in its solution.

This equation does have an infinite number of solutions, however.  Let
$x$ and $y$ be a solution of \eqnref{Euclid:Approx:Eq}.  Then $x+n_2
t$ and $y+ n_1 t$ is also a solution for integral values of $t$.  Thus
there are a countable number of solutions to \eqnref{Euclid:Approx:Eq}
as a diophantine problem.

Diophantine equations arise in a number of problems.  We give two more
examples here.  An integral \keyi{Pythagorean triangle} is a right
triangle (in Euclidean space) whose sides are each rational integers.
By the Pythagorean theorem their sides are zeroes of
\[
x^2 + y^2 = z^2,
\]
where $z$ is the length of the hypotenuse.  This equation involves three
unknown variables and is quadratic, not linear.  As an equation over
the reals, it has a continuously infinite number of solutions.  As a
diophantine equation, it has only a countably infinite number of
solutions, which can be parameterized as 
\begin{equation}\label{PythagEq:Soln:Eq}
x = r^2 - s^2, \quad y = 2rs, \quad z = r^2 + s^2.
\end{equation}
We leave the proof of this to the reader.  

What integers can arise as the sides of a Pythagorean triangle?  Any
even integer, \eg, $2k$ can be a side:
\[
(k^2 - 1 , 2k, k^2+1).
\]
Odd integers can also be sides, since every odd number is the difference
of two squares, $2k+1 = (k+1)^2 - k^2$.  Thus,
\[
(2k+1, k(k+1), 2k^2+2k +1)
\]
is a Pythagorean triple.

A more challenging question is: Which rational integers can be the
{\em area} of a Pythagorean triangle?  If the sides of the `triangle
are restricted to be rational integers then the area must be of the
form $rs(r^2-s^2)$.  However, it is more interesting to consider
triangles whose sides are rational numbers and whose area is an
integer.\Marginpar{Need to give an example here and give a bit more detail.}

Pythagorian triangles with rational sides can be determined by
allowing $r$ and $s$ to be rational in \eqnref{PythagEq:Soln:Eq}.  (Do
all Pythagorean triangles with rational sides arise in this fashion?)

The integer $n$ is the area of a Pythagorean triangle if $xy = 2n$ and
$x^2 +y^2 = z^2$.  Eliminating $y$ we have
\[
x^2 + \left(\frac{2n}{x}\right)^2 = z^2 \Longrightarrow x^4 + 4n^2 =
(xz)^2.
\]
Rewriting this slightly, we want to know if there exist rational
numbers $u$ and $v$ such that
\begin{equation}\label{Euclid:Cong:Eq}
u^2 = v^4 + 4n^2.
\end{equation}
This is a far harder diophantine equation to solve.  In this case
there are only a finite number of solutions and proving this is quite
difficult.  To do so requires techniques of {\em elliptic
curves}.\index{elliptic curve} This problem is beyond the grasp of the
techniques discussed in this book.  A nice presentation of the
mathematics relevant to this problem and its solution is given in
{\Koblitz}'s book \cite{Koblitz:Elliptic}.

\section*{Notes}

\small

If $F$ is a purely algebraic extension of the rational numbers, $\Q$,
then every element $\alpha \in F$ is the zero of a polynomial with
coefficients in $\Z$.  If $\alpha$ is the zero of a polynomial whose
\key{leading coefficient} is $1$ then $\alpha$ is said to be an
\keyi{integral element} of $F$.  The ring of integral elements of $F$
is called the \keyi{ring of integers} of $F$.  The ring of integers of
$\Q[\sqrt{-1}]$ is $\Z[\sqrt{-1}]$ and the elements of $\Z[\sqrt{-1}]$
are called \keyi{Gaussian integers}.  The ring of integers of
$\Q[\sqrt{5}]$ is $\Z[(1+\sqrt{5})/2]$.

The integral elements of $\Q$ are the elements of $\Z$.  Throughout
this book we use the term \keyi{rational integer} instead of just {\em
integer} to be more precise.

\normalsize
