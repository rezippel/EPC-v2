%$Id: rational-int.tex,v 1.1 1992/05/10 19:36:40 rz Exp rz $
\chapter{Integration of Rational Functions}
\label{Ration:Int:Chap}

Basic stuff on integration \cite{Rosenlicht72,Mead61,Ritt23a}.  {\Ritt}'s
book: \cite{Ritt48}  


In this section we will show how to compute the integral of a rational
function.  The integral of a rational function has the form
\[
\int {A(x) \over B(x)} \, dx 
  = {C(x) \over D(x)} + \sum_i \alpha_i \log P_i(x),
\]
as will be demonstrated later.  The first term of the right hand side is
called the {\em rational part} of the integral while the second portion is
called the {\em transcendental} or {\em irrational} part of the integral.
The first two subsections demonstrate algorithms for computing the rational
portion of the integral.  Subsection 3,  discusses the transcendental
portion. 

\section{Hermite's Algorithm}

All algorithms for computing the rational part of the integral begin with a
square free decomposition of the denominator.  Assume
\[
B(x) = B_1(x) B_2^2(x) \cdots B_n^n(x),
\]
which can be calculated using the algorithms from Chapters
\ref{PRS:Chap} and \ref{Hensel:Chap}.  Hermite's
algorithm begins with a {\em square-free partial fraction decomposition} of
$A(x)/B(x)$
\[
{A(x) \over B(x)} = 
{A_1(x) \over B_1(x)} + {A_2(x) \over B_2^2(x)} + \cdots +
{A_n(x) \over B_n^n(x)}
\]
Since the $B_i(x)$ are pairwise relatively prime, the $A_i$ can be
calculated by clearing fractions and using the Euclidean algorithm to
solve the resulting linear diophantine equation.

Then each of the terms is broken up into a {\em complete square-free
decomposition}
\[
{A_k(x) \over B_k^k(x)} = {A_{k1}(x) \over B_k(x)} +
{A_{k2}(x) \over B_k^2(x)} + \cdots + {A_{kk}(x) \over B_k^k(x)}.
\]
Clearing fractions again shows, that the $A_{ij}$ are the coefficients in a
$B_k(x)$-adic expansion of $A_k(x)$.

Finally the terms are integrated by rewriting them as 
\begin{equation} \label{Int:PF:Eq}
{A_{k\ell}(x) \over B_k^{\ell}(x)} = 
C(x) {B_k'(x) \over B_k^{\ell}(x)} +
{D(x) \over B_k^{\ell-1}(x)}.
\end{equation}
$C(x)$ and $D(x)$ are computed using the Euclidean algorithm.  The term
involving $B_l^{\ell}$ can be integrated by parts to give a terms of
degree $\ell - 1$ only.  By starting from the highest degree denominator
and working down, all the rational functions can be reduced to one that have
square-free denominators.  These rational functions have transcendental
integrals and can be integrated by the techniques given is Section 1.3.

\section{Mack's Algorithm}

Hermite's algorithm divides the rational function to be integrated
into a large number of parts using partial fraction decomposition.  It
is possible to avoid this significantly by using an algorithm due to
{\Mack} \cite{Mack:Integration}.  The idea is to apply equation
\eqnref{Int:PF:Eq} of section 1.1 directly to the original integral.

Again assume that $B(X)$ has a square-free decomposition of $B_1 B_2^2
\cdots B_n^n$, which we also write as $B = C B_n^n$.  The we write
\begin{equation}\label{Int:Mack:PF:Eq}
\int {A(x) \over B(x)} \, dx = {D(x) \over B_n^{n-1}(x) C(x)}
+ \int {E(x) \over B_n^{n-1}(x) C(x)}\, dx.
\end{equation}
We can assume that $\deg D < \deg B_n$.  Determining $D$ and $E$,
reduces the degree of the denominator of the integral.  Repeating this
procedure will eventually result in the integral of a rational function
which has a transcendental integral.  The algorithm of section 1.3 is
applicable here.

To simplify the notation slightly we will drop the functional representation
for now.  Differentiating \eqnref{Int:Mack:PF:Eq} gives
\[
{A \over B} =
  {D' \over B_n^{n-1} C} + {(1-n) B_n' D \over B_n^n C} -
  {D C' \over B_n^{n-1} C^2} + {E \over B_n^{n-1} C}.
\]
Multiplying both sides by $B = B_n^n C^2$
\[
CA = CD' B_n + (1-n) B_n' CD - D C' B_n + E CB_n.
\]
Now we can reduce everything modulo $B_n$
\[
CA = (1 - n) B_n' CD \mod{B_n}
\]
Since, $C(x)$ and $B_n(x)$ are relatively prime we have
\[
CA = (1 - n) B_n' CD \mod{B_n}
\]
from which $D$ can be computed straight away.  Subtracting the
rational functions leaves $E/B_n^{n-1}C$ to which this algorithm can
again be applied.

\section{Transcendental Part of the Integral}

Let $P(t)/Q(t)$ be a rational function with coefficients in some field $k$.
By the techniques of the previous two sections we can now assume that $Q(t)$
is square free.  Thus the integral of this function will be a sum of
logarithms.  In principal, we could just factor $Q(t)$ in its splitting
field and then use the previous techniques to write
\[
\int {P(t) \over Q(t)} \, dt = \beta_1 \log (t - \alpha_1) + \cdots +
\beta_n \log (t - \alpha_n).
\]
This representation of the integral can be expensive.  It requires $Q(t)$ be
factored into linears and potential forces the use of rather high degree
algebraic extensions of $k$.  Instead of expressing the integral as a sum of
logarithms of linear polynomials we will use logarithms of higher degree
polynomials.  This will allow us to use a lower degree extension to express
the integral.  

Since $Q(t)$ is a square-free, it has no multiple poles.  Like all rational
functions, $P(t)/Q(t)$ can be expressed as a sum of its principal parts
\[
{P(t) \over Q(t)} =
{\beta_1 \over t - \alpha_1} + \cdots + {\beta_n \over t - \alpha_n}.
\]
Notice the similarity between this and the previous equation.  The
$\beta_i$ are the residues of $P(t)/Q(t)$ at its poles.  We will first
concentrate on how to compute the residues of $P(t)/Q(t)$.  This will allow
us to generate a field in which they all lie.  It will turn out that this
field is the largest field required to express the integral.

The residue of $P(t)/Q(t)$ at $\alpha$ is defined to be
\[
\lim_{t \rightarrow \alpha} {P(t) \over Q(t)}\,(t - \alpha).
\]
Let the zeroes of $Q(t)$ be $\alpha_1, \ldots, \alpha_n$.  Then 
\[
{P(t) \over Q(t)} =
{P(t) \over (t - \alpha_1) (t - \alpha_2) \cdots (t - \alpha_n)},
\]
and the residue at $\alpha_1$ is
\[
{P(\alpha_1) \over (\alpha_1 - \alpha_2) (\alpha_1 - \alpha_3)
\cdots (\alpha_1 - \alpha_n)}.
\]
Differentiating the product representation of $Q(t)$, gives
\[
Q'(t) = (t - \alpha_2) \cdots (t - \alpha_n) +
(t - \alpha_1) (t - \alpha_3) \cdots (t - \alpha_n) + \cdots +
(t - \alpha_1) \cdots (t - \alpha_{n-1}).
\]
Thus the residue of of $P(t)/Q(t)$ at $\alpha_1$ is 
$P(\alpha_1)/Q'(\alpha_1)$.  

Therefore, given a rational function $P(t)/Q(t)$, with $Q(t)$ square-free,
the residue of $P(t)/Q(t)$ at a zero of $Q(t)$, $\alpha$ is the value of a
rational function $P(\alpha)/Q'(\alpha)$.  In fact we can do
slightly better than this.

\begin{proposition}
Let $P(t)$ and $Q(t)$ be polynomials with coefficients in a field $k$,
$Q(t)$ square-free.  There exists a polynomial $B(t)$ such that the
residue of $P(t)/Q(t)$ at a pole $\alpha$ is $B(\alpha)$.
\end{proposition}

\begin{proof}
At $\alpha$ we must have $B(\alpha) = P(\alpha) / Q(\alpha)$.  Use the
Euclidean algorithm to choose $A(t)$ and $B(t)$ so as to satisfy
\[
A(t) Q(t) + B(t) Q'(t) = P(t).
\]
Since $Q(t)$ is square-free, $Q(t)$ and $Q'(t)$ are relatively
prime and a solution always exists.
\end{proof}

If $Q(t)$ is square-free we can write the integral of $P(t)/Q(t)$ as
\[
\int {P(t) \over Q(t)} \,dt = {B(\alpha_1) \over t - \alpha_1}  +
{B(\alpha_2) \over t - \alpha_2} + \cdots +
{B(\alpha_n) \over t - \alpha_n}.
\]
If $B(\alpha_i) = B(\alpha_j)$ it is possible to collapse the two
logarithm terms into a logarithm of a quadratic.  For instance,
\[
\int {t^2 \over t^3 - 2} \, dt = {1 \over 3} \log (t^3 -2).
\]
Notice that is not necessary to use elements of $\Q(\root 3 \of{2})$.  This is
because the residues at the three cube roots of $2$ were all $1/3$.

The following proposition due to {\Trager} \cite{Trager76b}, makes this
more precise.

\begin{proposition}[Trager]
Let $P(t)/Q(t)$ be a rational function over $k$ and $Q(t)$ irreducible
over $k$.  If all the residues of $P(t)/Q(t)$ are contained in $k$,
then the non-zero residues are equal.
\end{proposition}

\begin{proof}
Since Q(t) is irreducible its \key{Galois group} is transitive.  Let
the roots of $Q(t)$ be $\alpha_1, \ldots, \alpha_n$, and $\sigma_i
\alpha_1 = \alpha_i$.  Let $B(t)$ be defined as in Proposition 1.
Recall that the coefficients of $B(t)$ are elements of $k$.  Let $c =
B(\alpha_1)$ be the residue at $\alpha_1$, which is assumed to be an
element of $k$.  The residue at $\alpha_i$ is
\[
B(\alpha_i) = B(\sigma_i \alpha_1) = \sigma_iB(\alpha_1)= \sigma_i c = c.
\]
since $c$ is an element of $k$.  Thus all the poles have residue $c$. 
\end{proof}

Stating this in terms of integrals, if $Q(t)$ is square-free, all the
residues of $P(t)/Q(t)$ are in in $k$, and $\alpha$ is a zero of $Q(t)$,
then
\[
\int {P(t) \over Q(t)} \, dt = B(\alpha) \log Q(t).
\]

More generally, let $P(t)/Q(t)$ be a rational function with coefficients in
$k$, $Q(t)$ still square-free, but without any restrictions on the field
in which the residues lie, $F$.  How do we determine the field?  If we know
$F$, then we can factor $Q(t)$ over $F$ and use the previous equation to
write down the integral.  This gives us an integration algorithm that uses
the smallest possible algebraic extension.

Let $B(t)$ be as above.  The residues are all conjugates of $B(\alpha)$,
where $\alpha$ is some zero of $Q(t)$.  We can form a polynomial whose
zeroes are the residues by using the resultant
$\res(t - B(x), Q(x), x)$.  The splitting field of this polynomial is field
required to express the integral.  $Q(t)$ is factoed in this field and
the partial fraction decomposition algorithm is used to  separate
$P(t)/Q(t)$ into pieces which satisfy the theorem.  

The following example illustrates these ideas.  Consider the integral
\[
\int {P(t) \over Q(t)} \, dt =
\int {7 t^{13} + 10t^8 +4 t^7 - 7t^6 - 4t^3 - 4t^2 + 3t +3 \over
t^{14} - 2t^8 - 2t^7 - 2t^4 - 4t^3 - t^2 + 2t +1} \, dt.
\]
The residue polynomial $B(x)$ is
\[
B(x) = {1 \over 2} (t^{12} - t^{11} + t^{10} - t^9 + t^8 - t^7 - t^6 -
2t^2 - 2t +2).
\]
The resultant $\res(t - B(x), Q(x), x)$ gives
\[
\begin{aligned}
  16384 t^{14} &- 114688 t^{13} + 315392 t^{12}
      401408 t^{11} + 164864 t^{10} + 121856 t^9 - 109312 t^8 \\
    & -23552 t^7 +27328 t^6 + 7616 t^5 - 2576 t^4 - 1568 t^3
      -308 t^2 - 28 t -1 
\end{aligned}
\]
which is $(4 t^2 - 4t -1)^7$ and has 
\[
{1 \pm \sqrt{2} \over 2}.
\]

Factoring $Q(t)$ over $\Q(\sqrt{2})$ gives
\[
Q(t) = (t^7 - \sqrt{2} t^2 - (1 + \sqrt{2}) t - 1)
(t^7 + \sqrt{2} - (1 - \sqrt{2}) t -1).
\]
So the integral is
\[
{1 + \sqrt{2} \over 2} \log (t^7 - \sqrt{2} t^2 - (1 + \sqrt{2}) t - 1)
+ {1 - \sqrt{2} \over 2} \log (t^7 + \sqrt{2} - (1 - \sqrt{2}) t -1).
\]

