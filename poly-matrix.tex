%$Id: poly-matrix.tex,v 1.1 1992/05/10 19:39:20 rz Exp rz $
\chapter{Matrices of Polynomials}
\label{Poly:Matrix:Chap}

\section{Determinants}
\label{PMat:Det:Sec}

\section{Smith and Hermite Normal Forms}
\label{PMat:Smith:Sec}

In this section we give a more detailed description of the process of
computing the Smith and Hermite normal forms of matrices over rings.
Throughout this section we will assume the entries of the matrices
are elements of a ring $R$ and that we compute GCD's (and canonical
associates) over $R$.  The set of $m$ by $n$ matrices over $R$ will be
denoted by $R^{m \times n}$.  Throughout the following we will assume
that $A$ and $B$ are elements of $R^{m \times n}$. 

\begin{definition}
$A$ and $B$ are \keyb{row equivalent} if either of the following two
equivalent conditions are met.
\begin{itemize} 
\item There exists a unimodular matrix $P$ such that $A = PB$.
\item The $R$-modules generated by the rows of $A$ and $B$ are the
same.
\end{itemize}
In this case we write $A \cong_{\rm row} B$.
\end{definition}

\begin{definition}
$A$ and $B$ are \keyb{equivalent}, written $A \cong_{\rm equiv} B$, if there
exist unimodular matrices $P$ and $Q$ such that $A = PBQ$.
\end{definition}

\begin{definition}
Let $H =(h_{ij}) \in R^{m \times n}$ and denote the rank of $H$ by
$r$.  $H$ is in \keyb{Hermite normal form} if there exists indices $j_1,
\ldots, j_r$ such that
\begin{itemize}
\item All but the first $r$ rows of $H$ are contain only zeroes.
\item $h_{ij_i} \not= 0$ and $h_{i\ell} = 0$ for $1 \le \ell < j_i$.
\end{itemize}
\end{definition}

The elements $h_{i j_i}$ in the above definition are called the {\em
special elements} of $H$.

\begin{definition}
The $s$\th{} \keyb{determinantal divisor}, $\delta_s(A)$, is defined
to be the GCD of all of the $s\times s$ submatrices of $A$.
$\delta_0(A)$ is defined to be $1$.
\end{definition}

\begin{definition}
The \keyb{invariant factors} of $A$ are defined to be
\[
\gamma_s(A) = \frac{\delta_s(A)}{\delta_{s-1}(A)}.
\]
\end{definition}

\begin{definition}
Let $r$ denote the rank of $A$ and $\gamma_i$, for $1 \le i \le r$ its
invariant factors.  Then the \keyb{Smith normal form} of $A$ is
\[
A \cong_{\rm equiv} 
  \begin{pmatrix}
      \gamma_1 & 0 & \cdots &0 &\cdots &0\cr 
      0 & \gamma_2& \cdots & 0 &\cdots &0\cr 
      \vdots & & \cdots &  &\cdots &\vdots\cr
      0 & 0 & \cdots & \gamma_r& \cdots & 0 \cr 
      0 & 0 & \cdots & 0 & \cdots & 0 \cr 
      \vdots & & \cdots &  &\cdots &\vdots\cr
      0 & 0 & \cdots & 0 & \cdots & 0 \cr
  \end{pmatrix}.
\]
\end{definition}


\begin{proposition}
Let $A$ be an $m \times n$ matrix over $R$, the rank of $A$ be
$r$ and $d = \delta_r(A)$.  Then 
\[
\begin{pmatrix} A \cr dI_n\cr \end{pmatrix}
  \cong_{\rm row} \begin{pmatrix} A \cr 0 \cr \end{pmatrix}.
\]
\end{proposition}

\begin{proof}
Exhibit the map when $A$ is in HNF and then use a unimodular
transformation to get back and forth between $A$ and its HNF.
\end{proof}


This gives an algorithm by reducing the entries by the determinant at
each stage.
