%$Header: /usr/u/rz/AMBook/RCS/intro.tex,v 1.1 1992/02/13 22:47:39 rz Exp rz $
\chapter{Introduction}

The modern era of computational science has made great strides by
reducing many scientific questions to calculations with floating point
(fixed precision) numbers. And while the hammer of numeric computation
has been enormously successful, not all problems are nails. Genome
sequencing, cryptography, finding information on the Internet are
among those problems for which other mathematical techniques are
needed.

Algebraic manipulation, or computer algebra, is the study of
algorithms for manipulating mathematical quantities.  What
distinguishes it from other areas of mathematical computation is the
attention paid to accurately modeling the mathematical semantics of
the objects being manipulated.  The first manifestation of this is
that while conventional computation systems define an integer to be a
whole number with absolute value less then $2^{31}$ or $2^{63}$
typically, algebraic manipulation systems define integers to be whole
numbers, with no limit set to their size.  Though this distinction
makes little difference in many computations, algebraic manipulations
focuses on those techniques and problems where exactness or the
ability to model a complex mathematical concept is important.

This book discusses the basic algorithms for manipulating mathematical
quantities and how they are implemented.  Mathematical quantities divide
into two basic types, exact and inexact.  Among the exact quantities are
rational integers, algebraic numbers, polynomials, differential forms and
vector fields.  Inexact quantities include truncated power series and
Fourier series, and continued fractions.  Computing with these types of
quantities instead of floating point numbers and fixed precision integers,
yields three benefits.  First, the computation can take place exactly. Thus
if successful, one can prove than an answer is exactly $\sqrt{3}$ or
$1/137$ not just that it is close to these values.  Second, symbolic
quantities represent an entire function, not just its value at a point.
This allows more global information about the behavior of a system to be
determined.  Finally, symbolic quantities can represent and manipulate
mathematical abstractions that are not usually considered part of
computational mathematics, like rings, ideals, Riemann surfaces and
homology groups.

This book discusses the algorithms for performing symbolic
mathematical calculations.  We will be dealing not only with large
integers, but with algebraic numbers, polynomials and transcendental
functions.  Systems with these capabilities are often called {\em
algebraic manipulation} systems and the field we will be studying is
called {\em algebraic manipulation\/}, {\em computer algebra} or {\em
symbolic computation}.  For example, an algebraic manipulation system
could take an expression like
\[
S = \sin^2 x + \cos^2 x
\]
differentiate it to get
\[
2 \sin x \cos x + 2 \cos x \, (- \sin x) = 0.
\]
From this it is clear that $\sin^2 x + \cos^2 x$ is a constant.  To
determine to which constant $S$ is equal, we ask the system to evaluate $S$
at $\pi$:
\[
\sin^2 \pi + \cos^2 \pi = 0^2 + (-1)^2 = 1.
\]
Notice that the value of $\cos \pi$ is exactly $1$---there is no
round-off error and it is not equal to $1.00000\pm\epsilon$.  Thus
$\sin^2 x + \cos^2 x = 1$.  This is a trivial example of one of the
ways algebraic manipulation systems can be used to solve problems.

Current algebraic manipulation systems have been designed to perform long
tedious calculations that, at one time, were the responsibility of graduate
students.  They can compute integrals, factor polynomials and develop power
series expansions.  They are called upon to solve differential equations
and invert matrices.  At one time all of the calculations performed by
algebraic manipulation systems were exact, although now approximation
techniques are proving to be very valuable.  Truncated power series and
arbitrary precision floating point calculations are often provided and have
been useful.  Even conventional floating point techniques are being used in
a mixed mode of computation, part of the calculation being numeric and part
being symbolic.

\medskip

Here we outline the basic algorithms and techniques algebraic manipulation
systems use to perform calculations.  We do not attempt to describe how
these systems are used to solve real problems.  But we feel that by
understanding the mechanisms used, it will be far easier to understand the
limitations and capabilities of algebraic manipulation systems.

