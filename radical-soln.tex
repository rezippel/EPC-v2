%$Id: radical-soln.tex,v 1.1 1992/05/10 19:44:06 rz Exp echris $
\chapter{Solutions in Terms of Radicals}
\label{Radical:Soln:Chap}

One of the fascinating challenges of renaissance mathematics was to
express the the zeroes of a polynomial in terms of radicals.  In the
fifteenth and sixteenth centuries, formulas for the zeroes of cubic
and quartic polynomials protected as valuable secrets, which they were
since they allowed on to excel and the public equation solving
contests.  For centuries, a major problem was to determine similar
formulas for higher degree equations.

All this came to a halt with the work of {\Abel} \cite{Abel26} who
proved that there existed polynomials of degree five whose zeroes
could not be expressed in terms of radicals.  Finally, {\Galois}
\cite{Galois62} developed a theory that detailed the possible polynomial
relations among the zeroes of a polynomial.  

Currently, with our understanding of Galois theory and ability to
compute with general algebraic expressions, it is less important than
ever to be able to express zeroes of polynomials in terms of radicals.
Nonetheless, it remains an interesting problem because people tend
have better intuition about (moderate sized) expressions involving
radicals.

In this chapter we consider several aspects of solving equations in
terms of radicals.

\section{Palindromic Equations}
\label{Palindrome:Eq}

A polynomial whose coefficients form a ``palindomic'' sequence, \ie, 
\[
f(X) = a_0 X^{2n} + a_1 X^{2n-1} + a_2 X^{2n-2} + \cdots +
a_2 X^2 + a_1 X + a_0,
\]
is called a \key{palindomic polynomial}.  The zeroes of such
polynomials can be expressed in terms of polynomials of degree $n/2$
using only square roots.  This is achieved as follows.  We can write
\[
\frac{f(X)}{X^{n/2}} = a_0 \left(X + \frac{1}{X}\right)^{n} 
  + a_1 \left(X + \frac{1}{X}\right)^{n-1}  
  + (a_2 - n a_0) \left(X + \frac{1}{X}\right)^{n-2} + \cdots.
\]
Let $\alpha_1, \ldots, \alpha_n$ be the zeroes of
\[
g(X) = a_0 X^n + a_1 X^{n-1} + (a_2 - n a_0) X^{n-2} + \cdots .
\]
Then 
\[
\frac{\alpha_i \pm \sqrt{\alpha_i^2 -4}}{2}, \quad \mbox{for $i =
1,\ldots, n$}
\]
are the zeroes of $f(X)$.

If
\[
g(X) = b_0 X^{n} + b_1 X^{n-1} + \cdots + b_n
\]
such that
\[
f(X) = X^n g\left(X + \frac{1}{X}\right),
\]
then the $b_i$ must satsify the following triangular systems of
equations:
\[
\begin{aligned}
a_0 &= b_0, \\
a_2 &= b_2 + n b_0, \\
a_4 &= b_4 + (n-2) b_2 + \frac{n(n-1)}{2} b_0, \\
a_6 &= b_6 + (n -4) b_4 + \frac{(n-2)(n-3)}{2} b_2 +
\frac{n(n-1)(n-2)}{3!} b_0, \\
 & \vdots
\end{aligned}
\]

\[
\begin{aligned}
a_k& = b_k + \sum_{i=1}^{k/2} {n - k + 2i \choose i} b_{k-2i}, \\
a_{k+1}& = b_{k+1} + \sum_{i=1}^{k/2} {n - k + 2i -1 \choose i}
b_{k-2i+1}.
\end{aligned}
\]


\section{Zeroes of Cyclic Equations}
\label{Cyclic:Zeroes:Sec}

This section gives an algorithm for expressing the zeroes of cyclic
polynomials in terms of radicals.

Let $\zeta_n$ be a primitive $n${\th} root of unity. $\Q(\zeta_n)$ is
a normal abelian extension of $\Q$ of degree $\phi(n)$.  Since the
Galois group is abelian it can be decomposed into a direct sum of
cyclic subgroups.  This decomposition corresponds to representing the
field $\Q(\zeta_n)$ as a compositum of cyclic extensions of $\Q$.  If
$n = ab$ represents a nontrivial factorization of $n$ with
$\gcd(a,b)=1$ and $ a,b \in \Z$, then $\zeta_a\zeta_b$ is a primitive
$n${\th} root of unity.  This permits us to assume $n$ to be a prime
power $p^k$.  Given a primitive $p${\th} root of unity $\zeta_p$,
$(\zeta_p)^{p^{1-k}}$ is a primitive $p^k${\th} root of unity. Thus if
we can find a radical representation for a primitive $p${\th} root of
unity where $p$ is any prime, then we can find the same for any
$n${\th} root of unity.  Note that given a particular primitive
$n${\th} root of unity, $\zeta_n$, all others can be expressed as
$\zeta_n^k$ where $0<k<n$ and $\gcd(k,n)=1$.

\begin{figure}
\[
\begin{diagram}
\node[2]{\Q(\zeta_{p^2 q r^3}) 
     = \Q(\zeta_{p^2} \zeta_{q} \zeta_{r^3})} \arrow{sw,-} \arrow{sse,-} \\
\node{\Q((\zeta_{p})^{1/p}) = \Q(\zeta_{p^2})} \arrow{s,-} \\
\node{\Q(\zeta_q)}\arrow{s,-} 
   \node[2]{\Q((\zeta_{r})^{1/r^2}) = \Q(\zeta_{r^3})} \arrow{s,-} \\
\node{\Q(\zeta_p)} \arrow{se,-} \node[2]{\Q(\zeta_r)} \arrow{sw,-}\\
\node[2]{\Q}
\end{diagram}
\]
\end{figure}




Thus the real task is to express primitive $p${\th} roots of unity in
radicals.  We will assume inductively that we can express primitive
$n${\th} roots of unity in radicals for any $n<p$, noting that
$-1^2=1$ yields a basis for the induction.  $\Q(\zeta_p)$ is of degree
$p-1$ over $\Q$, with a cyclic galois group that is isomorphic to
$\F_p^{\ast}$.  Our approach is based on the constructive properties
of the following proposition.

\begin{proposition}
Let $E$ be a cyclic extension of degree $n$ over a field $F$
containing the $n${\th} roots of unity where $n$ is not divisible by
the characteristic, then $E = F(x)$ where $x^n \in F$.
\end{proposition}

Before applying this theorem we will need to adjoin the $p-1${\th} roots
of unity to $\Q$ but by assumption they are already expressible in
radicals.  Since $\Q(\zeta_{p(p-1)}) = \Q(\zeta_p,\zeta_{p-1})$ and
$\phi(p(p-1)) = \phi(p) \phi(p-1)$, the minimal polynomial for
$\zeta_p$ over $\Q$ is still irreducible over $\Q(\zeta_{p-1})$ and
the Galois group, $G$, of $\Q(\zeta_{p(p-1)})$ over $\Q(\zeta_{p-1})$
is isomorphic to the group of $\Q(\zeta_p)$ over $\Q$.  If we were to
use this algorithm for more general cyclic extensions, then we might
need to perform an algebraic factorization at this point, but in our
case we can skip it.

Let $\sigma$ be a generator of the cyclic group $G$. Then $\sigma$
must send $\zeta_p$ to $\zeta_p^j$ where $1<j<p$ since the image must
also be a primitive $p${\th} root of unity.  Since $\sigma$ generates
$G$, $j$ must be a generator of $\F_p^{\ast}$.
We want to find the $x$ whose existence is guaranteed by the theorem.
$\sigma$ must map such an $x$ to $\xi x$ where $\xi$ is a primitive
$p-1${\th} root of unity, and conversely any nonzero element $y \in E$
that is mapped to $\xi y$ has the property that $E = F(y)$ and
$y^{p-1} \in F$.  We claim the following $y$ will work:
\begin{equation} \label{Rad:HilGen:Eq}
y=\sum_{i=1}^{p-1} {\sigma^i(\zeta_p) \over \xi^i}.
\end{equation}
Applying $\sigma$ and dividing by $xi$ only increases the summation
lints:
\[
{\sigma(y) \over \xi} = \sum_{i=2}^p {\sigma^i(\zeta_p) \over
\xi^i} = \sum_{i=1}^{p-1} {\sigma^i(\zeta_p) \over \xi^i},
\]
since $\sigma$ and $\xi$ both have order $p-1$, $\sigma^p=\sigma$ and
$\xi^p=\xi$. Thus, $\sigma(y) = \xi y$.  

We only need to show that $y$ is nonzero. $\sigma^i(\zeta_p)$ for
$1\leq i\leq p-1$ produces a permutation of the powers of $\zeta_p$
between 1 and $p-1$.  If $y$ is zero then $\zeta_p$ satisfies a
polynomial of degree $p-1$ whose constant term is zero.  Since
$\zeta_p \not = 0$ this implies that $\zeta_p$ satisfies a polynomial
of degree $p-2$ with coefficients in $\Q(\xi)$.  This contradiction
implies $y$ is nonzero.  In more general cyclic extensions of the form
$F(w)$ we would have substituted successive powers of $w$ for
$\zeta_p$ in the above formula until we found one that yielded a
non-zero $y$.  This would require at most $p-1$ such computations.

We have two remaining tasks left. First we must express $y$ in terms
of radicals, then we must express $\zeta_p$ in terms of $y$. Since
$y^{p-1}$ is invariant under $\sigma$ it must be an element of
$\Q(\zeta_{p-1})$. We compute $y^{p-1}$ and reduce by the minimal
polynomial for $\zeta_p$. This expresses $y$ as a $p-1${\th} root of
an element of $\Q(\zeta_{p-1})$ and by the induction hypothesis we can
express $\zeta_{p-1}$ in terms of radicals.

In a similar fashion we compute $y^k$ for $0<k<p-1$ as polynomials in
$\zeta_p$ of degree less than $p-1$.  If we now consider $\zeta_p^j$
for $0<j<p-1$ as a set of variables, we have $p-2$ linear equations in
$p-2$ variables.  This system has a unique solution that we can find
by linear algebra performed over the field $\Q(\zeta_{p-1})$.  This
expresses $\zeta_p$ in terms of $y$ and $\zeta_{p-1}$ both of which
were already expressed in terms of radicals.

Note that while we have demonstrated the capability of expressing
$n${\th} roots of unity in terms of radicals, we do not advocate
performing computations using this representation.  Questions of zero
recognition and the desire to minimize expression size makes
performing the arithmetic via minimal polynomials much more
attractive.  Only for the purpose of displaying the final answer to
the user should radical representation be used.

\subsection{Primitive Cube Root of Unity}
We illustrate the approach of this section with two examples.  This
subsection determines the form in radicals of the cube root of $1$,
$\zeta_3$, while the next subsection considers $\zeta_5$.  The cube
root of unity is particularly simple, since it is the zero of a
quadratic equation:
\[
X^2 + X + 1 =0.
\]
From this it is clear that 
\[
\zeta_3 = \frac{-1 \pm \sqrt{-3}}{2}.
\]

According to the theory, we will be able to express $\zeta_3$ in terms
of radicals over $\Q[\zeta_2]$.  Since $\zeta_2 = -1$, we will
actually be working over $\Q$.  The first step is to determine a
generator for the Galois group of $\Q[\zeta_3]$ over $\Q$.  Since this
is an extension of degree $2$, there is only one choice
\[
\sigma : \zeta_3 \mapsto \zeta_3^2.
\]
Using \eqnref{Rad:HilGen:Eq}, with $p = 3$, $\xi = \zeta_2 = -1$, we have
\begin{equation}\label{Zeta3:Eq}
\begin{aligned}
  y &= \frac{\sigma(\zeta_3)}{-1} + \frac{\sigma^2(\zeta_3)}{1} 
    = -\zeta_3^2 + \zeta_3^4, \\
    & = 1 + 2\zeta_3.
\end{aligned}
\end{equation}

We claim that $\sigma(y) = \zeta_2 y$:
\[
\sigma(y) = 1 + 2\zeta_3^2 = 1 + 2(- \zeta -1) = -1 - 2\zeta.
\]
Thus, $\sigma(y^2) = \zeta_2^2 y = y$ and $y^2$ must be an element of
$\Q$.
\[
y^2 = 1 + 4 \zeta_3 + 2\zeta_3^2 = -3.
\]
Using \eqnref{Zeta3:Eq} we have
\[
\zeta_3 = \frac{-1 + \sqrt{-3}}{2}.
\]

\subsection{Primitive Fifth Root of Unity}
The fifth root of unity somewhat more involved, $\zeta_5$

\[
\begin{aligned}
y &= 1 + 2 \zeta_5 + (1+ i) \zeta_5^2 + (1 - i) \zeta_5^3, \\
y^2 &= 1 + 2i + (2+4i) \zeta_5^2 + (2 + 4i) \zeta_5^3, \\
y^3 & = 3-4i + (6-8i)\zeta_5 - (1 + 7i) \zeta_5^2 + (7 - i)\zeta_5^3.
\end{aligned}
\]

\[
y^4 = -15 + 20 i
\]

\[
\zeta_5 = -\frac{1}{4} + \frac{1}{4} \left(20 i -15\right)^{1/4}
 - \frac{1-2i}{20} \left(20 i -15\right)^{1/2}
 + \frac{3+4i}{100} \left(20 i -15\right)^{3/4}.
\]

\section{Quintics}
{\em See Young \cite{Young88} and Dummit \cite{Dummit91}}

Example:
\[
X^5 - 10 X^3 - 20X^2 - 1505 X - 7412 = 0
\]

\[
\begin{aligned}
X &= \sqrt[5]{9}\sqrt[5]{(197 + 139 \sqrt{2}) + \sqrt{(197+139\sqrt{2})^2
    -\frac{1}{81}(1 + \sqrt{2})^5}} \\
 &+ \sqrt[5]{9}\sqrt[5]{(197 + 139 \sqrt{2}) - \sqrt{(197+139\sqrt{2})^2
    -\frac{1}{81}(1 + \sqrt{2})^5}} \\
 &+ \sqrt[5]{9}\sqrt[5]{(197 + 139 \sqrt{2}) + \sqrt{(197-139\sqrt{2})^2
    -\frac{1}{81}(1 + \sqrt{2})^5}} \\
 &+ \sqrt[5]{9} \sqrt[5]{(197 + 139 \sqrt{2}) - \sqrt{(197-139\sqrt{2})^2
    -\frac{1}{81}(1 + \sqrt{2})^5}}
\end{aligned}
\]

\section{Solvability in Terms of Radicals}

See \cite{Young88,Cayley??}

\section*{Notes}

\footnotesize

\notesectref{Cyclic:Zeroes:Sec} This discussion is based on notes due
to {\Trager}. 

\normalsize
