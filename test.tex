%% The bulk of this file is abbreviations used in the book.  At the
%% end there is some additional 

\usepackage{amsmath}
\usepackage{amssymb}

%% Commutative diagrams
%\usepackage{amscd}
\usepackage{tikz-cd}
%\usetikzlibrary{matrix,arrows,decorations.pathmorphing}

% Hebrew fonts
\usepackage{calrsfs}
\usepackage{semitic}

%%%%%% Font support
% Lucida
%\usepackage[T1]{lucidabr}
%\usepackage{lucidabr}

%CM Bright
\usepackage{cmbright}

%% Palantino/Euler
%\usepackage{pxfonts}
%\usepackage{eulervm}

%% Concrete Math and EulerVM
%\usepackage [ T1 ]{ fontenc } % Needed for Type1 Concrete
%\usepackage {concrete} % Loads Concrete + Euler VM

% Kurier
%\usepackage[math]{kurier}

% Palatino
%\usepackage{pxfonts}

% Arev sans and Arev Math
%\usepackage{arev}

% Bitstream Charter with Math Design math
%\usepackage[charter]{mathdesign}

% Utopiawith Math Design math
%\usepackage[utopia]{mathdesign}

% Artemsia with Euler Math
%\usepackage{gfsartemisia-euler}
%\usepackage[T1]{fontenc}

\usepackage{graphicx}
\graphicspath{{Pix/}}

%\usepackage{makeidx}
\usepackage{makeidx,showidx}
\usepackage{xcite}
\usepackage[font=small,totoc=true]{idxlayout}

\setlength{\textwidth}{11.5cm}
\setlength{\textheight}{19cm}
\setlength{\topmargin}{24pt}
\setlength{\headheight}{12pt}
\setlength{\headsep}{20pt}
\setlength{\footskip}{24pt}

\usepackage{losymbol}


\newcommand{\mathify}[1]{\ifmmode{#1}\else\mbox{$#1$}\fi}
\newcommand{\bbth}[1]{\mathify{{}^{{{\textit{#1}}}}}}
\newcommand{\rd}{\bbth{rd}}
\newcommand{\thHigh}{\bbth{th}}
\newcommand{\nd}{\bbth{nd}}
\newcommand{\st}{\bbth{st}}

% Math symbols
\newcommand{\A}{\mathbb{A}}    % Affine space
\newcommand{\C}{\mathbb{C}}
\newcommand{\F}{\mathbb{F}}
%\newcommand{\N}{\mathbb{N}}
\newcommand{\Q}{\mathbb{Q}}
\newcommand{\R}{\mathbb{R}}
\newcommand{\Z}{\mathbb{Z}}

\newcommand{\cont}{\mathop{\rm cont}\nolimits}
\newcommand{\prim}{\mathop{\rm prim}\nolimits}
\newcommand{\Lt}{\mathop{\rm Lt}\nolimits}
\newcommand{\lt}{\mathop{\rm lt}\nolimits}
\newcommand{\lc}{\mathop{\rm lc}\nolimits}
\newcommand{\lexp}{\mathop{\rm le}\nolimits}
\newcommand{\terms}{\mathop{\rm terms}\nolimits}
\newcommand{\coef}{\mathop{\rm coef}\nolimits}
\newcommand{\dens}{\mathop{\rm dens}\nolimits}
\newcommand{\skel}{\mathop{\rm skel}\nolimits}
\newcommand{\sep}{\mathop{\rm sep}\nolimits}
\newcommand{\Char}{\mathop{\rm char}\nolimits}
\newcommand{\Mor}{\mathop{\rm Mor}\nolimits}
\newcommand{\End}{\mathop{\rm End}\nolimits}
\newcommand{\prem}{\mathop{\rm prem}\nolimits}
\newcommand{\lcm}{\mathop{\rm lcm}\nolimits}
\newcommand{\num}{\mathop{\rm num}\nolimits}
\newcommand{\den}{\mathop{\rm den}\nolimits}
\newcommand{\res}{\mathop{\rm res}\nolimits}
\newcommand{\val}{\mathop{\rm val}\nolimits}
\newcommand{\Tr}{\mathop{\rm Tr}\nolimits}
\newcommand{\Ord}{\mathop{\rm ord}\nolimits}
\newcommand{\Res}{\mathop{\rm Res}\nolimits}
%\newcommand{\Mult}{\mathop{\rm Mult}\nolimits}
\newcommand{\Spec}{\mathop{\rm Spec}\nolimits}
\newcommand{\coker}{\mathop{\rm coker}\nolimits}
\newcommand{\Gal}{\mathop{\rm Gal}\nolimits}
\newcommand{\Norm}{\mathop{\mathbf{N}}\nolimits}
\newcommand{\Dscr}{\mathop{\mathbf{D}}\nolimits}
\newcommand{\Logint}{\mathop{\rm Li}\nolimits}
\newcommand{\unifdeg}{\mathop{\rm unifdeg}\nolimits}
\newcommand{\fieldDegree}[2]{[#1\!:\!#2]}
\newcommand{\groupDegree}[2]{(#1\!:\!#2)}
\newcommand{\legendre}{\overwithdelims()}
\newcommand{\dist}{\mathop{\rm dist}\nolimits}
\newcommand{\vol}{\mathop{\rm vol}\nolimits}

\newtheorem{theorem}{Theorem}
\newtheorem{lemma}[theorem]{Lemma}
\newtheorem{proposition}[theorem]{Proposition}
\newtheorem{definition}{Definition}
\newtheorem{corollary}{Corollary}
\newtheorem{conjecture}{Conjecture}
\newtheorem{fact}{Fact}
\newtheorem{problem}{Problem}
\newenvironment{proof}{\noindent{\bf Proof:}}{\unskip~~$\QEDbox$\medskip}
\def\QEDbox{\fbox{\rule{0ex}{1ex}}}

% Cross referencing commands
\newcommand{\chapref}[1]{Chapter~\ref{#1}}
\newcommand{\sectref}[1]{Section~\ref{#1}}
\newcommand{\appenref}[1]{Appendix~\ref{#1}}
\newcommand{\thmref}[1]{Theorem~\ref{#1}}
\newcommand{\propref}[1]{Proposition~\ref{#1}}
\newcommand{\conjref}[1]{Conjecture~\ref{#1}}
\newcommand{\corref}[1]{Corollary~\ref{#1}}
\newcommand{\defref}[1]{Definition~\ref{#1}}
\newcommand{\lemref}[1]{Lemma~\ref{#1}}
\newcommand{\figref}[1]{Figure~\ref{#1}}
\newcommand{\tableref}[1]{Table~\ref{#1}}
\newcommand{\eqnref}[1]{(\ref{#1})}

% English things in Latin
\newcommand{\ie}{{\em i.e.}}
\newcommand{\viz}{{\em viz.}}
\newcommand{\cf}{{\em cf.}}
\newcommand{\Eg}{{\em E.g.}}
\newcommand{\eg}{{\em e.g.}}
\newcommand{\etal}{{\em et.al.}}

%% Abbreviations used to simplify creating an index entry while
%% defining a term. 
\def\keyi#1{\textit{#1}\index{#1}}
\def\keyb#1{\textbf{#1}\index{#1}}
\def\keyw#1{\texttt{#1}\index{#1@\protect\texttt{#1}}}
\def\key#1{{#1}\index{#1}}

% The following  were intended to be used in in the index, but there
% are better ways to do this now. I'm commenting these out here to
% make sure they don't get used and have included the better approach
% below. 
%\def\bold#1{{\bf #1}}
% \index{cubic|texbf}
%\def\see#1{{\it see} #1}
%\index{cubic|see{Gauss}}

\def\Altran{{\sc Altran}\index{Altran@\sc Altran\rm}}
\def\Alpak{{\sc Alpak}\index{Alpak@\sc Alpak\rm}}
\def\Axiom{{Axiom}\index{Axiom\rm}}
\def\Camal{{\sc Camal}\index{Camal@\sc Camal\rm}}
\def\CLisp{{\sc Common Lisp}\index{Common Lisp@\sc Common Lisp\rm}}
\def\Formac{{\sc Formac}\index{Formac=\sc Formac\rm}}
\def\Lisp{{\sc Lisp}\index{Lisp=\sc Lisp\rm}}
\def\Macsyma{{\sc Macsyma}\index{Macsyma@\sc Macsyma\rm}}
\def\Maple{{\sc Maple}\index{Maple!@\sc Maple\rm}}
\def\Mathlab{{\sc Mathlab 68}\index{Mathlab@\sc Mathlab 68\rm}}
\def\Mathematica{{\sc Mathematica}\index{Mathematica@\sc Mathematica\rm}}
\def\Reduce{{\sc Reduce}\index{Reduce@\sc Reduce\rm}}
\def\Sac{{\sc Sac}\index{Sac@\sc Sac\rm}}
\def\Saint{{\sc Saint}\index{Saint@\sc Saint\rm}}
\def\Weyl{{Weyl}\index{Weyl}}
\def\pger{{\mathfrak p}}
\def\Pger{{\mathfrak P}}
\def\mger{{\mathfrak m}}

\def\BBoxExp{R_{{\cal B}_P}}

\def\lexord{\ge_{\rm lex}}
\def\totord{\ge_{\rm tot}}
\def\revord{\ge_{\rm rev}}
\def\domord{\ge_{\rm dom}}

%Used to be a \Sloppy here
\def\Marginpar#1{\marginpar{\tiny#1}}  
\marginparwidth 0.75in \marginparsep 4pt 

% Below here is special to Books

\newcommand{\longpropref}[1]{Proposition~\ref{#1} (page~\pageref{#1})}
\newcommand{\notesectref}[1]{\medskip\noindent{\bf \S\ref{#1}}}
\newcommand{\exeref}[1]{Exercise~\ref{#1} (page~\pageref{#1})}
\newcommand{\sign}{\mathop{\rm sign}\nolimits}


\makeatletter
\def\@showidx#1{\insert\indexbox{\tiny 
 \hsize\marginparwidth 
 \hangindent\marginparsep \parindent\z@ 
 \everypar{}\let\par\@@par \parfillskip\@flushglue 
 \lineskip\normallineskip 
 \baselineskip .8\normalbaselineskip\sloppy
 \raggedright \leavevmode 
 \vrule height .7\normalbaselineskip width \z@\relax
 #1\relax\vrule
 height \z@ depth .3\normalbaselineskip width \z@}}

\def\thebibliography#1{\chapter*{Bibliography\@mkboth
 {Bibliography}{Bibliography}} \par
 \addcontentsline{toc}{chapter}{\protect\numberline{Bibliography}}
The pages on which each reference is cited are included in brackets at
the end of each reference. \par\medskip\small\list
 {[\arabic{enumi}]}{\settowidth\labelwidth{[#1]}\leftmargin\labelwidth
 \advance\leftmargin\labelsep
 \usecounter{enumi}}
 \def\newblock{\hskip .11em plus .33em minus -.07em}
 \sloppy\clubpenalty4000\widowpenalty4000
 \sfcode`\.=1000\relax}
\makeatother

\def\mdline#1#2{
  \hbox to\hsize{\hskip\leftmargin\small{\tt (#2)}\hfill$\displaystyle{#1}$\hfill}
  \smallskip}

\newcounter{exercisenum}
\newenvironment{exercise}{ \begin{list}{\arabic{exercisenum}.}\item }%
 {\end{list}\addtocounter{exercisenum}{1}}

\newcounter{algstepnum}
\newenvironment{algorithm}[1]%
  {\begin{list}{\bf #1\arabic{algstepnum}.}%
   {\usecounter{algstepnum}}%
   \setcounter{algstepnum}{0}}%
  {\end{list}}
\newcommand{\stepref}[1]{step~{\bf\ref{#1}}}

\def\begindsacode{%
  \par\begin{center}\begin{minipage}{6in}%
  \begingroup\small\tt\begin{tabbing}12\=34\=\kill}
\def\enddsacode{\end{tabbing}\endgroup\end{minipage}\end{center}}

\usepackage{rzbook}
\usepackage[bookmarks=false,hyperindex=false]{hyperref}
\let\WriteBookMarks\relax
\raggedbottom

\makeindex

%\includeonly{front,zerotest,interp,spinterp,pgcd}
%\includeonly{euclids,contfrac,dio-anal,lattice,arithfun,f-fields}

\setcounter{secnumdepth}{3}     % Number subsubsection's as well
\setcounter{tocdepth}{3}        % Also put in contents 

\def\th{\thHigh}


% Names of people
\def\Abel{Abel\index{Abel, Niels Henrik}}
\def\Adleman{Adleman\index{Adleman, Leonard Max}}
\def\Aho{Aho\index{Aho, Alfred Vaino}}
\def\Alagar{Alagar\index{Alagar, Vangalur S.}}
\def\Amthor{Amthor\index{Amthor, Carl Ernst August}}
\def\Apostol{Apostol\index{Apostol, Thomas}}
\def\Arwin{Arwin\index{Arwin, A.}}
\def\Atiyah{Atiyah\index{Atiyah, Michael, Sir}}
\def\Babai{Babai\index{Babai, L{\'a}szl{\'o}}}
\def\Babbage{Babbage\index{Babbage, Charles}}
\def\BachE{Bach\index{Bach, Eric}}
\def\Bahr{Bahr\index{Bahr, Knut}}
\def\BartonDRa{Barton\index{Barton, David R.}}
\def\BartonDRb{Barton\index{Barton, David R?.}}
\def\BatemanPT{Bateman\index{Bateman, Paul T.}}
\def\Beauzamy{Beauzamy\index{Beauzamy, Bernard}}
\def\BenOr{Ben Or\index{Ben Or, Michael}}
\def\Bentley{Bentley\index{Bentley, Jon Louis}}
\def\Berlekamp{Berlekamp\index{Berlekamp, Elwyn Ralph}}
\def\Besicovitch{Besicovitch\index{Besicovitch, Abraham Samoilovitch}}
\def\BloomS{Bloom\index{Bloom, S.}}
\def\Borodin{Borodin\index{Borodin, Allan Bertram}}
\def\Bourne{Bourne\index{Bourne, Steven R.}}
\def\Brent{Brent\index{Brent, Richard Peirce}}
\def\Brezinski{Brezinski\index{Brezinski, Claude}}
\def\BrownWS{Brown\index{Brown, William Stanley}}
\def\Buchberger{Buchberger\index{Buchberger, Bruno}}
\def\Canny{Canny\index{Canny, John Francis}}
\def\CantorD{Cantor\index{Cantor, David Geoffrey}}
\def\CantorG{Cantor\index{Cantor, Georg Ferdinand Louis Philippe}}
\def\Capelli{Capelli\index{Capelli, A.}}
\def\Carlitz{Carlitz\index{Carlitz, Leonard}}
\def\Carmichael{Carmichael\index{Carmichael, Robert Daniel}}
\def\Cassels{Cassels\index{Cassels, John William Scott}}
\def\Cauchy{Cauchy\index{Cauchy, Augustin Louis}}
\def\Caviness{Caviness\index{Caviness, Bobby Forrester}}
\def\Cayley{Cayley\index{Cayley, Arthur}}
\def\Cerlienco{Cerlienco\index{Cerlienco, L.}}
\def\Chandra{Chandra\index{Chandra, Ashok Kumar}}
\def\CharBW{Char\index{Char, Bruce W.}}
\def\Chebyshev{\v{C}ebyshev\index{Cebyshev@\v{C}ebyshev, Pafnouty L'vovich}}
\def\Chebotarev{\v{C}ebotarev\index{Cebotarev@\v{C}ebotarev, Nikola\u{\i} Grigor'evich}}
\def\Chistov{Chistov\index{Chistov, A. L.}}
\def\Chrystal{Chrystal\index{Chrystal, George}}
\def\CohenP{Cohen\index{Cohen, Paul J.}}
\def\CohenS{Cohen\index{Cohen, S. D.}}
\def\CohnJHE{Cohn\index{Cohn, John H. E.}}
\def\Collins{Collins\index{Collins, George Edwin}}
\def\Cooley{Cooley\index{Cooley, James William}}
\def\Coppersmith{Coppersmith\index{Coppersmith, Don}}
\def\Cramer{Cram\'er\index{Cram\'er, Harold}}
\def\DavenportJ{Davenport\index{Davenport, James Harold}}
\def\DavenportH{Davenport\index{Davenport, Harold}}
\def\Dedekind{Dedekind\index{Dedekind, Richard}}
\def\Deligne{Deligne\index{Deligne, Pierre}}
\def\DeMillo{DeMillo\index{Demillo, Richard A.}}
\def\Descartes{Descartes\index{Descartes, Ren\'{e}e}}
\def\Diffie{Diffie\index{Diffie, Bailey Whitfield}}
\def\Dirichlet{Dirichlet\index{Dirichlet, Peter Gustav Lejeune}}
\def\Dorge{D\"{o}rge\index{Doerge@D\protect\"{o}rge, K.}}
\def\Dumas{Dumas\index{Dumas, G.}}
\def\Eisenstein{Eisenstein\index{Eisenstein, Ferdinand Gotthold Max}}
\def\Engleman{Engleman\index{Engleman, Carl}}
\def\Erdos{Erd\H{o}s\index{Erdos@Erd\protect\H{o}s, P\'{a}l}}
\def\Euler{Euler\index{Euler, Leonhard}}
\def\EvansR{Evans\index{Evans, R. J.}}
\def\Fagin{Fagin\index{Fagin, Ronald}}
\def\Faltings{Faltings\index{Faltings, Gerd}}
\def\Fateman{Fateman\index{Fateman, Richard J}}
\def\Feldman{Feldman\index{Feldman, Stuart I.}}
\def\Fermat{Fermat\index{Fermat, Pierre de}}
\def\Fibonacci{Fibonacci\index{Fibonacci, Leonardo Pisano}}
\def\Fitch{Fitch\index{Fitch, John}}
\def\Fried{Fried\index{Fried, Michael D.}}
\def\Gallagher{Gallagher\index{Gallagher, Patrick X.}}
\def\Galois{Galois\index{Galois, \'{E}variste}}
\def\Gathen{von zur Gathen\index{von zur Gathen, Joachim}}
\def\Gauss{Gauss\index{Gauss, Karl Friedrich}}
\def\Geddes{Geddes\index{Geddes, Keith O.}}
\def\Gelfond{Gel'fond\index{Gelfond@Gel'fond, Alexsandr Osipovich}}
\def\Genesereth{Genesereth\index{Genesereth, Michael R.}}
\def\Gentleman{Gentleman\index{Gentleman, W. Morven}}
\def\Gianni{Gianni\index{Gianni, Patrizia}}
\def\Golden{Golden\index{Golden, Jeffrey P.}}
\def\Goldwasser{Goldwasser\index{Goldwasser, Shafrira}}
\def\Gonnet{Gonnet\index{Gonnet, Gaston H.}}
\def\Gordan{Gordan\index{Gordan, Paul Albert}}
\def\Gosper{Gosper\index{Gosper, Ralph William, Jr.}}
\def\Griesmer{Griesmer\index{Griesmer, James}}
\def\Grigoriev{Grigor'ev\index{Grigor'ev, Dima Yu.}}
\def\Grothendieck{Grothendieck\index{Grothendieck, Alexandre}}
\def\Habicht{Habicht\index{Habicht, W.}}
\def\Hadamard{Hadamard\index{Hadamard, Jacques Salomon}}
\def\HallA{Hall\index{Hall, Andrew D.}}
\def\Hardy{Hardy\index{Hardy, Godfrey Harold}}
\def\Hearn{Hearn\index{Hearn, Anthony}}
\def\Heilbronn{Heilbronn\index{Heilbronn, H. A.}}
\def\Heintz{Heintz\index{Heintz, Joos}}
\def\Hellman{Hellman\index{Hellman, Martin Edward}}
\def\Hensel{Hensel\index{Hensel, Kurt Wilhelm Sebastian}}
\def\Hermite{Hermite\index{Hermite, Charles}}
\def\Hilbert{Hilbert\index{Hilbert, David}}
\def\Hironaka{Hironaka\index{Hironaka, Heisuke}}
\def\Holdt{van Holdt\index{van Holdt}}
\def\Holder{H\"{o}lder\index{Hoelder@H\protect\"{o}lder, O.}}
%\def\Holder{H\"{o}lder\index{Holder@H\protect\"{o}lder, Ludwig Otto}}
% Jordan-Holder
\def\Hopcroft{Hopcroft\index{Hopcroft, John Edward}}
\def\Hurwitz{Hurwitz\index{Hurwitz, Adolf}}
\def\Ireland{Ireland\index{Ireland, Kenneth}}
\def\Isaacs{Isaacs\index{Isaacs, I. Martin}}
\def\Jenks{Jenks\index{Jenks, Richard D.}}
\def\Jensen{Jensen\index{Jensen, Johan Ludwig William Valdemar}}
\def\JohnsonS{Johnson\index{Johnson, Steven C.}}
\def\JordanC{Jordan\index{Jordan, Camille}}
\def\Kahrimanian{Kahrimanian\index{Kahrimanian}}
\def\Kaltofen{Kaltofen\index{Kaltofen, Erich}}
\def\Kannan{Kannan\index{Kannan, Ravi}}
\def\Karatsuba{Karatsuba\index{Karatsuba, Anatoli\u\i\ Alekseevich}}
\def\Karpinski{Karpinski\index{Karpinski, Marek}}
\def\Kilian{Kilian\index{Kilian, Joseph}}
\def\Klein{Klein\index{Klein, Felix}}
\def\Knobloch{Knobloch\index{Knobloch, Hans--Wilhelm}}
\def\Knuth{Knuth\index{Knuth, Donald Ervin}}
\def\Koblitz{Koblitz\index{Koblitz, Neal}}
\def\Koepf{Koepf\index{Koepf, Wolfram}}
\def\Kronecker{Kronecker\index{Kronecker, Leopold}}
\def\Krull{Krull\index{Krull, Wolfgang}}
\def\Kulp{Kulp\index{Kulp, John L.}}
\def\Kung{Kung\index{Kung, Hsiang Tsung}}
\def\Kummer{Kummer\index{Kummer, Ernst Eduard}}
\def\Lagarias{Lagarias\index{Lagarias, Jeffrey C.}}
\def\Lagrange{Lagrange\index{Lagrange, Joseph Louis}}
\def\Lang{Lang\index{Lang, Serge}}
\def\Lame{Lam\'e\index{Lam\'e, Gabriel}}
\def\LandauE{Landau\index{Landau, Edmund Georg Hermann}}
\def\LandauS{Landau\index{Landau, Susan Eva}}
\def\LehmerD{Lehmer\index{Lehmer, Derrick Norman}}
\def\LehmerE{Lehmer\index{Lehmer, Emma Markovna Trotskaia}}
\def\LenstraA{Lenstra\index{Lenstra, Arjen Klaas}}
\def\LenstraH{Lenstra\index{Lenstra, Hendrik W., Jr.}}
\def\Liouville{Liouville\index{Lioouville, Joseph}}
\def\Lipton{Lipton\index{Lipton, Richard Jay}}
\def\Lovasz{Lov\'asz\index{Lov\'asz, L\'aszl\'o}}
\def\Loos{Loos\index{Loos, R\protect\"{u}diger Georg Konrad}}
\def\Lovelace{Lovelace\index{Lovelace, Ada Augusta, countess of}}
\def\Ma{Ma\index{Ma, Keju}}
\def\Macaulay{Macaulay\index{Macaulay, Francis Sowerby}}
\def\MacDonald{MacDonald\index{MacDonald, I. G.}}
\def\Mack{Mack\index{Mack, Dieter}}
\def\Mahler{Mahler\index{Mahler, Kurt}}
\def\Manove{Manove\index{Manove, Michael}}
\def\MartinW{Martin\index{Martin, William A.}}
\def\Mason{Mason\index{Mason, R. C.}}
\def\Massey{Massey\index{Massey, James L.}}
\def\Mazur{Mazur\index{Mazur, Barry}}
\def\McCarthy{McCarthy\index{McCarthy, John}}
\def\McIlroy{McIlroy\index{McIlroy, M. Douglas}}
\def\Merkle{Merkle\index{Merkle, Ralph}}
\def\Mertens{Mertens\index{Mertens, F.}}
\def\Mignotte{Mignotte\index{Mignotte, Maurice}}
\def\MillerG{Miller\index{Miller, Gary Lee}}
\def\MillerJCP{Miller\index{Miller, Jeffrey Charles Percy}}
\def\MillerV{Miller\index{Miller, Victor Saul}}
\def\Minkowski{Minkowski\index{Minkowski, Hermann}}
\def\Minsky{Minsky\index{Minsky, Marvin}}
\def\MitchellO{Mitchell\index{Mitchell, O. H.}}
\def\Moenck{Moenck\index{Moenck, Robert T.}}
\def\MosesJ{Moses\index{Moses, Joel}}
\def\MuirT{Muir\index{Muir, Thomas}}
\def\Musser{Musser\index{Musser, David Rea}}
\def\Nathanson{Nathanson\index{Nathanson, Melvyn B.}}
\def\Netto{Netto\index{Netto, Eugen}}
\def\Newton{Newton\index{Newton, Isaac, Sir}}
\def\NoetherE{Noether\index{Noether, Emma}}
\def\NoetherM{Noether\index{Noether, Max}}
\def\Norman{Norman\index{Norman, Arthur C.}}
\def\Odlyzko{Odlyzko\index{Odlyzko, Andrew Michael}}
\def\Ofman{Ofman\index{Ofman, Y}}
\def\Olds{Olds\index{Olds, Carl Douglas}}
\def\Ore{Ore\index{Ore, \protect\"{O}ystein}}
\def\Pan{Pan\index{Pan, Viktor \t{Ia}kovlevich}}
\def\Perron{Perron\index{Perron, Oskar}}
\def\Piras{Piras\index{Piras, F.}}
\def\Pollard{Pollard\index{Pollard, John Michael}}
\def\Polya{P\'olya\index{Polya@P\'olya, George}}
\def\Pratt{Pratt\index{Pratt, Vaugh Ronald}}
\def\Probst{Probst\index{Probst, David K.}}
\def\Rabin{Rabin\index{Rabin, Michael Oser}}
\def\Ramanujan{R\={a}m\={a}nujuan\index{Ramanujan@R\protect\={a}m\protect\={a}nujuan, 
   Sr\protect\={\i}niv\protect\={a}sa Aiya\protect\.{n}g\protect\={a}r}}
\def\Redei{R\'{e}dei\index{Redei@R\'{e}dei, Alfred}}
\def\Riemann{Riemann\index{Riemann, Georg Fiedrich Bernhard}}
\def\Renyi{R\'enyi\index{R\'enyi, Alfr\'ed}}
\def\Richards{Richards\index{Richards, Ian}}
\def\Risch{Risch\index{Risch, Robert H.}}
\def\Risler{Risler\index{Risler, Jean-Jacques}}
\def\Ritchie{Ritchie\index{Ritchie, Dennis M.}}
\def\Ritt{Ritt\index{Ritt, Joseph Fels}}
\def\Rivest{Rivest\index{Rivest, Ronald Linn}}
\def\Ronga{Ronga\index{Ronga, Felice}}
\def\Rosen{Rosen\index{Rosen, Michael}}
\def\Rothschild{Rothschild\index{Rothschild, Linda Preis}}
\def\Rump{Rump\index{Rump, Siegfried M.}}
\def\Runge{Runge\index{Runge, Carl David Tolme}}
\def\SalvyB{Salvy\index{Salvy, Bruno}}
\def\SchatzS{Schatz\index{Schatz, Steven}}
\def\Schinzel{Schinzel\index{Schinzel, Andrej}}
\def\Schnorr{Schnorr\index{Schnorr, Claus-Peter}}
\def\Schoenhage{Sch\"{o}nhage\index{Schoenhage@Sch\protect\"{o}nhage, Arnold}}
\def\Schoof{Schoof\index{Schoof, Ren\'ee}}
\def\SchwartzJ{Schwartz\index{Schwartz, Jacob Theodore}}
\def\SchwartzL{Schwartz\index{Schwartz, Laurent}}
\def\Seidenberg{Seidenberg\index{Seidenberg, Abraham}}
\def\Serre{Serre\index{Serre, Jean Pierre}}
\def\Shackell{Shackell\index{Shakell, John R.}}
\def\Shallit{Shallit\index{Shallit, Jeff}}
\def\Shamir{Shamir\index{Shamir, Adi}}
\def\Silverman{Silverman\index{Silverman, Joseph H.}}
\def\Singer{Singer\index{Singer, Michael F.}}
\def\Slagle{Slagle\index{Slagle, James R.}}
\def\Solovay{Solovay\index{Solovay, Robert Martin}}
\def\Stark{Stark\index{Stark, Harold M.}}
\def\Sterling{Sterling\index{Sterling, James}}
\def\Swan{Swan\index{Swan, R. G.}}
\def\SwinnertonDyer{Swinnerton-Dyer\index{Swinnerton-Dyer, Henry Peter Francis}}
\def\Sylvester{Sylvester\index{Sylvester, James Joseph}}
\def\Szego{Szeg\"o\index{Szego@Szeg\protect\"o, Gabor}}
\def\Tarjan{Tarjan\index{Tarjan, Robert Endre}}
\def\Tarski{Tarski\index{Tarski, Alfred}}
\def\Tate{Tate\index{Tate, John T.}}
\def\Tiwari{Tiwari\index{Tiwari, Prasoon}}
\def\Tompa{Tompa\index{Tompa, Martin}}
\def\TrabbPardo{Trabb Pardo\index{Trabb Pardo, Luis Isidoro}}
\def\Trager{Trager\index{Trager, Barry Marshall}}
\def\Traub{Traub\index{Traub, Joseph Fredrick}}
\def\Tukey{Tukey\index{Tukey, John Wilder}}
\def\Turing{Turing\index{Turing, Alan Mathison}}
\def\Hulzen{van Hulzen\index{van Hulzen, J. A.}}
\def\Ullman{Ullman\index{Ullman, Jeffrey David}}
\def\ValleePoussin{de la Vall\'ee-Poussin\index{Vallee-Poussin@de la
   Vall\protect\'{e}e-Poussin, Charles Jean Gustave Nicolas}}
\def\Vaughn{Vaughn\index{Vaughn, R. C.}}
\def\Waerden{van der Waerden\index{van der Waerden, Bartel Leendert}}
\def\Waldschmidt{Waldschmidt\index{Waldschmidt, Michel}}
\def\WangP{Wang\index{Wang, Paul Shyh-Horng}}
\def\Weierstrass{Weierstrass\index{Weierstrass, Karl}}
\def\Weinberger{Weinberger\index{Weinberger, Peter J.}}
\def\Weil{Weil\index{Weil, Andr\'e}}
\def\Weyl{Weyl\index{Weyl, Hermann Klaus Hugo}}
\def\WilliamsH{Williams\index{Williams, Hugh Cowie}}
\def\Wolfram{Wolfram\index{Wolfram, Steven}}
\def\Wright{Wright\index{Wright, Edward Maitland}}
\def\Yao{Yao\index{Yao, Andrew Chi-Chih}}
\def\Yun{Yun\index{Yun, David Yuan-Yee}}
\def\Zassenhaus{Zassenhaus\index{Zassenhaus, Hans Julius}}
\def\Zippel{Zippel\index{Zippel, Richard Eliot}}




\documentstyle[12pt]{rzbook}
\input{epsf}
\epsfverbosetrue

\begin {document}

Note that we have used the same notation here as for the $p$-adic
valuations discussed in, but this should not cause a problem number
fields.

\begin{figure}
\begin {center}
  \epsfpict{Pix/x3-xa.eps}{
   \put(-90,150){\line(4,1){260}}
   \put(83,230){\line(0,-1){220}}
   \put(-84,160){$P_1$}
   \put(-71,155){\circle*{4}}
   \put(-20,180){$P_2$}
   \put(-13,169){\circle*{4}}
   \put(58,200){$-P_3$}
   \put(83,193){\circle*{4}}
   \put(89,92){$P_3 = P_1 + P_2$}
   \put(83,90){\circle*{4}}
} 
\end{center}
\end{figure}

The volume of the fundamental domain is an important property of a
lattice.  As we shall show, it does not depend on the basis but is an
invariant of the lattice.  If the basis vectors are orthogonal, then
the volume of fundamental domain is just the product of the lengths of
the vectors.  Recall that if two vectors $\vec{a}$ and $\vec{b}$ are
orthogonal then the dot product vanishes

\end{document}

