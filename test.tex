\input{Macros}

\documentstyle[12pt]{rzbook}
\input{epsf}
\epsfverbosetrue

\begin {document}

Note that we have used the same notation here as for the $p$-adic
valuations discussed in, but this should not cause a problem number
fields.

\begin{figure}
\begin {center}
  \epsfpict{Pix/x3-xa.eps}{
   \put(-90,150){\line(4,1){260}}
   \put(83,230){\line(0,-1){220}}
   \put(-84,160){$P_1$}
   \put(-71,155){\circle*{4}}
   \put(-20,180){$P_2$}
   \put(-13,169){\circle*{4}}
   \put(58,200){$-P_3$}
   \put(83,193){\circle*{4}}
   \put(89,92){$P_3 = P_1 + P_2$}
   \put(83,90){\circle*{4}}
} 
\end{center}
\end{figure}

The volume of the fundamental domain is an important property of a
lattice.  As we shall show, it does not depend on the basis but is an
invariant of the lattice.  If the basis vectors are orthogonal, then
the volume of fundamental domain is just the product of the lengths of
the vectors.  Recall that if two vectors $\vec{a}$ and $\vec{b}$ are
orthogonal then the dot product vanishes

\end{document}

