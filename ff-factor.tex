%$Id: ff-factor.tex,v 1.1 1992/05/10 19:40:35 rz Exp rz $

\chapter{Factoring over Finite Fields}
\label{Poly:FF:Factor:Chap}

Factoring polynomials is of the most challenging problems in algebraic
computation.  The complete algorithm for factoring multivariate
polynomials over the rational integers will use nearly all of the
techniques we have developed thus far.  In this chapter we consider
the comparatively simpler problem of factoring polynomials over the
finite fields.  This problem arises in coding theory and is a step in
factoring polynomials over the rational integers.

Let $F(X)$ be a univariate polynomial with coefficients in $\F_q$,
where $q=p^r$.  As it turns out there are two simple steps we can take
that simplify the factorization problem.  In \sectref{FFac:Sqfr:Sec}
we perform a \keyi{square free factorization} of $F$, \ie, factor $F$
so that
\[
F(X) = F_1(X) F_2(X)^2 F_3(X)^3 \cdots F_k(X)^k,
\]
where the $F_i$ are pairwise relatively prime and $F_i$ and $F_i'$ are
relatively prime.  Square free decomposition of polynomials is a
simple sequence of GCD computations and for univariate polynomials
direct computation is almost always efficient

Distinct degree factorization of the $F_i$, \ie,
\[
F_i(X) = F_{i1}(X) F_{i2}(X) F_{i3}(X) \cdots F_{i\ell}(X),
\]
where the $F_{ij}(X)$ are products of irreducible polynomials of
degree $j$.  Again this is a relatively simple process for univariate
polynomials over finite fields.


The most expensive part of univariate factorization over finite fields
is factoring the $F_{ij}$.  In the case of factoring $F_{i1}$ this is
just finding zeroes in $\F_q$.  The determining the factors of
$F_{ij}$ corresponds to finding zeroes in $\F_{q^j}$.  The techniques
for doing this are discussed in \sectref{}.

\section{Square Free Decomposition}
\label{FFac:Sqfr:Sec}

The square free decomposition algorithms we are about to describe are
valid for polynomials over an integral domain.  For simplicity
however, we will describe the technique for polynomials over a field
$k$, even though the only field for we will use these algorithms is
$\F_q$.  \sectref{GFac:Sqfr:Sec} describes how to combine the
techniques of this section with the lifting techniques of
\chapref{Sparse:Hensel:Chap} to compute square free decompositions of
multivariate polynomials and polynomials over the rational integers.

Let $F(X)$ be a monic polynomial over the field $k$,
\[
F(X) = X^n + f_1 X^{n-1} + \cdots + f_n.
\]
The \keyi{square
free decomposition} of $F(X)$ is a set of polynomials $F_1, \ldots,
F_k$ such that
\begin{equation}\label{FFac:Sqfr:Eq}
F(X) = F_1(X) \cdot F_2(X)^2 \cdot F_3(X)^3 \cdots F_k(X)^k,
\end{equation}
the $F_i$ are relatively prime and for each $i$, $F_i$ and $F_i'$ are
relatively prime.  

The derivative of $F(X)$ is
\[
F'(X) = nX^{n-1} + (n-1) f_1 X^{n-2} + \cdots + f_{n-1}.
\]
If $F(X)$ is a constant, its derivative will be zero.  If the
characteristic of $k$ is positive, then $F'$ will also be zero if all
of the exponents are multiples of $p$, \ie, if $F(X) = \hat{F}(X^p)$.
For instance, for $k= \F_7$
\[
A(X) = X^{49} + 3 X^{14} + 3
\]
has derivative zero.  However, if the characteristic of  $k$ is $\F_{p^r}$
then then $A(X) = \hat{A}(X^p) = (\hat{F}(X))^p$, where $\hat{A}(X) =
X^7 +3X^2 + 3$.  This
may not for more general fields.  For instance, if $B(X) = X^7 +Y$,
where $k = \F_7(Y)$, then $B'(X) = 0$, but
\[
X^7 + Y = (X+Y^{1/7})^7.
\]
So algebraic extensions are needed in this case.

The condition that $F_i$ and $F_i'$ be relatively prime is equivalent
to saying that the $F_i$ have no multiple zeroes.  For instance, let
$G(X) = (X - \alpha)^r H(X)$, where $(x - \alpha)$ does not divide
$H(X)$ and $r \ge 0$.  We say that $\alpha$ is a {\em zero of
multiplicity}\index{multiplicity!of polynomial zeroes} $r$ in this
case. 

 The multiplicity of a polynomial factor\index{multiplicity!of
polynomial factor} is defined is precisely the same fashion.  So each
of the factors of $F_i$ in \eqnref{FFac:Sqfr:Eq} have multiplicity $i$
in $F(X)$.

Assume that $r$ is positive and assume the derivative of $G$ is not
zero:
\[
\begin{eqalign}
  G'(X) &= r (X - \alpha)^{r-1} H(X) + (X - \alpha)^r H'(X), \\
  & = (X - \alpha)^{r-1}\left[r H(X) + (X - \alpha) H'(X)\right].
\end{eqalign}
\]
Since $(X - \alpha)$ does not divide $H(X)$, $\alpha$ has multiplicity
precisely $r-1$ in $G'(X)$.  If the multiplicity of $\alpha$ in $G$ is
greater than one then $\gcd(G, G') \not = 1$.  So $(G, G') = 1$
implies that $G(X)$ has no multiple zeroes. Notice that the
requirement that $G'(X) \not= 0$ is also covered by $(G, G') = 1$.

Following proposition generalizes this result to polynomial factos.
Its proof is the same, but requires that each irreducible factor
possess a zero.  This is true in the algebraic closure of $k$.

\begin{proposition}
Let $G(x)$ be an irreducible factor of $F(X)$, a polynomial over a
field.  If the GCD of $F(X)$ and $F'(X)$ is $1$ then the multiplicity
of $G(X)$ in $F(X)$ is exactly $1$.
\end{proposition}

\medskip
With these ideas, we can construct an algorithm for computing the
square free decomposition $F(X)$.  By the previous reasoning $F_i(X)$
has multiplicity precisely $i-1$ in $F'(X)$, so
\[
(F(X), F'(X)) = F_2 F_3^2 \cdots F_k^{k-1} = G(X).
\]
Repeating this process with $G(X)$ gives
\[
(G(X), G'(X)) = F_3 F_4^2 \cdots F_k^{k-2} = H(X).
\]
Therefore,
\[
\begin{eqalign}
\frac{F}{G} & = F_1 F_2 \cdots F_k, \\
\frac{G}{H} & = F_2 \cdots F_k,
\end{eqalign}
\]
and thus
\[
\frac{F \cdot H}{G \cdot G} = F_1.
\]
Repeating this process, we can compute the the $F_i$ in sequence.

The following routine implements this approach.  Note that we have
assumed that $F'$ is not zero.

\begindsacode
PolySQFR ($F(X)$) := $\{$ \\
\> $G \leftarrow \mbox{PolYGCD}(F, \mbox{PolyDeriv}(F, X))$;\\
\> $H \leftarrow \mbox{PolYGCD}(G, \mbox{PolyDeriv}(G, X))$;\\
\> $i \leftarrow 1$;\\
\> loo\=p $\{$\\
\> \> $F_i \leftarrow (F/G)/(G/H)$; \\
\> \> $(F, G, H) \leftarrow \mbox{PolyGCD}(G, H, (H, \mbox{PolyDeriv}(H, X)))$;\\
\>\> $i \leftarrow i + 1$; \\
\>\> $\}$\\
\> $\}$
\enddsacode

\section{Distinct Degree Factorization}
\label{FFac:Distinct:Sec}

While square free decomposition algorithm applies to polynomials over
any ring, distinct degree factorization is an operation that appears
practical only for univariate polynomials over finite fields.  At this
point, we assume that $F(X)$ is a square free monic polynomial over
$\F_q$ of degree $n$, where $q= p^r$.  A \keyi{distinct degree
factorization} of $F(X)$, is a set of polynomials $F_1, \ldots,
F_{\ell}$ such that
\[
F(X) = F_1(x) \cdot F_2(X) \cdots F_k(X)
\]
and $F_j$ is the product of irreducible polynomials of degree $j$.

By Fermat's little theorem, each element of $\F_q$ is a zero of $X^q-
X$, \ie, 
\[
\prod_{\alpha \in \F_q} (X - \alpha) = X^q - X.
\]
So $X^q- X$ is the product of all the monic linear polynomial over
$\F_q$.  Consequently, $F_1$ is the GCD of $F(X)$ and $X^q - X$.

Similarly, the product of all monic polynomials of degree less than
$\ell$ is $X^{q^\ell} -X$ and we have
\[
(F(X), X^{q^{\ell}} - X) = F_1(X) \cdots F_{\ell}(X).
\]

Distinct degree factorization is thus just a matter of computing a few
GCD's.  However, a trick is necessary.  For large values of $q$ it is
the standard remainder algorithm for the first step of the Euclidean
algorithm with $X^{q^{\ell}}-X$ by $F(X)$ will take $O(q^{\ell})$
steps, which is too costly.  Instead, it is preferable to compute
$X^{q^{\ell}}$ by repeated squaring modulo $F(X)$.  The following
routine implements this.\index{polynomial exponentiation}

\begindsacode
PolyExptMod (base, $M$, $F$) := $\{$ \\
\> $\mbox{prod} \leftarrow 1$; \\
\> $\mbox{base} \leftarrow \mbox{PolyRem}(\mbox{base}, F)$; \\
\> loo\=p $\{$\\
\>\> if $\mbox{oddp}(M)$ then $\mbox{prod} \leftarrow \mbox{PolyRem}(\mbox{prod} \times \mbox{base}, F)$;\\
\>\> $M \leftarrow \lfloor M/2 \rfloor$; \\
\>\> if $M = 0$ then $\mbox{return}(\mbox{prod})$;\\
\>\> else $\mbox{base} \leftarrow \mbox{PolyRem}(\mbox{base}
  \times\mbox{base}, F)$; \\
\>\> $\}$\\
\>$\}$
\enddsacode

\keyw{PolyExptMod} computes $\mbox{\tt base}^M \mod{F}$.  The loop
inside {\tt PolyExptMod} is executed $O(\log M)$ times.  During each
loop at most two polynomial multiples are performed, and one
polynomial remainder.  Each of the polynomials involved is of degree
no larger than $\deg F = n$.  Thus the number of coefficient
operations performed is $O(n\log M)$.

The GCD of $F(X)$ and $X^{q^\ell} -X$ is computed as follows
\[
\mbox{\tt PolyGCD}(F(X), \mbox{\tt PolyExptMod}(X, q^{\ell}, F(X)) - X)
\]
That is, the first remainder is computed using repeated squaring, and
then the rest of the computation is done using any of the usual
Euclidean algorithms.  

Combining these ideas we have the following routine for computing the
distinct degree factorization of $F$.
\begindsacode
PolyDistinctDegree ($F$) := $\{$ \\
\> declare $F \in \F_q[X]$; \\
\> $i = 1$; \\
\> loo\=p while $F \not= 1$ $\{$ \\
\>\> if \=$i | \deg F$ then $\{$ \\
\>\>\> $F_i \leftarrow
      \mbox{PolyGCD}(F(X), \mbox{PolyExptMod}(X, q^i, F(X)) - X)$;\\
\>\>\> $F \leftarrow F/F_i$ ; \\
\>\>\> $\}$ \\
\>\> $i \leftarrow i + 1$;\\
\>\> $\}$ \\
\> $\}$
\enddsacode

\noindent
Notice that each time we determine an $F_i$ we divide it out of $F$.
This makes the succeeding GCD's a bit smaller and simplifies dealing
with factors of degree less than $i$.  

Let $n$ denote the degree of $F$ in $X$.  Asymptotically, the
computation of each $F_i$ takes time $O(n \log q + n^2)$.  The loop in
\keyw{PolyDistinctDegree} is repeated at most $d(n)$ time, where
$d(n)$ is the number of divisors of $n$.  Since $d(n) = O(n^{1/2})$ by
\longpropref{NT:Divisor:Est:Prop} the total time taken is bounded by
$O(n^{3/2}(\log q) + n^{5/2})$.


\section{Finding Linear Factors}
\label{FFac:Linear:Sec}

Using the techniques of the last two sections, we are left with the
problem of factoring a univariate polynomials all of whose factors are
of the same degree.  Splitting polynomials of this type into
irreducible components essentially reduces to zero finding in $\F_q$
or some extension of $\F_q$.  This section illustrates the basic ideas
involved, and their analysis, by focusing on the case where $F(X)$ is
the product of linear factors.  Next section will consider this
problem in more generality.

Thus, in this section we assume that $F(X)$ is a monic, square free
polynomial over $\F_q$ that is the product of linear factors.
Further, assume that $q$ is odd and denote $n$ the degree of $F(X)$.
Since $F(X)$ is both square free an the product of linear factors, it
must divide $X^q - X$.  This polynomial is easily factored:
\[
X^q - X = X\cdot (X^{(q-1)/2} - 1)\cdot (X^{(q-1)/2} + 1) 
  = X \cdot R(X) \cdot N(X)
\]
If the zeroes of $F(X)$ are uniformly distributed then about half of
them will be zeroes of $N(X)$ and half of $R(X)$, \ie, half will be
quadratic residues and half will not be.  Denote the GCD of $F(X)$ and
$X^{(p-1)/2} - 1$ by $F_1(X)$.  The expected degree of $F_1(X)$ is
$n/2$.  Notice that by using \keyw{PolyExptMod} we can do this in $O(n
\log q)$ time.  So we have reduced the problem of factoring a
polynomial of degree $n$ to that of factoring one of degree $n/2$.  If
we can continue this splitting process with $F_1(X)$ and
$F(X)/F_1(X)$, then after about $n$ GCD computation we should be able
to split $F(X)$ into linear factors.  This would lead to an $O(n^2
\log q)$ factoring algorithm. 

The technique of the previous paragraph split the zeroes of $F(X)$ up
into classes depending upon whether they were quadratic residues or
non-residues.  We can scramble the quadratic character of the zeroes
by adding an arbitrary constant to each zero.  That is, by working
with $F_1(X-b)$ instead $F(X)$.  Intuitively we are using the
following proposition, which is proven more precisely as
\longpropref{QRDistribution:Prop}. 

\begin{proposition}
Let $\{ a_i \}$ be a set of quadratic residues modulo $p$.  The
expected portion of quadratic residues in $\{ a_i + b\}$ is $1/2$.
\end{proposition}

Thus we can split $F_1(X)$ further by computing the GCD of $F_1(X- b)$
and $X^{(q-1)/2} - 1$.  Further splits can be obtained by choosing
other values for $b$.  It turns out that is is slightly more efficient
to take the GCD of $F_1(X)$ and $(X+b)^{(q-1)/2} - 1$ so that we can
avoid the cost of shifting the $F_1(X)$ (we always need to raise $X$
to the $(q-1)/2$ power).

\begindsacode
PolyLinearFactors ($F$) := $\{$ \\
\> declare $F \in \F_q[X]$; \\
\> if $\deg F = 1$ then $\{ F \}$; \\
\> els\=e $\{$\\
\> \> $F_1 \leftarrow 
   \mbox{PolyGCD}(F, \mbox{PolyExptMod}(X+\mbox{random}(\F_q), (q-1)/2, F) -1)$;\\
\>\> if $\deg F_1 = \deg F$ or $0 = \deg F_1$ then $\mbox{PolyLinearFactors}(F)$;\\
\>\> else $\mbox{PolyLinearFactors}(F_1) \cup \mbox{PolyLinearFactors}(F/F_1)$;\\
\>\> $\}$\\
\> $\}$
\enddsacode

\noindent
The routine \keyw{random} chooses a uniformly distributed random
element of from its argument, which is assumed to be a finite set.

This algorithm will require $O(n)$ GCD computations to separate the roots of
$F(X)$.  Each one will require one call to \keyw{PolyExptMod}, which
requires $O(n^2 \log p)$ coefficient operations, a small polynomial GCD of
about $O(n^3)$ operations and a division of no more than $O(n^2)$ operations.
Thus each separation requires about $O(n^2 \log p)$, and the whole
process will require $O(n^3 \log p)$ operations to compute the
linear factors of a polynomial.

\section{The Cantor-Zassenhaus algorithm}

\section{Distribution of Quadratic Residues}
\label{Distribution:QR:Sec}

Let $G$ be an multiplicative abelian group.  For instance, $\F_p^{\ast}$,
the quadratic residues of $\F_p$, etc.  A map $\chi: G \rightarrow \C$ is a
called a \keyi{character} if the absolute value of $\chi(g)$ for all
elements of $G$ is $1$ and if $\chi(ab) = \chi(a) \chi(b)$ for all $a, b
\in G$.  If $\chi_1$ and $\chi_2$ are two characters, then $\chi_3$ is also
a character, where $\chi_3(a) = \chi_1(x) \chi_2(x)$.  The character
$\chi_0$, where $\chi_0(a) = 1$ is the identity element for character
multiplication.  Furthermore, each character $\chi$ has an inverse, \eg{} the 
character $\overline{\chi}$ defined by $\overline{\chi(a)} =
\overline{\chi(a)}$ (complex conjugate).  Thus the characters of $G$ form a
group.  A character $\chi$ has {\em degree}\index{degree, of a character}
$d$ if $\chi^d = \chi_0$.

The proof of the following theorem and related results are in
\cite{Schmidt:Equations:FF}.

\begin{proposition}
\label{CharSum:Estimate:Prop}
Let $f(x) \in \F_q[x]$ have $m$ distinct zeroes and assume that $y^d -
f(x)$ is absolutely irreducible.  If $\chi \not= \chi_0$ is a
multiplicative character of degree $1$ then
\[
\left|\sum_{x \in \F_q} \chi(f(x))\right| \le (m-1) q^{1/2}.
\]
\end{proposition}

The following proposition is due to {\BenOr} \cite{BenOr81}.

As usual let $\F_q$ be a finite field of characteristic $p$.  Let
$\chi$ be the quadratic residuacity character:
\[
\chi(a) = 
  \begin{cases}
    0, & if $a = 0$\\
    1,& if $a$ is a quadratic residue\\
    -1,& otherwise
  \end{cases}
\]
 
\begin{proposition}
\label{QRDistribution:Prop} Let $a_1, \ldots, a_k$ be distinct
elements of $\F_q$, and let $\epsilon_\ell = \pm1$ for $\ell = 1,
\dots, k$.  The set $N$ is defined to be:
\[
N = \{\, \delta \in \F_q \mid \chi(a_\ell + \delta) = \epsilon_\ell, \ell =
1, \ldots, k\,\}.
\]
Then the cardinality of $N$ satisfies
\[
\left|\#N - {q \over 2^k}\right| < {k \over 2}(1 + \sqrt{q}).
\]
\end{proposition}

\begin{proof}
The proof of this proposition proceeds by translating the statement
$\chi(a_{\ell} + \delta) = \epsilon_{\ell}$ into a statement about the
number of solutions of a system of $k$ equations in $k+1$ unknowns
over $\F_q$.  We can then estimate $\#N$ by counting the solutions of
the equations.

First notice that if neither $k_1$ nor $k_2$ is a quadratic residue,
then $k_1 k_2$ is a quadratic residue.  Choose $d$ to be a quadratic
non-residue, and let
\[
d_i = 
  \begin{cases}
    1, & if $\epsilon_i = 1$\\
    d & if $\epsilon_i = -1$
  \end{cases}
\]
Now construct the system of equations
\begin{equation}\label{FFac:BenOr:Eqs}
\begin{eqalign}
  y_1^2 & = d_1 (x+a_1) \\
    & \vdots\\
  y_k^2 & = d_k (x+a_k)
\end{eqalign}
\end{equation}
If $\chi(a_i + \delta) \not= \epsilon_i$ then $y_i^2 = d_i(\delta +
a_i)$ will not have any solutions.  If $\chi(a_i + \delta) =
\epsilon_i$ then $y_i^2 = d_i(\delta + a_i)$ will have either $1$
solution if $\delta+a_i =0$ or two solutions otherwise.  Let $M$ be
the number of solutions of \eqnref{FFac:BenOr:Eqs}, and $M_0$ the
number of solutions where any of the $y_i$ are zero.  Then
\[
M - M_0 = 2^k \# N.
\]

We can count $M_0$ easily.  There are at most $k$ different values of
$x$ for which \eqnref{FFac:BenOr:Eqs} has a solution with one of the
$y_i = 0$.  For each of these, the each of the other $y_j$ can take on
two different values.  Thus
$M_0 \le k 2^{k-1}$, and combining with the expression for $M$ we have
\begin{equation}\label{FFac:BenOr:Eqsa}
M =2^k\left(\#N + \frac{k}{2}\right).
\end{equation}

Now we count the total number of solutions to \eqnref{FFac:BenOr:Eqs},
$M$.  By \longpropref{NumQuadSol:Prop}, the number of solutions of
$y_\ell^2 = d_\ell (x+a_\ell)$ is $1 + \chi(d_\ell x + d_\ell
a_\ell)$.  So,
\[
\begin{eqalign}
  M &= \sum_{x \in \F_q} \left[ 
      \prod_{\ell = 1}^k \left(1 + \chi(d_\ell x + d_\ell a_\ell)\right)
      \right] 
     = \sum_{x \in \F_q} \sum_{0 \le i_1, \ldots, i_k \le 1}
      \prod_{\ell = 1}^k \chi(d_\ell x + d_\ell a_\ell)^{i_\ell} \\
    & =  \sum_{0 \le i_1, \ldots, i_k \le 1} \sum_{x \in \F_q} 
      \chi\left(\prod_{\ell = 1}^k (d_\ell x + d_\ell
      a_\ell)^{i_\ell}\right)
\end{eqalign}
\]

For $i_1 = \cdots = i_\ell = 0$ the argument to the $\chi$ is $1$, and
thus the sum over the elements of $\F_q$ will be $q$.  In the other
cases, we have a sum like
\[
\sum_{x \in \F_q} \chi(g(x)),
\]
where $g(x)$ is square free polynomial in $x$ of degree at most
$\ell$.  By \propref{CharSum:Estimate:Prop}, we have
\[
\left|\sum_{x \in \F_q} \chi(g(x))\right| < (\ell -1 ) \sqrt{q}.
\]
So,
\[
|M - q| < \sum_{1 \le \ell \le k} {k \choose \ell} (\ell -1) \sqrt{q} <
k 2^{k-1} \sqrt{q}.
\]
Combining this with \eqnref{FFac:BenOr:Eqsa} gives
\[
\left| \#N + \frac{k}{2} - \frac{q}{2^k}\right| < \frac{k}{2}\sqrt{q},
\]
or
\[
\left|\#N - {q \over 2^k}\right| < {k \over 2} (1 + \sqrt{q}).
\] 
\end{proof}


\section*{Notes}

Additional papers: Moenck \cite{Moenck77}.
