%$Id: finite-fields.tex,v 1.2 1992/02/24 01:01:16 rz Exp rz $
\chapter{Residue Rings}
\label{Finite:Fields:Chap}

This chapter presents some background material on computation in the
ring $\Z/m\Z$.  The basic properties of the integers modulo an
arbitrary integer are discussed in \sectref{FF:Basic:Sec}.  These
objects are rings and not necessarily fields.  One of the most
important tools in symbolic computation, the Chinese remainder
theorem, is discussed in \sectref{Integer:Chinese:Remainder:Sec}.  The
set of elements of $\Z/m\Z$ that have an inverse form a multiplicative
group.  The structure of this group is discussed in
\sectref{FF:Multiplicative:Sec}.  \sectref{Quadratic:Reciprocity:Sec}
contains one of the most elegant theorems in number theory, {\Gauss}'s
law of quadratic reciprocity. \sectref{FF:AlgExt:Sec} discusses
algebraic extensions of finite fields, which are themselves fields.
In \sectref{padic:Arith:Sec} we show how the sequence of rings
$\Z/p\Z, \Z/p^2\Z, \ldots$ can be used to recover $\Z$ from its images
in the residue rings.  In addition to recovering $\Z$ we also get many
other elements.  This is the {\em completion} of $\Z$ at $(p)$ and its
elements are called $p$-adic numbers.

The last two sections of this chapter contain interesting applications
of residue ring arithmetic.  \sectref{Crypto:Sec} shows how the
properties of $\Z/m\Z$ can be used to develop cryptographic encryption
schemes.  One scheme is developed that is not secure, but the method
used to crack it provides an interesting technique for reconstructing
two integers from their quotient modulo a prime number.  The other
scheme is probably the hardest to crack cryptosystem known, although
encryption is quite costly.  Finally, in \sectref{FF:SumSquares:Sec}
we combine the Chinese remainder theorem, results on quadratic
residues and Minkowski's convex body theorem to show that all integers
are the sum of four squares.

\section{Basic Properties of ${\protect\Z}/m{\protect\Z}$}
\label{FF:Basic:Sec}

The rational integers can be separated into equivalence classes based
on their remainders when divided by some integer $m$.  Two integers
are in the same equivalence class if their remainders are the same; or
equivalently, if their difference is divisible by $m$.  The elements
of an equivalence class are of the form $a_0 + m t$, where $t$ is a
rational integer and $a_0$ is some element of the class.  We indicate
that $a$ and $b$ satisfy this equivalence relation by $a = b\bmod{m}$.
The set of such equivalence classes is denoted by $\Z/m\Z$.
\addsymbol{$a= b\mod{m}$}{$a-b$ is divisible by $m$}
\addsymbol{$\Z/m\Z$}{The ring of integers modulo $m$}

We can define the sum of two equivalence classes by the sum of their
elements; the sum of the two classes $a_0 + mr$ and $b_0 + ms$ is $a_0
+ b_0 + mt$.  Subtraction and multiplication are defined similarly.
From here on we only use a representative of the equivalence class.
We have not lost any information by doing this since the equivalence
class can always be reconstructed from a representative.  Because of
the way the arithmetic operations on classes have been defined, the
sum or product of two classes can be determined by computing the sum
or product of the representatives (and removing possibly extra
multiples of $m$).  The set of representatives of the equivalence
classes is called a \keyi{presentation} of $\Z/m\Z$.

For example, consider the integers modulo $5$, $\Z/5\Z$.  There are 5
equivalence classes:
\[
\begin{array}{c}
\{\,\ldots, -10, -5, 0, 5, 10, 15, \ldots\,\} \\
\{\,\ldots, -9, -4, 1, 6, 11, 16, \ldots\,\} \\
\{\,\ldots, -8, -3, 2, 7, 12, 17, \ldots\,\} \\
\{\,\ldots, -7, -2, 3, 8, 13, 18, \ldots\,\} \\
\{\,\ldots, -6, -1, 4, 9, 14, 19, \ldots\,\}
\end{array}
\]
One common set of canonical representatives is the integers between
$0$ and $4$.  The product of $3$ and $4$ (\ie, their corresponding
equivalence classes) is $12$ whose canonical representative is $2$.
This set of canonical representatives is called the {\em one sided
presentation}.\index{presentation!one sided} One feature of using this
presentation is that arithmetic is quite efficient on current
computers.  To reduce any positive representative to a canonical
representative one only needs to compute the remainder when divided by
$5$ ($m$).

Another common presentation uses the integers with smallest absolute
value, \ie, for $m = 5$ we would use $\{\,-2, -1, 0, 1, 2\,\}$.
This set of representatives is called the \key{balanced
presentation} of the integers modulo
$m$.  The comparisons required by the reduction process for the
balanced presentation are somewhat more complex in this case, and the
computations can be slightly more expensive than with the one sided
presentation.  Thus arithmetic is usually performed with a one sided
presentation and converted to the balanced presentation when needed.

The integers modulo $m$ have a few characteristics that differ from
the ring of rational integers.  For instance, if $a$ and $b$ are elements
of $\Z$ and their product is $0$, then one of them must be zero.
However, $2 \cdot 3 = 0 \mod{6}$.  More generally, if $a\cdot b = 0
\mod{m}$ then $a$ and $b$ are said to be {\em zero
divisors}\index{zero divisor} modulo
$m$.  In this case, $a \cdot b$ must be a multiple of $m$ and thus
each must have a factor in common with $m$.

Another manifestation of this issue arises in division.  Over $\Z$, if
$ax = ay$ and $a$ is not zero, then $x = y$.  For $\Z/m\Z$ the
condition ``$a$ is not zero'' must be replaced with ``$a$ is not a
zero divisor,'' since $ax = ay \mod{m}$ implies $a (x -y) =
0 \mod{m}$.  For instance, 
\[
3 \cdot 1 = 3 \cdot 5 \pmod{6}.
\]

The only elements of $\Z$ that have multiplicative inverses in $\Z$
are $\pm 1$.  They are called the {\em units\/} of $\Z$.\index{unit,
of a ring} In $\Z/m\Z$ there are many more units.  Let $a \in \Z/m\Z$
and let $x$ be its inverse, which is to be determined.  Then $ax - 1$
is a multiple of $m$ or
\[
ax - m y = 1.
\]
By the discussion in \sectref{Linear:Dio:Sec} this equation has a
solution if and only if $a$ and $m$ are relatively prime.  The next to
last continued fraction convergent to $m/a$ is $x/y$, so the inverse
can be easily computed if it exists.

Conversely, assume $(a, m) = d$.  Then
\[
a \cdot \frac{m}{d} = 0 \pmod{m},
\]
so $a$ is a zero divisor if it does not have an inverse.  The non-zero
elements of $\Z/m\Z$ are either units or zero divisors.

If $p$ is a prime then $\Z/p\Z$ has no zero divisors and thus is a
field.  We denote this field by $\F_p$.  If $m$ is not a prime, then
the units of $\Z/m\Z$ form a group which is denoted by $U(\Z/m\Z)$.
\addsymbol{$\F_p$}{Finite field with $p$ elements}
\addsymbol{$U(\Z/m\Z)$}{The units of the ring of integers modulo $m$}

\medskip
Using the one sided presentation, the units of $\Z/m\Z$ are the
integers less than $m$ that are relatively prime to $m$.  Let
$\{\,u_1, u_2, \ldots, u_{\phi(m)}\,\}$ be an enumeration of the units
modulo $m$.  If $a$ is also a unit then the set $\{\,a u_1, a u_2,
\ldots, a u_{\phi(m)}\,\}$ is also an enumeration of the units.  Since
these two sets contain the same elements, the products of their
elements are equal:
\[
u_1 u_2 \cdots u_{\phi(m)} = a^{\phi(m)} u_1 u_2 \cdots u_{\phi(m)}.
\]
This proves {\Fermat}'s ``Little Theorem:''\index{Fermat's little theorem}

\begin{proposition} \label{FermatLittle:Prop}
If $a$ is an invertible element of $\Z/m\Z$ then 
\[
a^{\phi(m)} = 1 \mod{m}.
\]
\end{proposition}

The order\index{order, multiplicative} of $a \in \Z/m\Z$ is the
smallest positive integer $r$ such that $a^r = 1 \mod{m}$.  The order
of a zero divisor is not defined.  Notice that the inverse of $a$ is
$a^{r-1} \mod{m}$, so any element that has a well defined order also
has an inverse.  The order of a unit must divide $\phi(m)$ by
\propref{FermatLittle:Prop}.

We can determine the order of $a \in \Z/m\Z$ fairly easily.  Factor
$\phi(m)$:
\[
\phi(m) = p_1^{n_1} \cdots p_k^{n_k}.
\]
Thus we only need to determine the power of each $p_i$ that divides
the order of $a$.  For each $p_i$, compute the sequence of elements
\[
u_0 = a^{\phi(m)/p_i^{n_i}},\quad u_1 = a^{\phi(m)/p_i^{n_i-1}},\quad 
  u_2 = a^{\phi(m)/p_i^{n_i-2}},\quad \ldots \pmod{m}. 
\]
Let $u_j$ be the first element of this sequence that is $1$.  Then
$p_i^j$ divides the order of $a$.  By using repeated squaring, $u_0$
can be computed in approximately $O(\log \phi(m))$ operations.  The 
subsequent elements of the sequence are easily computed since $u_{k+1}
= u_k^{p_i}$.  Each subsequent element of the sequence can be computed
using $O(\log p_i)$ operations.  Thus the total number of operations
required to compute one factor of the order of $a$ is at most
\[
O(\log \phi(m)) + \overbrace{O(\log p_i) + \cdots + O(\log p_i)}^{n_i}
  = O(\log \phi(m)).
\]
Since $\phi(m)$ has at most $\log \phi(m)$ prime factors, the time
required to compute the order of $a$ is $O(\log^2 \phi(m)) = O(\log^2
m)$ (by \longpropref{Totient:Average:Prop}) assuming the factorization
of $\phi(m))$ is known.

Now assume that $q$ is a prime.  By adding $1$ to itself some number
of times we can generate any element of $\Z/q\Z$.  In this case $1$ is
called an {\em additive generator}\index{generator, additive} of
$\Z/q\Z$.  There are also {\em multiplicative generators}
\index{generator, multiplicative} for $\Z/q\Z$, which are  called {\em
primitive roots\/}.\index{primitive root} Finding a primitive root
can be difficult, but verifying that $g$ is one is easy.  Assume that
$\phi(q)$ factors as
\[
\phi(q) = q - 1 = p_1^{n_1} \cdots p_k^{n_k}.
\]
Since $g$ is an element of $\Z/q\Z$, its order must divide $q-1$.  To
be a primitive root, the order of $g$ must be $q-1$.  If
$g^{(q-1)/p_i} = 1 \mod{q}$ for any $p_i$ that divides $q-1$ then its
order is less than $q-1$ and it cannot be a generator.  As we saw
before, the cost of checking the order of a candidate is $O(\log^2 q)$
plus the cost of factoring $q-1$.  For large $q$ the cost of factoring
$q-1$ can easily dominate $O(\log^2 q)$.

\section{Chinese Remainder Theorem}
\label{Integer:Chinese:Remainder:Sec}

\index{Chinese remainder theorem|(}

In algebraic manipulation, there are numerous problems for which we
know, {\em a priori\/}, that the solution is an integer.  Often it
would be much easier to solve the problem if we were looking for a
solution modulo $p$ because then it would be possible to divide
quantities without producing fractions.  Furthermore, the size of the
numbers that occur in arithmetic computation in finite fields is
constant, while this is not true over the integers.  The Chinese
remainder theorem allows us to combine the solutions to the problem
modulo $p$ to determine the original solution over the rational
integers.

For simplicity, assume that the original problem is to determine the
integer $x$ and we have a way of determining the value of $x$ modulo
any prime $p_i$.  We will refer to the value of $x$ modulo $p_i$ as
$k_i$.  Just knowing the solution modulo $p$ is not enough.  If we
know the value of $x$ modulo $p_1$ and $p_2$, we will be able to
determine a unique value for $x$ modulo $p_1 p_2$.  Given a bound for
the absolute value of $x$, $B$, and picking a set of primes $p_i$
whose product is greater than $2B+1$ we can determine the value of $x$
from its values modulo $p_i$.

We begin with the simple case.  Given the value of $x$ modulo $p_1$
and $p_2$, determine the value of $x$ modulo $p_1 p_2$, which we call
$k_{12}$.  Since $x = k_1$ modulo $p_1$ we know that
\[
x = k_1 + q p_1
\]
for some integral value of $q$.  We use the fact that $x$ is equal to
$k_2$ modulo $p_2$ to fix the value of $q$.

If we choose 
\[
q = p_1^{-1} (k_2 - k_1) \pmod{p_2}
\]
then we get
\[
x = k_1 + {p_1^{-1}}_{(p_2)}\,(k_2 - k_1) p_1 + q^{\prime} p_1
p_2,
\]
where ${p_1^{-1}}_{(p_2)}$ indicates that the computation of reciprocal
of $p_1$ is done modulo $p_2$.  Thus we can choose 
\[
k_{12} = x \pmod{p_1 p_2}.
\]
Notice that the only operation needed other than addition and multiplication
was the  computation of the inverse of $p_1$ modulo $p_2$.  This can be done
whenever $p_1$ and $p_2$ are relatively prime---it is not necessary to
restrict the $p_i$ to be primes.

\smallskip
Given the value of $x$ modulo several primes, $p_i$, the Chinese
remainder algorithm can be repeatedly applied to pairs of equations to
give:
\[
\begin{aligned}
  k_{12}&= x \pmod{p_1 p_2}, \\
  k_{123}&= x \pmod{p_1 p_2 p_3}, \\
    & \vdots\\
  k_{1 \cdots m}&= x \pmod{p_1 p_2 \cdots p_m}.
\end{aligned}
\]

An equivalent way to understand the the Chinese remainder algorithm is
that it provides a constructive isomorphism between $\Z/p_1\Z \oplus
\Z/p_2\Z$ and $\Z/p_1p_2\Z$ when $p_1$ and $p_2$ are relatively prime.
We have only shown the Chinese remainder algorithm to be injective,
but since $\phi(p_1 p_2) = \phi(p_1) \phi(p_2)$ it must also be
surjective.  We state this result as the following proposition.
\addsymbol{$A \oplus B$}{The tensor product of $A$ and $B$.}

\begin{proposition} \label{FF:UnitDecomp:Prop}
Let $q_1, \ldots, q_m$ be pairwise relatively prime (but not necessarily
prime).  As rings we have the following constructive isomorphism
\[
\Z/q_1 \cdots q_m \Z \cong (\Z/q_1\Z) \oplus \cdots \oplus (\Z/q_m\Z),
\]
and as groups
\[
U(\Z/q_1 \cdots q_m \Z) \cong U(\Z/q_1\Z) \oplus \cdots \oplus U(\Z/q_m\Z).
\]
\end{proposition}

\index{Chinese remainder theorem|)}

\section{Multiplicative Structure of ${\protect\Z}/m{\protect\Z}$}
\label{FF:Multiplicative:Sec}

The simplest case of $\Z/m\Z$ is when $m$ is a prime.  For clarity, we
will write this case as $\Z/p\Z$.  In this case, we shall see that the
units of $\Z/p\Z$ form a cyclic group.  The first question we have
about this group is how many primitive roots\index{primitive root} it
has.  We first need the following useful proposition.

\begin{proposition}\label{FFZeroBound:Prop}
Let $k$ be a field, and $F(X) \in k[X]$ be a polynomial of degree $n$.
Then $F(X)$ has at most $n$ distinct zeroes.
\end{proposition}

\begin{proof}
If $F(X)$ has no zeroes, then the proposition is true.  Let $\alpha$
be a zero of $F(X)$, and divide $F(X)$ by $X-\alpha$,
\[
F(X) = Q(X) (X - \alpha) + R.
\]
Substituting $X = \alpha$, we see that $R = 0$ and $F(X) = Q(X) (X -
\alpha)$.  Since, $k$ has no zero divisors, the degree of $Q(X)$ is
$n- 1$.

We now proceed by induction.  For $n = 1$, $Q(X)$ is a constant and
$F(X)$ only vanishes for $X = \alpha$.  For larger $n$, $Q(X)$ has at
most $n-1$ zeroes, so $F(X)$ has at most $n$.
\end{proof}

By Fermat's little theorem \propref{FermatLittle:Prop}, $X^{p-1} - 1$
has $p-1$ distinct zeroes.  Let $k$ be a divisor of $p-1$.  As a
consequence of \propref{FFZeroBound:Prop}, $X^k-1$ has at most $k$
solutions over $\Z/p\Z$.  Since $X^k-1$ divides $X^{p-1}-1$, $X^k-1$
must have exactly $k$ zeroes.  Consequently the number of elements of
$\Z/p\Z$ whose order divides $k$ is $k$.

Let $\psi(k)$ denote the number of elements of $\Z/p\Z$ that have
order $k$. ($k$ still divides $p-1$.)  Then
\[
k = \sum_{d \mid k} \psi(d).
\]
Applying the M\"obius inversion formula (\propref{Mobius:Prop}), we have
\[
\psi(k) = \sum_{d \mid k} \mu(d) \frac{k}{d}.
\]
By \propref{TotientSum:Prop}, $\psi(k) = \phi(k)$.  Therefore the
number of primitive roots in $\Z/p\Z$ is $\phi(p-1)$.\index{primitive root}

Recall from \propref{Totient:Average:Prop} that the average order of
$\phi(p-1)$ is
\[
\frac{6}{\pi^2} (p-1) = 0.60793 (p -1).
\]
If the generators of $\Z/p\Z$ are distributed uniformly then
approximately $0.60793 p$ candidates need to be examined before
finding a multiplicative generator.

\medskip
For composite $m$, $\Z/m\Z$ contains zero divisors and thus cannot be
a field.  However, the units of $\Z/m\Z$ form a group.  By
\propref{FF:UnitDecomp:Prop}, we need only to determine the structure
of $\Z/p^{\ell}\Z$.  If $p$ is odd then $\Z/p^{\ell}\Z$ is cyclic as
we show in \propref{FF:Odd:p:Struct:Prop}.  If $p$ is even the
situation is slightly more complex.

\begin{proposition} \label{FF:Odd:p:Struct:Prop}
For odd primes $p$, $U(\Z/p^{\ell}\Z)$ is a cyclic
group. 
\end{proposition}

\begin{proof}
This proposition is proven by constructing a \key{primitive root} of
$\Z/p^{\ell}\Z$.  Let $g$ be a primitive root of $\Z/p\Z$.  Modulo
$p^2$ we must have 
\[
g^{p-1} = 1 + a p \pmod{p^2}.
\]
If $a = 0$ then replace $g$ by $g+p$, so
\[
(g+p)^{p-1} = g^{p-1}+ (p-1) p g^{p-2} = 1 + a' p \pmod{p^2}.
\]
In either case we can produce a primitive root modulo $p$, $g$, such
that
\[
g^{p-1} = 1 + a p \pmod{p^2},
\]
and $a$ is not a multiple of $p$.

Since the order of $g$ modulo $p$ is $p-1$, $p-1$ divides $g$'s order
as an element of $\Z/p^{\ell}\Z$.  If we can show that $1+ap$ has
order $p^{\ell-1}$ modulo $p^{\ell}$, then $g$'s order is
$p^{\ell-1}(p-1) =
\phi(p^{\ell})$ and $g$ must be a \key{primitive root}.

To show that $1+ap$ has order $p^{\ell-1}$ we observe that 
\[
\begin {aligned}
 (1+ap)^{p^{k-1}} & = 1 + p^{k-1} \cdot p \cdot a
   + \frac{p^{k-1}(p^{k-1} - 1)}{2} \cdot p^2 \cdot a^2 + \cdots 
   + p^{p^{k-1}} a^{p^{k-1}} \\
  & = 1 + p^k \cdot a + p^{k+1} \frac{a^2 (p^{k-1} - 1)}{2} 
   + \cdots + p^{p^{k-1}} a^{p^{k-1}} \\
  & = 1 + p^k(a + bp)
\end{aligned}
\]
modulo $p^{\ell}$, where $b$ is not a multiple of $p$.
So 
\[
(1+ap)^{p^{k-1}} 
 \begin{array}{rl}
  =  1 & \mbox{if $k=\ell$,} \\
  \not= 1 & \mbox{for all smaller $k$.}
 \end{array}
\]
\end{proof}

%\addsymbol{$\mathfrak{C}_n$}{The cyclic group of $n$ elements}
The structure of $U(\Z/2^{\ell}\Z)$ is a bit more
complex.  $U(\Z/2\Z)$ has one element and $U(\Z/4\Z)$ has two, so both
are cyclic.  However, $U(\Z/8\Z) = \{\,1, 3, 5, 7\,\}$ and none of its
elements has order greater than $2$.  Letting $\mathfrak{C}_n$ denote the
cyclic group of $n$ elements, we have $U(\Z/8\Z) \cong \mathfrak{C}_2
\oplus \mathfrak{C}_2$.  More generally, we claim that for $\ell \ge 3$,
$U(\Z/2^{\ell}\Z) \cong \mathfrak{C}_2 \oplus \mathfrak{C}_{2^{\ell-2}}$.
We can prove this by showing that $5$ has order $2^{\ell - 2}$ modulo
$2^{\ell}$. 

To do this we must show that $5^{2^{\ell -3}} \not= 1
\mod{2^{\ell}}$ and $5^{2^{\ell -2}} = 1
\mod{2^{\ell}}$.  For $\ell$ equal to $3$, $4$ and $5$ we have
\[
\begin {aligned}
  5^{2^{3-3}} & = 1 + 2^2 \pmod{2^3}, \\
  5^{2^{4-3}} & = 1 + 2^3 \pmod{2^4}, \\
  5^{2^{5-3}} & = 1 + 2^4 \pmod{2^5}.
\end{aligned}
\]

Assume $5^{2^{\ell-3}} = 1 + 2^{\ell -1} \mod{2^\ell}$, \ie,
\[
  5^{2^{\ell-3}} = 1 + 2^{\ell-1} + m 2^{\ell}.
\]
Squaring gives
\[
\begin{aligned}
  5^{2^{\ell-2}} 
    &= 1 + 2^{\ell} + 2^{2\ell-2} + m 2^{\ell+1} + m 2^{2\ell} + m^2 2^{2\ell}, \\
    & = 1 + 2^{\ell} \pmod{2^{\ell+1}},
\end{aligned}
\]
for $\ell \ge 3$. This gives the following proposition.

\begin{proposition}
$\Z/2^{\ell}\Z$ is cyclic for $\ell$ equal to $1$ and $2$.  For $\ell
\ge 3$, 
\[
\Z/2^{\ell}\Z \cong \mathfrak{C}_2 \oplus \mathfrak{C}_{2^{\ell-2}}
\]
and $5$ is an element of order $2^{\ell-2}$.
\end{proposition}


\section{Quadratic Reciprocity}
\label{Quadratic:Reciprocity:Sec}

One of the most beautiful results of elementary number theory is the
quadratic reciprocity theorem, which is discussed in this section.   

Throughout we assume that $p$ is an odd prime. Some of the non-zero residues
modulo $p$ are perfect squares and some are not.  Those that are
perfect squares are called {\em quadratic residues}\index{quadratic
residue} and those that are not, are called {\em quadratic
non-residues}.\index{quadratic residue} The property of being a
quadratic residue is \keyi{quadratic residuacity}.

The image of $\F_p$ under the map $a \mapsto a^2$ is the set of
quadratic residues of $\F_p$.  Since $a^2 = (p-a)^2 \mod{p}$, no
quadratic residue has fewer than $2$ pre-images.  Since the equation
$x^2 = a
\mod{p}$ can have no more than $2$ zeroes, each quadratic residue has
exactly 2 pre-images.  There are exactly $(p-1)/2$ quadratic residues
and $(p-1)/2$ quadratic non-residues in $\F_p^{\ast}$.

By \propref{FermatLittle:Prop} we have
\[
0 = a^{p-1} - 1 = (a^{(p-1)/2} - 1)\,  (a^{(p-1)/2} + 1) \pmod{p}
\]
for every element $a$ of $\F_p^{\ast}$.  Half of the units modulo $p$
satisfy $a^{(p-1)/2} = 1$ and the other half satisfy $a^{(p-1)/2} =
-1$ modulo $p$.  No quadratic residue can satisfy the second equation,
so we have the following proposition:

\begin{proposition} \label{FF:Euler:Residue:Prop}
Modulo $p$, $a$ is a perfect square if $a^{(p-1)/2}$ is equal to $1$
and is not a perfect square if $a^{(p-1)/2}$ is equal to $-1$.
\end{proposition}

We define the Jacobi symbol 
\[
\left(\frac{p}{q}\right) = 
  \begin{cases}
    1 & \text{if $p$ is a quadratic residue modulo $q$,} \\
    -1 & \text{if $p$ is not a quadratic residue,} \\
    0 & \text{if $p$ is a zero divisor of $q$.}
  \end{cases}
\]
From the previous discussion we have:
If $p$ is an odd prime
\begin{equation}\label{Jacobi:Sym:Eq}
\left(\frac{a}{p}\right) = a^{frac{p -1}{2}} \pmod{p}.
\end{equation}

The theorem of quadratic reciprocity is:

\begin{proposition}[Gauss]
\label{QuadraticReciproc:Prop}
If $p$ and $q$ are both odd primes
\[
\left({p \over q}\right)  \left({q \over p}\right) =
(-1)^{{p -1 \over 2} {q - 1 \over 2}}.
\]
Also,
\[
\left(\frac{2}{q}\right) = (-1)^{(q^2-1)/8}.
\]
\end{proposition}

Let $S = \{\,1, \ldots, (p-1)/2\,\}$ be a subset of $\F_p^{\ast}$.
Notice that $S$ is disjoint from $-S$.  If $a$ is an element of
$\F_p^{\ast}$ and $s \in S$ then either $as$ or $-as$ is an element of
$S$.  We can write this as $as = (a, s)_G s_a$, where $s_a \in S$ and
$(a, s)_G = \pm 1$.

Let $a$ be fixed and let $s$ and $s'$ be two distinct elements of $S$.
Then $s_a \not= s'_a$, otherwise $as  = \pm as'$ which implies $s =
\pm s'$.  Therefore, $s \mapsto s_a$ injectively maps $S$ onto itself.

The first step in the proof of quadratic reciprocity is Gauss's lemma:

\begin{proposition}[Gauss's Lemma]
\[
\left(\frac{a}{p}\right) = \prod_{s \in S} (a, s)_G
\]
\end{proposition}

\begin{proof}
We have $as = (a,s)_G s_a$.  Multiplying this for all $s \in S$ gives
\[
a^{(p-1)/2} \prod_{s\in S} s = \prod_{s\in S} (a, s)_G s_a
 = \prod_{s\in S} (a, s)_G s.
\]
The proposition follows from \eqnref{Jacobi:Sym:Eq}.
\end{proof}


Gauss's lemma allows us to determine the quadratic character of $2$,
because we can explicitly evaluate $(2, s)_G$.
\[
(2, s)_G = \left\{
\begin{array}{ll}
1 & \mbox{if $1 \le s \le (p-1)/4$} \\
-1 & \mbox{if $(p-1)/4 < s \le (p-1)/2$} 
\end{array}
\right.
\]
So to determine $(2/p)$ we need to count the number of integers
in the sequence
\[
\left\lceil \frac{p-1}{4} \right\rceil, 
\left\lceil \frac{p-1}{4} \right\rceil + 1, \ldots, 
\frac{p-1}{2}.
\]
There are four cases to consider depending
on the residue of $p$ modulo $8$:
\[
\begin{array}{crccclcc}
p= 8n+1: & 2n     & < & s & \le & 4n   & \Longrightarrow & 2n \\
p= 8n+3: & 2n+1/2 & < & s & \le & 4n+1 & \Longrightarrow & 2n+1 \\
p= 8n+5: & 2n+2   & < & s & \le & 4n+2 & \Longrightarrow & 2n+1 \\
p= 8n+7: & 2n+3/2 & < & s & \le & 4n+3 & \Longrightarrow & 2n+2 
\end{array}
\]
so $2$ is a quadratic residue if $p = 8n\pm 1$ and a non-residue if $p
= 8n\pm3$.  This can be abbreviated as
\[
\left(\frac{2}{p}\right) = (-1)^{(p^2-1)/8}.
\]

We now need three simple computational propositions before proving
\propref{QuadraticReciproc:Prop}. 

\begin{proposition}\label{QR:Help:a:Prop}
When $n > 0$ is odd and $\zeta = e^{2\pi i / n}$ is a primitive
$n$-th root of $1$,
\[
x^n - y^n = \prod_{k = 0}^{n-1} ( \zeta^k x - \zeta^{-k} y).
\]
\end{proposition}
\begin{proof}
Since $-2$ has an inverse modulo $n$, the sequence
\[
0, -2, \ldots, -2k, \ldots , -2(n-1) \quad\mbox{modulo $n$}
\]
is just a permutation of the integers $0, 1, \ldots, n-1$.  Therefore,
\[
\begin{aligned}
  \prod_{k=0}^{n-1}(\zeta^k x - \zeta^{-k}y) 
     & = \zeta^{n(n-1)/2} \prod_{k=0}^{n-1}(x - \zeta^{-2k}y) 
       = \prod_{\ell=0}^{n-1}(x - \zeta^{\ell}y) \\
     & = x^n - y^n
\end{aligned}
\]
\end{proof}

This proposition can be used to derive another identity involving
trigonometric functions.  For simplicity, we write this in terms of
complex exponentials although {\Eisenstein} used trigonometric functions.
Define $f(z) = e^{2 \pi i z} - e^{-2 \pi i z}$ ($ = 2 i \sin 2\pi z$).
We have adjusted the constants so that $f(z)$ satisfies the following
identities:
\[
 f(z) = f(z+1),\quad \mbox{and} \quad f(-z) = - f(z).
\]

\begin{proposition} \label{QR:Help:b:Prop}
\[
\frac{f(nz)}{f(z)} = 
  \prod_{k=1}^{(n-1)/2} 
     f\left(z + \frac{k}{n}\right) f\left(z - \frac{k}{n}\right)
\]
\end{proposition}

\begin{proof}
Letting $x = e^{2 \pi i z}$ and $y = e^{-2 \pi i z}$ we have $f(z) = x
- y$, $f(n) = x^n - y^n$ and 
\[
f(z + \frac{k}{n}) = \zeta^k x - \zeta^{-k} y.
\]
Using \propref{QR:Help:a:Prop}, 
\[
\frac{f(nz)}{f(z)} =
 \prod_{k=1}^{n-1} f\left( z + \frac{k}{n}\right) = 
 \prod_{k=1}^{(n-1)/2} f\left( z + \frac{k}{n}\right) 
 \prod_{k=(n+1)/2}^{n-1} f\left( z + \frac{k}{n}\right)
\]
Since $f(z +k/n) = f(z - (n-k)/n)$,
\[
\frac{f(nz)}{f(z)}  =
 \prod_{k=1}^{(n-1)/2} f\left( z + \frac{k}{n}\right)
 \prod_{k=(n+1)/2}^{n-1} f\left( z - \frac{n-k}{n}\right).
\]
Replacing $k$ by $n - \ell$ in the second product and observing that 
\[
\frac{n+1}{2} \le n - \ell \le n - 1
\qquad\mbox{implies}\qquad
1 \le k \le \frac{n-1}{2}
\]
gives
\[
\frac{f(nz)}{f(z)}  =
 \prod_{k=1}^{(n-1)/2} f\left( z + \frac{k}{n}\right)
 \prod_{k=1}^{(n-1)/2} f\left( z - \frac{k}{n}\right),
\]
which proves the proposition.
\end{proof}

Finally, we have a proposition that relates the complex exponentials
$f(z)$ with the Jacobi symbol.

\begin{proposition}\label{QR:Help:c:Prop}
If $p$ is an odd prime integer and $p \nmid a$ then 
\[
\prod_{s=1}^{(p-1)/2} f\left(\frac{sa}{p}\right) = 
\left(\frac{a}{p}\right) 
  \prod_{s=1}^{(p-1)/2} f\left(\frac{s}{p}\right),
\]
where $(a/p)$ is the Jacobi symbol.
\end{proposition}

\begin{proof}
Using $sa = (s, a)_G s_a$ modulo $p$ and the identities satisfied by
$f(z)$
\[
f\left(\frac{sa}{p}\right) = (s, a)_G f\left(\frac{s_a}{p}\right).
\]
Taking the product over all $s \in S = \{\, 1, \ldots, (p-1)/2\,\}$
and applying Gauss's lemma gives the result
\end{proof}

Now the proof quadratic reciprocity theorem is quite simple.
Combining \propref{QR:Help:c:Prop} and the identity for $f(z)$,
\propref{QR:Help:b:Prop}, gives 
\[
\left(\frac{q}{p}\right) = 
\prod_{s=1}^{(p-1)/2} \frac{f(sq/p)}{f(s/p)} = 
\prod_{\ell=1}^{(p-1)/2} \prod_{s=1}^{(q-1)/2}
  f\left(\frac{s}{p} + \frac{l}{q}\right) 
  f\left(\frac{s}{p} - \frac{l}{q}\right).
\]
Interchanging $p$ and $q$ gives
\[
\left(\frac{p}{q}\right) = 
\prod_{s=1}^{(p-1)/2} \prod_{s=1}^{(q-1)/2}
  f\left(\frac{s}{q} + \frac{l}{p}\right) 
  f\left(\frac{s}{q} - \frac{l}{p}\right).
\]
Since, $f(-z) = - f(z)$ we have
\[
\left(\frac{p}{q}\right) 
= \left(\frac{q}{p}\right) (-1)^{\frac{p-1}{2} \frac{q-1}{2}}.
\]

\section{Algebraic Extensions of $\protect\F_p$}
\label{FF:AlgExt:Sec}

Not all finite fields have a prime number of elements.  This section
gives some basic results about the structure of other finite fields
as well as some more general results about fields.  

Let $F$ be a field whose multiplicative identity we denote by $e$.
The {\em characteristic}\index{characteristic! of a field} of $F$ is
either the smallest $n$ such that 
\[
\overbrace{e + e + \cdots + e}^{n-{\rm times}} = 0,
\]
or $0$ if there is no finite sum of $e$'s that vanishes.  We write
$\Char F = n$ in this case, \eg, $\Char \Q = 0$ and $\Char \F_p = p$.

Consider the map $\Z\rightarrow F$ defined by $n \mapsto n \cdot e$.
This is a ring homomorphism, so the image of $\Z$ in $F$ must be a
subring of $F$.  In particular, it must be an integral domain since
$F$ has no zero divisors.   The kernel of $\Z \rightarrow F$ must
therefore be a prime ideal, either $(0)$ or $(p)$, where $p$ is a
prime.  We therefore have:
\begin{proposition}
The characteristic of a field is either $0$ or a prime number.
\end{proposition}

Let $F$ be a finite field of characteristic $p$ ($> 0$) and with $q$
elements.  Identify $\F_p$ with the image of $\Z$ in $F$.  It is easy
to see that $F$ is a finite dimensional vector space over $\F_p$.
Denote its generators by $\omega_1, \ldots, \omega_r$, so each element
$f \in F$ is of the form
\[
f = a_1 \omega_1 + \cdots + a_r \omega_r.
\]
We have just shown
\begin{proposition}
Every finite field of characteristic $p$ has $p^r$ elements for some
integer $r$.
\end{proposition}

Now consider the multiplicative group of $F$, $F^{\ast}$.  $F^{\ast}$
has $q-1$ elements, so for every $a \in \F^{\ast}$, 
\[
a^{q-1} = 1.
\]
We immediately have the factorization
\[
X^q - X = \prod_{a \in F} (X - a).
\]

\begin{proposition} \label{FF:AlgExtOrder:Prop}
$F^{\ast}$ is cyclic and has $\phi(k)$ elements of multiplicative order
$k$, where $k\mid q-1$.
\end{proposition}

\begin{proof}
This is nearly identical to the approach in
\sectref{FF:Multiplicative:Sec}.  Let $\psi(k)$ denote the number of
elements of $F^{\ast}$ with order $k$.  The elements whose order
divides $k$ are all zeroes of $X^k - 1$.  If $k$ divides $q$, then $X^k
-1$ has $k$ zeroes, thus
\[
k = \sum_{d\mid k} \psi(d).
\]
By the M\"{o}bius inversion formula,\index{Moebius inversion
formula@M\"{o}bius inversion formula}
\[
\psi(k) = \sum_{d\mid k} \mu(d)\frac{k}{d} = \phi(k).
\]
Therefore there are $\phi(q-1) > 1$ elements of $F^{\ast}$ of order
$q-1$.
\end{proof}

As a corollary to \propref{FF:AlgExtOrder:Prop} we see that $\F_{p^r}^{\ast}$
is a cyclic group since it must have a \key{primitive root}.

\begin{proposition}\label{Cycl:Factor:Prop}
Let $X^d-1$ and $X^n -1 $ be polynomials over a field.  $X^d-1$
divides $X^n-1$ if and only if $d$ divides $n$.
\end{proposition}

\begin{proof}
Let $n = d\cdot m + r$, where $r < d$, so
\begin{equation}\label{Cycl:Factor:Eq}
\frac{X^n -1}{X^d -1} = X^r \frac{X^{d\cdot m} -1}{X^d -1} +
\frac{X^r - 1}{X^d - 1}.
\end{equation}
The last term in \eqnref{Cycl:Factor:Eq} is a polynomial if and only if
$r = 0$.  The coefficient of $X^r$ is a polynomial since
\[
\frac{Y^m - 1}{Y-1} = Y^{m-1} + Y^{m-2} + \cdots + 1.
\]
\end{proof}

Let $F_d(X)$ denote the product of the monic irreducible polynomials of
degree $d$ over $\F_p$.  So,
\[
F_1(X) = X^p - X.
\]
The following proposition is central to the polynomial factorization
algorithm discussed in \sectref{PolyFF:Cantor:Sec}.

\begin{proposition}
\[
X^{p^n}- X = \prod_{d\mid n} F_d(X)
\]
\end{proposition}

\begin{proof}
This is equivalent to saying that if $F(X)$ is a monic irreducible
polynomial of degree $d$, then $F(X) \mid X^{p^n}-X$ if and only if
$d\mid n$, since $X^{p^n} -X$ is square free.

Let $K = \F_p[\alpha]/(F(\alpha))$.  $K$ is a finite field with $p^d$
elements so every element  $\beta \in K$ satisfies $\beta^{p^d-1} = 1$
and all the zeroes of $X^{p^d}-X$ lie in $K$.  In particular,
$\alpha^{p^d-1} = 1$.  Divide $X^{p^d}-X$ by $F(X)$:
\[
X^{p^d}-X = F(X) Q(X) + R(X),
\]
where the degree of $R(X) < d$.  Substituting $X = \alpha$ into this
equation, we see that $0 = R(\alpha)$ so $R(X) = 0$ and $F(X)$ divides
$X^{p^d}-X$.  By \propref{Cycl:Factor:Prop}, $F(X)$ divides
$X^{p^n}-X$ since $d$ divides $n$.

Now assume $f(X)$ divides $X^{p^n}-X$.  We need to show $d \mid n$.
Since
\[
X^{p^n}-X = f(X) q(X),
\]
$\alpha^{p^n}-\alpha = 0$.  Every element of $K$ can be written in the
form, $a_0 \alpha^{d-1} + \cdots a_{d-1}$ and
\[
  (a_0 \alpha^{d-1} + \cdots a_{d-1})^{p^n}  =
a_0 (\alpha^{d-1})^{p^n} + \cdots a_{d-1}
= a_0 \alpha^{d-1} + \cdots a_{d-1}.
\]
So, every element of $K$ is a zero of $X^{p^n} -X$.  This means that
$X^{p^d}-X$ divides $X^{p^n}-X$, and $d\mid n$.
\end{proof}

\section{$p$-adic Numbers}
\label{padic:Arith:Sec}

\index{p-adic number@$p$-adic number|(}

One of the most useful tools for problems over the real numbers is the
absolute value valuation.  Using the absolute value we can define the
distance between two real numbers. For instance, $x$ and $y$ are
considered close if $|x - y|$ is small.  For the integers there is
another valuation that is useful.  We would like to consider $x$ and
$y$ close if $x=y \mod{p^k}$ for large values of $k$. This is called
the $p$-adic valuation.

We define the $p$-adic valuation as follows.  Let $a$ be an integer,
and assume $p^r$ divides $a$ but $p^{r+1}$ does not.  We can write $a$
as $p^r m$, where $m$ is an integer relatively prime to $p$.  Then the
{\em $p$-adic valuation}\index{p-adic valuations@$p$-adic valuations}
of $a$ is
\[
 \| a \|_p = \| p^r m \|_p = p^{-r}.
\]
When $a=0$ we will say that $r = \infty$ and $\|a\|_p = 0$.

This definition can also be extended to the rational numbers.  All
rational numbers can be written in the form $p^r m/n$ where $p$ does
not divide $m$ or $n$, and $r$ can be positive, negative or zero.
Then the $p$-adic valuation of $p^r m/n$ is
\[
\left\| p^r \frac{m}{n} \right\|_p = p^{-r}.
\]
Notice that we have used the same notation for the $p$-norm of a
lattice (see \sectref{Lattice:Fund:Sec}) and the $p$-adic valuation.
This should not cause confusion since the $p$-norms are only applied
to elements of $\R^n$ while $p$-adic valuations are only applied to
elements of $\Q$.

\smallskip
In elementary analysis and topology, absolute values are used to define
convergent sequences and the convergent sequences are used to ``complete''
the rational numbers to yield the reals.  We will now use the $p$-adic
valuation to complete the integers to give the {\em $p$-adic
integers},\index{p-adic integer@$p$-adic integer}
$\Z_p$.
\addsymbol{$\Z_p$}{The $p$-adic integers}

The basic idea is quite simple.  Let
\[
A = \{\,\alpha_0, \alpha_1, \alpha_2, \ldots, \alpha_k, \ldots\,\}
\]
be a sequence of rational integers such that
\[
\| \alpha_i - \alpha_j \|_p \le \frac{1}{p^{1+\min (i,j)}}.
\]
Thus $\alpha_i - \alpha_{i-1} = a_i p^i$.  We view $\{\,\alpha_0,
\alpha_1, \ldots, \alpha_k\,\}$ as a sequence of increasingly better
approximations of $\alpha_k$.  The whole set is thus a sequence of
increasingly good approximations to
\[
\lim_{k \rightarrow \infty} \alpha_k.
\]
This limit is often represented by the infinite series
\begin{equation}
a_0 + a_1 \cdot p + a_2 \cdot p^2 + \cdots.
\label{PSeries:Eq}
\end{equation}
The $a_i$ are chosen from the range
$0 \le a_i < p$.  This is called the {\em power series representation} for a
$p$-adic integer.

It is clear that the positive integers have a unique representation of this
form.  The $p$-adic integers also include quantities for which
\eqnref{PSeries:Eq} is an infinite series.  Assume that
$\alpha = \{\,\alpha_0, \alpha_1, \ldots\,\}$ and $\beta$ are two
$p$-adic integers such that $\alpha_k = \beta_k \mod{p^k}$, for all
$k$; that is, $\| \alpha_k - \beta_k \|_p \le p^{-k}$.  We then say
that $\alpha$ and $\beta$ are equal.  If this is the case then they
must have the same power series representations.  From now on we will
only work with the power series representations of $p$-adic integers.

Negative integers also have $p$-adic representations.  Consider the number
$-1$.  One set of good approximations is the sequence of positive integers
$p^k - 1$, since $\|-1 - (p^k -1) \|_p$ is small.  But this can be
written in the following manner
\[
\begin{aligned}
  p^k -1 & = (p-1)(p^{k-1} + p^{k-2} + \cdots + p + 1), \\
    & = (p-1) + (p-1)\cdot p + (p-1) \cdot p^2 + \cdots + (p-1) \cdot p^{k-1}.
\end{aligned}
\]
Taking the limit as $k \leftarrow \infty$ we have
\[
-1 = (p-1) + (p-1)\cdot p + (p-1) \cdot p^2 + \cdots .
\]
It is easy to see that this is one less than zero:
\[
\begin{aligned}
  -1 + 1 &=1 +  (p-1) + (p-1)\cdot p + (p-1) \cdot p^2 + \cdots \\
    &= p + (p-1) \cdot p + \cdots\\
    &= p^2 + (p-1) \cdot p^2 + \cdots\\
    &=0.
\end{aligned}
\]

\smallskip
Arithmetic operations with $p$-adic integers are performed in much the same
manner as with power series.  The only difference is that a normalization
step must be performed to ensure each coefficient of $p^k$ lies in the
range $0 \le a_k < p$.  For instance, to compute
\[
q^2 = (1 + 3 + 3^2 + 2\cdot 3^4 + \cdots)^2
\]
in the $3$-adic integers we begin by performing the power series
multiplication. 
\[
\begin{aligned}
  q^2&= 1 + 2\cdot 3 + 3 \cdot 3^2 + 2\cdot 3^3 + 5 \cdot 3^4 + \cdots\\
    &= 1 + 2\cdot 3 + (1 + 2) \cdot 3^3 + 5 \cdot 3^4 + \cdots\\
    &= 1 + 2 \cdot 3 + (1 + 5) \cdot 3^4 + \cdots\\
    &= 1 + 2 \cdot 3 + \cdots
\end{aligned}
\]

The $p$-adic integers clearly include all of $\Z$ since the positive
integers and $-1$ are $p$-adic integers.  There are other elements also.
For instance,
\[
-{1 \over 2} = {1 \over 1 - 3} = 1 + 3 + 3^2 + 3^3 + \cdots.
\]
So, $-1/2$ is an element of $\Z_3$.

Certain algebraic numbers also lie in $\Z_p$.  We will now show that $X^2 -
7$ has a solution that lies in $\Z_3$.  Let the symbol
$\sqrt{7}$ represent a solution of $X^2-7$.  
If $\sqrt{7}$ does lie in $\Z_3$ then we must have
\[
\sqrt{7} = a_0 + a_1 \cdot 3 + a_2 \cdot 3^2 + \cdots.
\]
Therefore, we must have $a_0^2 - 7 = 0 \mod{3}$, so $a_0$ is either 1
or $-1$.  We pick 1.  Choosing $-1$ leads to the other solution of $X^2 -
7$ in a similar manner.

As with the computation of $a_0$, the computation of the other $a_i$
proceeds by considering the equation
\[
(a_0 + a_1 \cdot 3 + \cdots)^2 - 7 = 0 \pmod{3^{i+1}}
\]
and solving the resulting equation for $a_i$.  Thus to determine
$a_1$,
\[
(1 + a_1 \cdot 3 + \cdots)^2 - 7 = 
   1 +2 a_1 \cdot 3 - 7 = 0 \pmod{3^2}
\]
So $a_1 = 1$ and $\sqrt{7} = 1 + 3 + \cdots.$
Continuing
\[
7 = (1+ 3+ a_2 \cdot 3^2)^2 = 16 + 8a_2 \cdot 3^2 \pmod{3^3}
\]
so $a_2 = 1$.  This may be continued arbitrarily far:
\[
\sqrt{7} = 1 + 3 + 3^2 + 2 \cdot 3^4 + \cdots.
\]
Notice that the equations for $a_1$ and $a_2$ were both linear.  This
is generally the case.  This approach is discussed in more generality
in \chapref{Hensel:Chap}.

\index{p-adic number@$p$-adic number|)}

\section{Cryptosystems}
\label{Crypto:Sec}

\index{cryptography|(} 
Although Alan {\Turing}, arguably the first
computer scientist, spent World War II decoding German code boxes,
until 1976 there was relatively little public interest in cryptography
among computer scientists.  Cryptography was widely viewed as the
activity of the intelligence services, not a ``respectable'' academic
discipline.  All of this changed in 1976 when the the concept of {\em trap
door functions}\index{functions!trap door} and their use in
cryptograhy was introduced by {\Diffie} and {\Hellman} \cite{Diffie76}.

Assume two people, George and Bill, wish to communicate in secret.
However, all of their communications are being observed by Ross,
someone they do not want to understand their communications.  The
simplest approach to this problem is for George and Bill to agree on a
secret before hand that ``unlocks'' their communications.  For
instance, George and Bill could agree to communicate in the
Akkadian language\footnote{An ancient language used in Mesopotamia from roughly
  3000 BCE till about 500 BCE. It was replaced as he lingua franca of
  the Middle East by Aramaic and has not been widely used in over 2000 years.}
instead of English.  If Ross does not recognize Akkadian then he will
not be able to decode the transmission, even if he hears the entire
transmission.  The secret that George and Bill share, {\em before the
communication begins\/}, is that they have agreed to use Akkadian.
This approach to cryptogaphy, where both parties have a pre-agreed
secret, dates to the origin of secret sharing. 

The ``modern'' approaches to cryptogrpahy arise from asking if George
and Bill can communicate secretly {\em without} a pre-arranged key?

A general approach to this problem is based on the conjectured
existence of \keyi{one way function}s.  That is, a function $f$ for
which it is relatively easy to compute $f(M)$ for some message $M$,
but such that it is very difficult to determine $M$ from $f(M)$.  The
function $f$ can be publicized, and George can send a message to Bill
by transmitting $f(M)$.  Nothing has been revealed to Ross, since it
is difficult to determine $M$ from $f(M)$.  Unfortunately, nothing has
been revealed to Bill either.

The solution, is to find one way functions that have \key{trap doors}.
Assume that Bill has carefully constructed the function $f_B$ such
that it is exceedingly difficult to determine $M$ from $f_B(M)$ unless
some special property of $f_B$ is known.  If Bill knows this property,
then he can publicize $f_B$, and George can send Bill messages by
transmitting $f_B(M)$.  Since Ross is unaware of the special property
of $f_B(M)$ he cannot decode the message.  In fact, George cannot decode
$f_B(M)$ either, even though he computed it!

Thus one way trap door functions can be used to create public key
cryptosystems.  The question is whether one way trap door functions exist.  


\paragraph{Knapsack Cryptosystem}

The first one way, trap door cryptosystem is the \keyi{knapsack
system} originally suggested by {\Merkle} and {\Hellman}
\cite{Merkle:Hellman}.  In a knapsack system we are given a vector of
rational numbers, $(a_1, \ldots, a_n)$.  The encrypted form of an $n$
bit message, $m_1 m_2 \cdots m_n$, is the sum
\[
E = m_1 a_1 + m_2 a_2 + \cdots + m_n a_n.
\]
The $a_i$ are chosen in such a way that the receiver can decrypt the
message easily.  In particular, if $a_1 = 1$, $a_2 = 2$, $a_2 = 4$ and
so on, we can easily determine the $m_i$ from $E$.  More generally, if
we choose the $a_i$ such that 
\[
a_1 + a_2 + \cdots + a_k \le a_{k+1}
\]
for all $k$, then we can decode $E$ easily.  Rather than publishing the
$a_i$, which would then allow the an enemy to easily decrypt, we attempt
to hide the information in the $a_i$ by instead publishing
$b_i =  w \cdot a_i \mod{p}$, where $p$ is chosen larger than the
largest $a_i$.  The multiplier $w$ effectively makes the $b_i$ all
roughly the same size, hiding the quick growth of the $a_i$ that makes
the encrypted message easy to decode.  The encoded message is now
\[
E' = m_1 b_1 + m_2 b_2 + \cdots + m_n b_n.
\]
if $w$ is known, we can easily compute
\[
E' \cdot w^{-1} = m_1 a_1 + m_2 a_2 + \cdots + m_n a_n,
\pmod{p}
\]
which is easy to decode.

As it turns out, it is quite easy to decode the message even if $w$ is
not known.  The basic idea is that if $m$ is the ratio of two small
integers modulo a prime, $p$, then we can use our earlier results on
rational approximations to recover the original fraction.
Let $m$ be the ratio of two small integers $x$ and $y$ modulo $p$ and assume
the product $2xy$ is less than $p$.  Then there is an integer $q$ such that
\[
my - pq = x.
\]
Rewriting this we get
\[
{m \over p} -  {q \over y} = {x \over p y} < {x \over 2 x y^2} = {1 \over
2 y^2}.
\]
Since $x$ is assumed to be small $q/ y$ must be a convergent of the
continued fraction expansion of $m/p$.  From this it is easy to
determine $x$.  The fact that this is an algorithm depends upon
\propref{RationalCF:Prop}.\Marginpar{Check JSC 1995, vol. 20,
  287--297, Collins and Encarnacion}

\begindsacode
RationalizeMod ($m$, $p$) := $\{$ \\
\> $q_{i-1} \leftarrow 1$; $q_i \leftarrow 0$; \\
\> $y_{i-1} \leftarrow 0$; $y_i \leftarrow 1$; \\
\> $r \leftarrow m$; $s \leftarrow 1$; \\
\> loo\=p while $y_i^2 \le 2p$ do \{ \\
\>\> $a_i \leftarrow \lfloor r/s \rfloor$; \\
\>\> $\left(\!\!\begin{array}{c}s \\ r\end{array}\!\!\right) \leftarrow
   \left(\!\!\begin{array}{c} r - a_i \cdot s \\ s \end{array}\!\!\right)$; \\
\>\> $\left(\!\!\begin{array}{c}q_i \\ q_{i-1}\end{array}\!\!\right) \leftarrow
   \left(\!\!\begin{array}{c} a_i q_i + q_{i-1} \\ q_i \cdot s 
   \end{array}\!\!\right)$; \\
\>\> $\}$ \\
\> return( $(m y_{i-1} - p q_{i-1})/y_{i-1}$);
\enddsacode

\keyw{RationalizeMod} allows us to determine a rational number from
its residue modulo an integer.  If we have a problem that is easy to
solve modulo a prime, we can use the Chinese remainder theorem to
obtain the integer solutions using the technique described in the
previous section.  Actually we obtain a solution modulo $m$ for some
very large integer $m$.  Using the technique just described we can
convert this solution modulo $m$ to a rational number solution of the
problem.  This idea has been successfully used to solve very large
systems of linear equations where the solutions are known to be small
rational numbers \cite{Bender:Keener:Zippel}.

\paragraph{RSA Cryptosystem}

\index{RSA cryptosystem|(}

One of the best suggestions for a one way trap door function is
contained in the {\Rivest}-{\Shamir}-{\Adleman} cryptosystem
\cite{Rivest78}.  This system is quite simple and elegant, has not
been cracked and has helped awaken interest in the problem of
factoring rational integers.

Let $A = pq$ be the product of two large prime numbers and $R$ an
integer relatively prime to $\phi(A) = (p-1)(q-1)$.  The {\sc rsa} trap door
function is
\[
f_B(M) = M^R \mod A.
\]
If $\phi(A)$ is known, $f_B(M)$ can be easily decoded by determining
an $S$ such that
\[
RS = 1 \pmod{\phi(A)}.
\]
Then
\[
f_B(M)^S = M^{RS} = M \pmod{A}
\]
by \propref{FermatLittle:Prop}.  So Bill can decode messages easily.
In order to allow George to encrypt messages, Bill publicizes $f_B$,
\ie, $R$ and $A$, while keeping $S$ to himself.  

How hard is it for Ross to decode $f_B(M)$?  If Ross can factor $A$,
then he can compute $\phi(A)$ and thus $S$.  But if $A$ is hard to
factor some other approach is needed to crack $f_B$.  Thus far, no
general alternative to factoring $A$ has been devised.  So it appears
that the {\sc rsa} scheme is as secure as factoring integers.  

\index{RSA cryptosystem|)}
\index{cryptography|)}

\section{Sums of Squares}
\label{FF:SumSquares:Sec}

In this section
 we observe that sets of integers that satisfy systems
of linear equations over finite fields form a lattice.  When combined
with the \key{Minkowski convex body theorem}, this result allows us to prove
that all primes of the form $p= 4n+1$ are the sum of two squares and
that all integers are the sum of four squares.

The zeroes of systems of linear equations modulo primes form lattices.
The simplest example of this is a single equation in one unknown
\[
ax = 0 \pmod{p}.
\]
If $p$ divides $a$, then the set of solutions, $L$, is equal to $\Z$.  If $a$ and $p$ are relatively
prime, then $x$ must be a multiple of $p$, \ie, $L = p\Z$.  In either
case, $L$ is a lattice.  Notice that in the first case the determinant
of the lattice is $1$, while in the second it is $p$.  

Several linear equations in one variable,
\[
\begin {aligned}
a_1 \cdot x & = 0 \pmod{p_1} \\
 & \vdots \\
a_m \cdot x & = 0 \pmod{p_m} 
\end{aligned}
\]
can be converted into a a single equation, using the Chinese Remainder
theorem\index{Chinese remainder theorem}.
\[
A \cdot x = \pmod{p_1 \cdots p_m}.
\]
Again this is a lattice, and its determinant is $p_1 \cdots p_m$.

The general result is given in the following proposition. 

\begin{proposition} \label{FF:Lattice:Prop}
Let $\vec{a}_1, \ldots, \vec{a}_m$ be elements of $\Z^n$ and $p_1, \ldots,
p_m$ be primes in $\Z$.  Then the $\vec{x} \in \Z^n$ that satisfy
\begin{equation} \label{Lat:Modular:Eq}
  \begin {aligned}
    \vec{a}_1 \cdot \vec{x} & = 0 \pmod{p_1} \\
       & \vdots \\
    \vec{a}_m \cdot \vec{x} & = 0 \pmod{p_m} 
\end{aligned}
\end{equation}
form a lattice ${\cal L}$ and $\det {\cal L} \le p_1 \cdots p_m$.
\end{proposition}

\begin{proof}
To show that ${\cal L}$ is a lattice we use
\longpropref{Lattice:Condition:Prop}.  ${\cal L}$ is clearly discrete,
in particular 
we can use $0 < \eta < 1$.  If both $\vec{u}$ and $\vec{v}$ satisfy
\eqnref{Lat:Modular:Eq} then clearly $\vec{u} \pm \vec{v}$ does also.

${\cal L}$ is a sublattice of $\Z^n$.  The index of ${\cal L}$ in
$\Z^n$ is no more  than $p_1 \cdots p_m$.  Thus, by
\propref{SubLattice:Index:Prop}, $\det L \le p_1 \cdots p_m$.
\end{proof}

Recall that when $p = 4n+1$ is a prime there exists an element $i
\in \Z/p\Z$ such that $i^2+1 = 0 \mod{p}$.  By \propref{FF:Lattice:Prop},
the set of $(x_1, x_2)$ that satisfy
\[
x_1 - i x_2 = 0 \pmod{p}
\]
form a lattice with determinant no greater than $p$.  Note that
multiplying the previous equation by $x_1 + i x_2$ shows that a $k \in
\Z$ exists such that
\[
x_1^2 + x_2^2 = kp.
\]
We want to show that there exist $x_1$ and $x_2$ such that $k = 1$.

This disk $C : x_1^2 + x_2^2 < 2p$ has area 
\[
\pi \cdot (\sqrt{2 p})^2 = 2 \pi p > 2^2 p.
\]
It is a symmetric \key{convex body} and so by \propref{Minkowski:Convex:Prop}
there exists $(u_1, u_2) \in L \cap C$. Thus
\[
u_1^2 + u_2^2 = kp  < 2p,
\]
so $k= 1$ as desired.  This gives the following proposition.

\begin{proposition} \label{FF:2Squares:Sec}
All rational primes of the form $p= 4n+1$ can be written as the sum of
two squares.
\end{proposition}

The approach used to prove \propref{FF:2Squares:Sec} can also be used
to prove that all positive rational integers can be written as the sum
of four squares.
\[
m = u_1^2 + u_2^2 + u_3^2 + u_4^2.
\]
The proof follows the same general lines, but is slightly more
complicated.

\begin{proposition}
All positive rational integers $m$ can be written as the sum of four squares.
\end{proposition}

\begin{proof}
Without loss of generality we can assume that $m$ is square free and
is the product of the primes $p_1, \ldots, p_k$.  In the previous
proposition, we used the number $i$, which satisfied $i^2+1 = 0$ to
generate the lattice.  For this proposition we need to find two
integers $a_p$ and $b_p$ such that
\[
a_p^2 + b_p^2 + 1 = 0 \pmod{p},
\]
for each of $p_1, \ldots, p_k$.  If $p = 2$, then we use $a_2 = 1$,
$b_2 = 0$.

When $p$ is odd consider the two sets
\[
\{\,a^2 \mid 0 \le a \le \frac{p-1}{2}\}
\quad\mbox{and}\quad
\{\,-1 - b^2 \mid 0 \le b \le \frac{p-1}{2}\}.
\]
Each set contains $(p+1)/2$ distinct elements modulo $p$, so they must
have an element in common.  Call these elements $a_p^2$ and $-1
-b_p^2$ respectively.

Now consider the $2k$ congruences
\[
\begin{aligned}
   u_1 = a_{p_i} u_3 + b_{p_i} u_4 \pmod{p_i}, \\
   u_2 = b_{p_i} u_3 - a_{p_i} u_4 \pmod{p_i}.
\end{aligned}
\]
By \propref{FF:Lattice:Prop}, the zeroes of these equations form a
lattice of determinant $p_1^2 \cdots p_k^2 = m^2$.

The symmetric \key{convex body} $C: x_1^2 + x_2^2 + x_3^2 + x_4^2 < 2m$ has
volume 
\[
\frac{1}{2} \pi^2 (2m)^2 = 2\pi^2 m^2 > 2^4 m^2,
\]
so by \propref{Minkowski:Convex:Prop} there is a non-trivial point in
the lattice inside $C$, which we denote by $(u_1, u_2, u_3, u_4)$.  We
claim that $m$ divides $u_1^2 + u_2^2 + u_3^2 + u_4^2$.  For each $p_i$:
\[
\begin {aligned}
  u_1^2 + u_2^2 + u_3^2 + u_4^2
    & = (a_{p_i} u_3 + b_{p_i} u_4)^2 + (b_{p_i} u_3 - a_{p_i} u_4)^2
+ u_3^2 + u_4^2 \\
    & = (a_{p_i}^2 + b_{p_i}^2 + 1) u_3^2 + 
        (a_{p_i}^2 + b_{p_i}^2 + 1) u_4^2
\end{aligned}
\]
which is clearly divisible by $p_i$, for each $i$.  Since
$u_1^2+u_2^2+u_3^2+u_4^2$ is both a multiple of $m$ and less than
$2m$, we must have
\[
u_1^2+u_2^2+u_3^2+u_4^2 = m.
\]
\end{proof}

\section*{Notes}

\small

\notesectref{Quadratic:Reciprocity:Sec} The proof given for quadratic
reciprocity in this section is due to {\Eisenstein}
\cite{Eisenstein:QR} and is one of the most elegant.  We have 
generally followed the presentation given by {\Ireland} and {\Rosen}
\cite{Ireland:Rosen}.

\notesectref{padic:Arith:Sec} $p$-adic numbers were first
introduced by {\Hensel} in a sequence of papers
\cite{Hensel97,Hensel02,Hensel04,Hensel05a,Hensel09} around the turn
of the century.  His definitive treatment of $p$-adic techniques is
given in Chapters 4 and 5 of \cite{Hensel13} and \cite{Hensel18}.
{\Hensel} had informally used these techniques as early as
\cite{Hensel87,Hensel88}.  A detailed collection of references on the
early history of $p$-adic numbers is given in \cite{Dickson23}.

Over the years the $p$-adic outlook has permeated large portions of
algebraic geometry and number theory and is now one of the
cornerstones of modern mathematics.

\normalsize
