%$Header: /usr/u/rz/AMBook/RCS/series.tex,v 1.1 1992/05/10 19:36:06 rz Exp rz $
\chapter{Background}
\label{Background:Chap}

\section{Asymptotic Estimates}
\label{Back:Asymp:Sec}

\begin{figure}
\begin{center}
\ForceWidth{3truein}\BoxedEPSF{AsympFun.eps}\end{center}
\caption{Behavior of $f_1(x)$ and $x^2$\label{AsympFun:Fig}}
\end{figure}


In numerous places we need to estimate the how functions behave as
their arguments grow.  Consider for instance, the function
\[
f_1(x) = x^5 e^{3-x} + x^2 \left(1 + \frac{\sin \sin x}{x}\right),
\]
which is plotted against $x^2$ (in grey) in \figref{AsympFun:Fig}.
For small values of $x$, $f(x)$ grows quite quickly, like $x^5$, but
for large values of $x$ it behaves more like $x^2$.  

This section briefly summarizes the different notations used to
quantify this behavior.  This notation was originally introduced by
{\LandauE} \cite{Landau:Primzahlen} and was extended by {\Knuth}
\cite{Knuth76b}.  A more detailed discussion of these concepts is
contained in \cite{Cormen91}.

We start by bounding the function from above.  We say that $f(x) =
o(g(x))$ if
\[
\overline{\lim_{x\rightarrow \infty}} \frac{f(x)}{g(x)} = 0,
\]
and that 
$f(x) = O(g(x))$ if 
\[
0 \le \overline{\lim_{x\rightarrow \infty}} \frac{|f(x)|}{g(x)} \le \infty.
\]
Notice that if $f = O(g)$ then $f = o(g)$.  In both of these cases,
$g$ is an upper bound on $f$, but $f=O(g)$ implies that $g$ may be a
``tight'' upper bound.  For instance, using the example function
$f_1$, we have
\[
f_1(x) = o(x^3), \quad f_1(x) \not= o(x^2) \quad\mbox{and}\quad
f_1(x) = O(x^2).
\]
\addsymbol{$f = O(g)$}{Tight asymptotic upper bound}
\addsymbol{$f = o(g)$}{Asymptotic upper bound equivalent}


The corresponding lower bounds on a function $f$ are indicated by 
\[
\begin{eqalign}
f = \Omega(g) &\Longleftrightarrow g = O(f)  \Longleftrightarrow 
0 \le \overline{\lim_{x\rightarrow \infty}} \frac{g(x)}{|f(x)|} \le \infty,\\
f = \omega(g) &\Longleftrightarrow g = o(f) \Longleftrightarrow 
\overline{\lim_{x\rightarrow \infty}} \frac{g(x)}{f(x)} = 0.
\end{eqalign}
\]
\addsymbol{$f = \Omega(g)$}{Tight asymptotic lower bound}
\addsymbol{$f = \omega(g)$}{Asymptotic lower bound equivalent}

Finally, we write $f = \Theta(g)$ if $f = O(g)$ and $f = \Omega(g)$.
\addsymbol{$f = \Theta(g)$}{Asymptotically equivalent}

\section{Algebra}
\label{Back:Algebra:Sec}

We say that a field $k$ is perfect if either the characteristic of $k$
is zero or $k^p = k$, where $p = \Char k$.  When the characteristic of
$k$ is positive, this means that every element of $k$ has a $p${\th}
root.  All finite fields are perfect, but $\F_p(T)$ is not.




