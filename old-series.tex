
As the title would indicate in this section we will describe the basic
technique that is used in determining the first few terms in the
Taylor series expansion of a function at a point.  For simplicity we
will assume that the point of expansion is the origin.  If it is not
then a simple linear translation will give a function whose expansion
at the origin is the same as the original function's at the original
point of expansion.  Thus if we wanted to expand $f(z)$ at $\alpha$ we
would instead expand $f(z + \alpha)$ to get
\[
f(z + \alpha)  =  a_0 + a_1 z + a_2 z^2 +\cdots
\]
and deduce that
\[
f(z) = a_0 + a_1 (z - \alpha) + a_2(z - \alpha)^2 +\cdots
\]

We will begin by noting that the elementary functions all have known power
series expansions.  A table of these expansions is contained in Appendix 3.
A number of these are quite easy to derive, for instance, let $f(z) = e^z$
then 
\begin{equation}
e^z = \sum_{k = 0}^{\infty} {e^0 \over k!} z^k 
= \sum_{k = 0}^{\infty} {z^k \over k!} 
\label{Exp:Series:Eq}
\end{equation}
since all the derivatives of the exponential function are $e^z$.
Similarly for $\sinh z$ we have 
\[
\begin{array}{cc}
  \sinh 0 = 1 & \frac{d}{dz} \sinh z = \cosh z \\
  \cosh 0 = 0 & \frac{d}{dz} \cosh z = \sinh z
\end{array}
\]
and thus we have
\[
\begin{eqalign}
\sinh z&= \sum_{k=0}^{\infty} {z^{2k+1} \over (2k+1)!}\\
\cosh z& = \sum_{k=0}^{\infty} {z^{2k} \over (2k)!}.
\end{eqalign}
\]
These power series could have also been determined in another way.  Recall
the definitions of $\sinh z$ and $\cosh z$.
\[
\sinh z = \frac{e^z + e^{-z}}{2} \qquad \cosh z = \frac{e^z - e^{-z}}{2}.
\]
Using \eqnref{Exp:Series:Eq} and these last equations we can also get
the power series expansions for $\sin z$ and $\cos z$.

In a similar way we can determine a number of terms of the expansion
$e^z \cos z$ by multiplying the series expansions for $e^z$ and 
$\cos z$ together 
\[
\begin{eqalign}
e^z \cos z &= \left( 1 + z + {z^2 \over 2} + {z^3 \over 6} + \cdots \right)
\left( 1 - {z^2 \over 2} + \cdots \right)\\
&= 1 + z - {z^3 \over 3} + \cdots,
\end{eqalign}
\]
which, if carried out to more terms, yields 
\[
e^z \cos z = 1 + z - {z^3 \over 3} - {z^4 \over 6}
- {z^5 \over 30} + {z^7 \over 630} + \cdots.
\]

As we shall see the sort of functions that we will be able to handle
by this technique are those that are constructed by elementary operations
from those functions for whom we already know power series expansions.
By an elementary operation we mean some operation on function(s) for
which we also know the operation on the corresponding power series'(s).
Since we know how to add power series, addition is an elementary operation.
In this section we will only be working with a finite portion of a
power series.  Thus we only have a limited amount of information about
a function and certain operations, such as substitution,
that seem ``elementary'' cannot be performed.

Recall that if a function takes on a finite, non-zero value at a 
point then we say it has {\em valence} zero.  The valence of a function
$f(z)$, at a point $p$, is written $\val_p f(z)$ or $\val f(z)$ if the
point is clear by context.  We will extend the definition of valence to that
of a truncated power series by letting the valence be the degree of the
lowest order non-zero term.  Let $\bar f$ be a the Taylor series expansion of
$f(z)$ at the origin.  The largest degree term of the Taylor series
expansion of $f(z)$ that can be determined from $\bar f$ is called the {\em
order} of $\bar f$.  The order of a {\em truncated} power series is a measure
of how much information is known about the function.  Even though we
have a power series expansion for 1 of order 1000000, and
999999 of them are zero, we still do not know that the function the
power series represents is a constant.


%%%%%%%%%%%%%%%%%%%%%%%%%%%%%%%%%%%%%%%%%%%%%%%%

Now, let's try to compute the power series expansion of 
$f(z) = \sin z e^z$ to order 3.  Proceeding as before, 
\[
\begin{eqalign}
f(z)& = \left( z - {z^3 \over 6} + \cdots \right)
\left( 1 + z + {z^2 \over 2} + {z^3 \over 6} + \cdots \right)\\
&= z + z^2 + {z^3 \over 3} + \cdots.
\end{eqalign}
\]
The first thing to notice is that we really don't need the $z^3$ term
of $e^z$.  It is possible to have the reverse of this behavior occur.
We may be in need of more terms than first thought, rather than have s surplus.
For instance consider $(e^z - 1)/ z$.  When expanding at the origin to
degree 3. 
\[
f(z) = {1 \over z}\, \left( z + {z^2 \over 2} + {z^3 \over 6} + \cdots
\right)
\]
Now we need another term from $e^z$, the $z^4$ term, in order
to determine the $z^3$ term correctly in the answer.  The solution
to this problem is stated in the following theorem. 
